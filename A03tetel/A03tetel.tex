\chapter{Terek \'es potenci\'alok}
 
 \section{Mechanika}
  
  \subsection{Kényszerfeltételek}\label{ss3:kenyszerfeletetelek}
   
   Egy rendszerre ható erők több csoportra bonthatóak:
   \begin{description}
    \item[Szabaderők] az olyan erők, melyek nem adnak megkötést a rendszer pontjainak elmozdulásaira.
    \item[Kényszererők] az olyan erők, amelyek irányába a tömegpont nem tud elmozdulni, van a rendszer fázisterében olyan tartomány, ami elérhetetlen. 
   \end{description}
   
   A kényszererőknek a leírás szempontjából több fajtája van.
   
   {\bf Holonom kényszerek:}
   A d'Alembert-elv alapján a rendszer a Newton-egyenletek szerint mozog, ha 
   \eq{
   \delta A=\suml{i=1}{3n}(X_i-m_i\ddtx_i)\delta x_i=0,
   }
   ahol $X_i$ a szabaderőknek megfelelő tagok.
   
   Holonom kényszerek esetén a konfigurációs térre való megszorítások egyenletek formájában megadhatóak.
   A kényszeregyenletek teljes differenciáját képezhetjük, és $\dd t$-t nullának választjuk, mert azonos időpillanatban keressük a virtuális elmozdulások közötti kapcsolatot:
   \aln{
    &\phi_j(x_1,x_2,\dots,x_{3n},t)=0\qquad\forall j=1\dots s &
    &\Rightarrow&
    &\dd\phi_j=\suml{i=1}{3n}\pder{\phi_j}{x_i}\dd x_i+ \pder{\phi_j}{t}\underbrace{\dd t}_{=0}=0.&\label{eq:03-holonomkenyszer}
   }
   A d'Alembert-elvben a kényszereket $s$ db.
   Lagrange-multiplikátoros taggal tudjuk figyelembe venni:
   \al{
   &\delta A=\sum\limits_{i=1}^{3n} \left(X_i -m_i\ddtx_i + \suml{j=1}{s}\lambda_j\pder{\phi_j}{x_i}\right)\delta x_i=0&
   &\Rightarrow&
   &X_i + \suml{j=1}{s}\lambda_j\pder{\phi_j}{x_i}=m_i\ddtx_i.&
   }
   Ez a $3n$ egyenlet az $s$ db kényszert leíró egyenlettel együtt egy megoldható egyenletrendszert ad a $3n+s$ db. ($\{x_i\}, \{\lambda_j\}$) ismeretlenre.
    
   {\bf Anholonom kényszerek:}
   Ebben az esetben a kényszer nem írható fel \eqaref{eq:03-holonomkenyszer} egyenlet formájában.
   Kényszer csak a virtuális elmozdulásokra adható meg.
   \footnote{Példa: guruló biciklikerék: fázistér: $(y,x,\alpha,\beta)$, ahol $(x,y)$ a kerék helye a síkon, $\alpha$ a kerék tengelyének és az $x$ tengelynek a bezárt szöge, illetve $\beta$ az egyik küllő síkkal bezárt szöge.
   A fázistér minden pontját el lehet érni, de $\delta x$ és $\delta \beta$ nem függetlenek. }
   Általános kapcsolat csak a teljes differenciálokra adható:
   \aln{
   &\suml{i=1}{3n}a_{ji}\der{x_i}{t}+a_{j0}=0 \qquad\forall j=1\dots s&
   &\Rightarrow&
   &\suml{i=1}{3n}a_{ji}\dd x_i+a_{j0}\underbrace{\dd t}_{=0}=0.&\label{eq:03-kenszerdef}
   }
   Így a virtuális elmozdulások:
   \eq{
   \suml{i=1}{3n}a_{ji}\delta x_i=0.
   }
   Most ezekkel a feltételekkel tudjuk kiegészíteni a d'Alembert-elvet, így:
   \al{
   &\delta A=\sum\limits_{i=1}^{3n} \left(X_i-m_i\ddtx_i + \suml{j=1}{s}\lambda_ja_{ji}\right)\delta x_i=0&
   &\Rightarrow&
   &X_i + \suml{j=1}{s}\lambda_ja_{ji}=m_i\ddtx_i .&
   }
   Ez az egyenletrendszer is a kényszereket leíró egyenletekkel együtt válik megoldhatóvá.
   
   Az általános koordinátákra való átérésnél is lehetnek a rendszerben anholonom kényszerek.
   Ekkor a $\delta q_i$ $i=1\dots f$-re nem lesz lineárisan független.
   A kényszerek egyenleteiből származtathatóak a fentiekhez hasonló Lagrange-multiplikátoros tagok, amelyekkel a $\delta q_i$-k már függetlenek lesznek, így
   \eq{
    Q_j+\suml{k=1}{s'}\lambda_k a_{kj}=\der{}{t}\pder{K}{\dtq_j}-\pder{K}{q_j}.
   }

   A kényszerek másik csoportosítása a következő.
   Tekintsük az elsőfajú Lagrange-egyenleteket anholonom és holonom $\left(a_{ji}=\pder{\phi_j}{x_i}\right.$, $\left.a_{j0}=\pder{\phi_j}{t}\right)$ rendszerekre együtt, bővítsük mindegyiket $\dtx_i$-vel, majd összegezzük őket  $i=1\dots 3n$-re:
   \al{
    \suml{i=1}{3n}m_i\ddtx_i\dtx_i
    &=\suml{i=1}{3n}\der{}{t}\left(\frac{1}{2}m_i\dtx_i^2\right)
    =\suml{i=1}{3n}X_i\dtx_i+\suml{i=1}{3n}\suml{j=1}{s}\lambda_j a_{ji}\dtx_i=\\
    &=\suml{i=1}{3n}X_i\dtx_i+\suml{j=1}{s}\lambda_j\underbrace{\suml{i=1}{3n} a_{ji}\dtx_i}_{=-a_{j0}}
    =\suml{i=1}{3n}X_i\dtx_i-\suml{j=1}{s}\lambda_ja_{j0},
   }
   ahol kihasználtuk a kényszerekre vonatkozó \eqref{eq:03-kenszerdef} egyenletet.
   Az utolsó egyenlet első tagja a szabaderők teljesítménye.
   Ha ezek potenciálosak, akkor 
   \al{
   &\suml{i=1}{3n}m_i\ddtx_i\dtx_i=-\der{}{t}U-\suml{j=1}{s}\lambda_ja_{j0}&
   &\Rightarrow&
   &\boxed{\der{}{t}(K+U)=-\suml{j=1}{s}\lambda_ja_{j0}}&
   }
   
   {\bf Szkleronom rendszer} az a holonom rendszer, ahol $\pder{\phi_j}{t}=0$ $\forall j=1\dots s$, azaz a kényszerfüggvények időfüggetlen, illetve az az anholononom rendszer, amelynél $\pder{a_{ji}}{t}=0$ és $a_{j0}=0$ $\forall i=1\dots 3n, j=1\dots s$. 
   
   {\bf Reonom rendszer} az olyan rendszer, amely nem szkleronom, vagyis ahol a kényszeregyenletek nem időfüggetlenek. 
   
   Szkleronom rendszerben a teljes energia megmarad.
   Reonom rendszerben általában nem.
   
  \subsection{Konzervatív erők}
   
   A Newton-egyenlet alapján:
   \al{
    &\vect{F}=\der{}{t}(m\dtr)&
    &\Rightarrow&
    &\dtr\vect{F}=m\ddtr\dtr=\der{}{t}\left(\frac{1}{2}m\dtr^2\right),&
   }
   melyet ha idő szerint integrálunk:
   \aln{
   &\intl{t_1}{t_2}\ddt\vect{F}\dtr=\left[\frac 12m\dtr^2\right]_{t_1}^{t_2}&
   &\Rightarrow&
   &\boxed{W=\Delta K}&\label{eq:03-munkatetel}
   }
   A bal oldalon definiáljuk az $\vect{F}$ erő (tömegponton végzett) munkáját ($W$).
   Ha az erő nem idő és sebességfüggő, azaz statikus, akkor a munkát átírhatunk vonalintegrál alakban: $W=\intl{\gamma}{}\dd\vect{s}\,\vect{F}(\vect{s})$.
   Ekkor az erő munkája csak a pályától függ.
   A jobb oldalon definiáljuk a kinetikus energiát: $K(t)=\frac{1}{2}m\dtr^2$.
   A kinetikus energia tömegpont pillanatnyi mozgásállapotától függ csak.
   Ez a munkatétel.
   
   \paragraph{A potenciálelmélet alaptétele}
   
    Egy statikus erőtér potenciálos, ha $\exists U(\vect{r})$ differenciálható függvény, hogy 
    \eq{\vect{F}=-\grad{U(\vect{r})}.}
    
    Egy statikus erőtér konzervatív, ha $\forall\gamma$ zárt pályára:
    \eq{
     \ointl{\gamma}{}\dd\vect{s}\,\vect{F}(\vect{s})=0.
    }
     
    Egy statikus erőtér örvénymentes egy egyszeresen összefüggő tartományon, ha ott
    \eq{\rot{\vect{F}(\vect{r})}=0.}
    
    A potenciálelmélet alaptétele pedig kimondja, hogy egy $\vect{F}$ statikus erőtér  potenciálos $\overset{1)}{\Leftrightarrow}$ konzervatív $\overset{2)}{\Leftrightarrow}$ örvénymentes. 
    
    \emph{Bizonyítás:}
    \begin{itemize}
     \item[$\overset{1)}{\Rightarrow}$] Tekintsünk egy tetszőleges zárt $\gamma$ görbét.
   Vágjuk el ezt valahol ($\gamma_0$), és vonalintegráljuk rajta az erőteret: $\ointl{\gamma}{}\dd\vect{s}\,\vect{F}(\vect{s})=-\ointl{\gamma}{}\dd\vect{s}\,\grad{U}(\vect{s})=-[U(\vect{s})]_{\vect{s}(\gamma_0)}^{\vect{s}(\gamma_0)}=U({\vect{s}(\gamma_0)})-U({\vect{s}(\gamma_0)})=0$. $\square$
     
     \item[$\overset{1)}{\Leftarrow}$] Állítás, hogy egy fix $\vect{r}_0$-at lerögzítve az $U(\vect{r})=-\intl{\vect{r}_0}{\vect{r}}\dd\vect{s}\,\vect{F}(\vect{s})$ választás megfelelő lesz.
   Lássuk be, hogy ennek a negatív gradiense tényleg az erőteret adja: 
     \eq{
      U(\vect{r}+\Delta\vect{r})-U(\vect{r})=-\intl{\vect{r}}{\vect{r}+\Delta\vect{r}}\dd\vect{s}\,\vect{F}(\vect{s}).
     }
     Paraméterezzük az integrált: $\vect{s}(t):=\vect{r}+t\cdot\Delta\vect{r}$, hogy $t=0\dots 1$.
   Az integrál ekkor $-\intl{0}{1}\dd t\,\vect{F}(\vect{r}+t\Delta\vect{r})\Delta\vect{r}$, ami az integrálszámítás alaptétele szerint egy $\theta\in[0,1]$-re:
     \eq{
      U(\vect{r}+\Delta\vect{r})-U(\vect{r})=-\intl{0}{1}\dd t\,\vect{F}(\vect{r}+t\Delta\vect{r})\Delta\vect{r}=-\vect{F}\big(\vect{r}+\theta\Delta\vect{r}\big)\Delta\vect{r}=-\vect{F}(\vect{r})\Delta\vect{r}+\ep(\vect{r},\Delta\vect{r}).
     }
     Ez éppen a derivált definíciója, itt $\ep(\vect{r},\Delta\vect{r})={o}(\Delta\vect{r}^2)$.
     
     A konzervatívságot ott használtuk fel, hogy csak akkor egyértelmű a potenciálfüggvény.
   Ha nem lenne konzervatív az erőtér, akkor a potenciálfüggvény értéke függene az integrálási úttól. $\square$
     
     \item[$\overset{2)}{\Rightarrow}$] Indirekt módon bizonyítjuk.
   Tegyük fel, hogy $\vect{r}_0$-ban $\rot{\vect{F}(\vect{r_0})}\neq 0$.
   Ekkor legyen $\gamma$ egy olyan kis zárt görbe, melynek középpontja $\vect{r}_0$.
   A feltétel szerint $0=\ointl{\gamma}{}\dd\vect{s}\,\vect{F}(\vect{s})$.
   Ezt átalakítva a Stokes-tétellel: $0=\intl{\text{int }\gamma}{}\df\rot\vect{F}(\vect{r})$.
   Tartsunk a görbe sugarával nullához, majd paraméterezzük meg a felületet: $0=\iint\limits_{T}\dd u\dd v\,\rot\vect{F}\big(\vect{r}(u,v)\big)\cdot(\vect{r}_u\times\vect{r}_v)$, ahol a $(\vect{r}_u\times\vect{r}_v)$ a felületelem-vektort jelöli.
   Az integrandusban szerepel két vektor, és mind a kettő arányos a $\rot{\vect{F}(\vect{r}_0)}$-lal, így egy szigorúan pozitív mennyiséget integrálunk, mely így nagyobb mint nulla.
   Ez ellentmondás. $\square$
     
     \item[$\overset{2)}{\Leftarrow}$] Vegyünk egy tetszőleges egyszeresen összefüggő felületet: $A$, ennek határa pedig $\partial A$.
   Minden zárt görbe felírható egy egyszeresen összefüggő felület határaként.
   A Stokes-tétel alapján azonnal látszik, hogy $0=\intl{F}{}\df\rot\vect{F}=\ointl{\partial F}{}\dd\vect{s}\,\vect{F}(\vect{s})$. $\square$
    \end{itemize}
    
   A munkatétel konzervatív erőtérre:
   \eq{
    K_2-K_1=\intl{\vect{r}_1}{\vect{r}_2}\dd\vect{r}\,\vect{F}(\vect{r})=\intl{\vect{r}_0}{\vect{r}_2}\dd\vect{r}\,\vect{F}(\vect{r})-\intl{\vect{r}_0}{\vect{r}_1}\dd\vect{r}\,\vect{F}(\vect{r})=-U(\vect{r}_2)+U(\vect{r}_1),
   }
   azaz a kinetikus és a potenciális energia összege állandó:
   \eqn{
    \der{}{t}(K+U)=0.
   }
   
  \subsection{Ponttöltés elektromágneses térben}\label{ss:03-toltesEMterben}
   
   Az $U=q\phi-q\vect{A}\vect{v}$ potenciállal előállítható az elektromágneses térben mozgó részecske mozgásegyenlete.
   A Lagrange:
   \eq{
    L=K-U=\frac{1}{2}m\vect{v}^2-q\phi+q\vect{A}\vect{v},
   }
   ahonnan a Lagrange-egyenlet:
   \al{
    0&=\der{}{t}\pder{L}{v_i}-\pder{L}{x_i}
      =\der{}{t}\left(mv_i+qA_i\right)-\left(-q\der{\phi}{x_i}+\suml{j=1}{3}qv_j\der{A_j}{x_i}\right)\\
     &=ma_i-q\left(-\pder{A}{t}-\der{\phi}{x_i}\right)-q\suml{j=1}{3}v_j\left(\der{A_j}{x_i}-\pder{A_i}{x_j}\right),
   }
   A összefoglalva pedig a három komponenst, és felhasználva, hogy $\vect{E}=-\partial_t\vect{A}-\grad{\phi}$ és $\vect{B}=\rot{A}$:
   \eq{
    m\ddot{\xv}=q\vect{E}+q\vect{v}\times\vect{B},
   }
   ami megfelel a Lorentz-erővel felírt Newton-egyenletnek. 
   
   A Hamilton-függvényt is elkészíthetjük, $\pv=\pder{L}{\vv}=m\vv+q\Av$, így
   \al{
    H
     =\pv\vv-L
     =m\vv^2+q\Av\vv-\frac{1}{2}m\vect{v}^2+q\phi-q\vect{A}\vect{v}
     =\frac{1}{2}m\vect{v}^2+q\phi
     =\frac{1}{2m}\big(\pv-q\Av\big)^2+q\phi,
   }
   ahonnan a kanonikus egyenletek:
   \al{
    \dot p_i
     &=-\pder{H}{x_i}
     =-q\partial_i \phi+ \frac{q}{m} (p_j-qA_j) \partial_i A_j
     =-q\partial_i \phi+ q v_j \partial_i A_j\\
    \dot x_i
     &=\pder{H}{p_i}
      =\frac{1}{m} (p_i-q A_i)
      =v_i.
   }
   Az elsőt átalakítva:
   \al{
    \dot p_i
     &=\dd_t (mv_i+q A_i)
      =m \ddot{x}_i+q \partial_t A_i+q \partial_j A_i \dot{x}_j\\
    m \ddot{x}_i
     &=-q\partial_i \phi-q \partial_t A_i+ q v_j \partial_i A_j-q \partial_j A_i \dot{x}_j
      =q\underbrace{-\partial_i \phi - \partial_t A_i}_{E_i}+q v_j\underbrace{( \partial_i A_j- \partial_j A_i)}_{\ep_{ijk}B_k}\\
    m\ddot{\xv}&=q\vect{E}+q\vect{v}\times\vect{B}.
   }

 \section{Elektrodinamika}
  
  \subsection{Elektrosztatikus és mágneses skalárpotenciál}
   
   Tekintsük az időfüggetlen, áramokat, azaz a mágneses tér szempontjából forrásokat nem tartalmazó Maxwell-egyenleteket:
   \al{
   \divo{\vect{D}(\vect{r})}&=\rho(\vect{r}) &
       \divo\vect{B}(\vect{r})&=0 \\
       \rot{\vect{E}(\vect{r})}&=0 &
       \rot{\vect{H}(\vect{r})}&=0,
   }
   melyekhez a határfeltételek: $D_n$, $E_t$, $B_n$ és $H_t$ folytonosan megy át a határfelületeken.
   
   \paragraph{Lineáris anyag}
   
    A harmadik egyenletből következik, hogy az $\vect{E}$ tér potenciálos, azaz $\vect{E}=-\grad{\phi(\vect{r})}$.
   Ezt az első egyenletbe helyettesítve, homogén esetben a Poisson-egyenletet kapjuk:
    \eq{
     \Delta\phi(\vect{r})=-\frac{1}{\ep}\rho(\vect{r}).
    }
    A mágneses szektor egyenletei teljesen hasonlóak.
   A negyedik Maxwell-egyenlet alapján: $\vect{H}=-\grad{\phi_M}$, melyet a második egyenletbe helyettesítve:
    \eq{
     \Delta\phi_M=0.
    }
    Az egyenletekhez tartozó határfeltételeket a terekre vonatkozó egyenletekből kaphatjuk: $\phi$ és $\phi_M$, illetve $\ep\pder{\phi}{\vect{n}}$ és $\mu\pder{\phi}{\vect{n}}$ is folytonos a határon.
    
    A két egyenlet és a határfeltételeik ekvivalensek, ugyanolyan megoldási módszerek alkalmasak a kezelésükre.
    
   \paragraph{Nemlineáris anyag}
    
    A Maxwell-egyenleteket ki kell egészíteni a két anyagi egyenlettel: $(\vect{P}(\vect{E})$, $\vect{M}(\vect{H}))$.
   A 3.
   Maxwell-egyenlet szerint itt is van skalárpotenciál: $\vect{E}=-\grad{\phi}$, ezt pedig az első egyenletbe helyettesítve:
    \eq{
     \rho(\vect{r})=\divo{\left(\ep_0\vect{E}+\vect{P}\right)}=\divo{\left(-\ep_0 \grad{\phi}+\vect{P}\right)} = -\ep_0\Delta\phi+\divo{\vect{P}}
    }
    \eq{
     \Delta\phi=-\frac{1}{\ep_0}\rho-\divo{\big(\vect{P}(\vect{E})\big)}.
    }
    
    A mágneses esetben hasonlóan: $\vect{H}=-\grad{\phi_M}$, akkor
    \eq{
    0=\mu_0\divo{\left(\vect{H}+\vect{M}\right)}
     =\mu_0\divo{\left(-\grad{\phi_M}+\vect{M}\right)}
     =-\mu_0\Delta\phi_M+\mu_0\divo{\vect{M}}
    }
    Vagyis 
    \eq{
     \Delta\phi_M=\divo{\vect{M}(\vect{H})}.
    }
    
    Itt is ugyanolyanok a mágneses és az elektromos szektorra vonatkozó egyenletek.
   A határfeltételek megegyeznek a lineáris esettel.
    
  \subsection{A vektorpotenciál}
   
   Ha a Maxwell-egyenletek áramokat is tartalmaznak, akkor a mágneses skalárpotenciál nem értelmezhető.
   A 2. Maxwell egyenlet alapján azonban létezik $\vect{A}$, hogy $\vect{B}=\rot{\vect{A}}$.
   Lineáris és nemlineáris anyagban is:
   \eq{
   \vect{J}=\rot{\left(\frac{1}{\mu_0}\vect{B}-\vect{M}\right)}
           =\frac{1}{\mu_0}\rot{(\rot{\vect{A}})}-\rot{\vect{M}},
   }
   ahol pedig felhasználjuk, hogy $\divo{A}=0$.
   Ez a Coulomb-mérték, ez mindig megválasztható.
   Ekkor $\vect{A}$-ra a
   \eq{
    \Delta{\vect{A}}=-\mu_0\big(\vect{J}+\rot{\vect{M}(\vect{B})}\big)
   }
   egyenletet kapjuk.
   Ennek három komponense szintén megegyezik a fenti egyenletek struktúrájával.
   A határfeltételek: $\vect{A}$ és $\frac{1}{\mu}\rot{\vect{A}}$ folytonos a határon. 

  \subsection{Az elektrodinamika egyértelműsége}
   
   Az elektrosztatikai feladatok a Poisson-egyenlet megoldására redukálódnak.
   A megoldandó feladat általánosan:
   \al{
    &\Delta{\Psi}=-f&
    \text{ahol }\qquad \Psi(\vect{r})\Big|_\text{felületen}=\text{adott} \qquad \text{vagy} \qquad \pder{\Psi}{\vect{n}}\Bigg|_\text{felületen}=\text{adott}.
   }
   Az első a Dirichlet (fémes eset), a második pedig a Neumann (szigetelő) határfeltételnek felel meg. 
   
   A Poisson-egyenletre igaz az unicitás: ha $\exists\;\Psi_1,\Psi_2$, hogy $\Delta\Psi_1=\Delta\Psi_2=-f$, akkor $\Psi_1-\Psi_2=\text{Const.}$, és a határfeltételt is kielégítik.
   
   {\bf Bizonyítás:} A $K=\Psi_1-\Psi_2$ függvény megoldása a $\Delta K=0$ egyenletnek a $K=0$ vagy a $\pder{K}{\vect{n}}=0$ határfeltétellel.
   A határfeltétel miatt a határon a $K\grad{K}$ mennyiség mindenképpen eltűnik.
   Erre a Gauss-tétel:
   \al{
    0=\ointl{\partial V}{}\df K\grad{K}
     =\intl{V}{}\drh\divo{(K\grad K)}
     =\intl{V}{}\drh\left[(\grad K)^2+\underbrace{\Delta K}_{=0}\right]
     =\intl{V}{}\drh(\grad K)^2,
   }
   ahonnan $\grad{K}=0$, vagyis $K=\text{Const}$. $\square$

  \subsection{Megoldás Green-függvénnyel}
   
   A Poisson-egyenlet lineáris differenciálegyenlet, így megkereshető annak Green-függvénye, majd abból elkészíthető a teljes megoldás.
   A fenti Poisson-egyenlet Green-függvénye:
   \aln{
    &\Delta G(\vect{r},\vect{r}')=-\frac{1}{\ep_0}\delta(\vect{r}-\vect{r}')&
    &\Rightarrow&
    &\Psi(\vect{r})=\int\drkh G(\vect{r},\vect{r}')\rho(\vect{r}').&
   }
   Mivel az egyenlet szimmetrikus az $\vect{r}$ $\vect{r}'$ cserére, ezért $G(\vect{r},\vect{r}')=G(\vect{r}-\vect{r}')$.
   Az általános megoldás a Green-függvényből a fenti módon állítható elő:
   \al{
    \Delta\Psi(\vect{r})
    =\int\drkh \Delta G(\vect{r},\vect{r}')\rho(\vect{r}')
    =-\frac{1}{\ep_0}\int\drkh \delta(\vect{r},\vect{r}')\rho(\vect{r}')
    =-\frac{1}{\ep_0}\rho(\vect{r}).
   }
   A Green-függvényre érvényes határfeltételek a Green-tétel alapján kaphatóak.
   A Green-tétel kimondja két ($\psi,\phi$) skalárfüggvényre, és egy egyszeresen összefüggő $V$ térfogatelemre:
   \eq{
    \ointl{\partial V}{}\df \big(\phi\grad{\psi}-\psi\grad{\phi}\big)=\intl{V}{}\drh \big(\phi\Delta\psi-\psi\Delta\phi\big).
   }
   Ez alkalmazva a $\phi=\Psi$, $\psi=G(\vect{x},\vect{x}')$-re:
   \al{
   \ointl{\partial V}{}\dd^2\vect{f}_{\vect{r}}\,\left[\Psi(\vect{r})\grad{G(\vect{r},\vect{r}')}-G(\vect{r},\vect{r}')\grad{\Psi(\vect{r})}\right]
    &=\intl{V}{}\drh\left[\Psi(\vect{r})\underbrace{\Delta {G(\vect{r},\vect{r}')}}_{-\frac{1}{\ep_0}\delta(\vect{r}-\vect{r}')}-G(\vect{r},\vect{r}')\underbrace{\Delta{\Psi(\vect{r})}}_{-\frac{1}{\ep_0}\rho(\vect{r})}\right] \\
    &=-\frac{1}{\ep_0}\Psi(\vect{r}')+\frac{1}{\ep_0}\intl{V}{}\drh G(\vect{r},\vect{r}')\rho(\vect{r}).
   }
   \begin{description}
    \item[Dirichlet] határfeltételre választhatjuk a $G(\vect{r}\in\partial V,\vect{r}')\equiv 0$, ekkor az előző egyenlet alapján:
    \eqn{
     \Psi(\vect{r}')=\intl{V}{}\drh G(\vect{r},\vect{r}')\rho(\rv)-\ep_0\ointl{\partial V}{}\dd^2\vect{f}_{\vect{r}}\,\Psi(\vect{r})\grad{G(\vect{r},\vect{r}')}.\label{eq:3-dirichletmo}
    }
    \item[Neumann] határfeltételre pedig mivel $\ointl{\partial V}{}\dd^2\vect{f}_{\vect{r}}\,\grad{G(\vect{r},\vect{r}')}=\intl{V}{}\drh \Delta G(\vect{r},\vect{r}')=-\frac{1}{\ep_0}$, így $\grad{G(\vect{r},\vect{r}')}$-t nem választhatjuk nullának.
   Azonban konstansnak választhatjuk: 
    \eq{
     \grad{G(\vect{r}\in\partial V,\vect{r}')}=-\frac{1}{\ep_0\abs{\partial V}}.
    }
    Ezzel: 
    \eq{
     \Psi(\vect{r}')=\frac{1}{\abs{\partial V}}\ointl{\partial V}{}\df\Psi + \intl{V}{}\drh G(\vect{r},\vect{r}')\rho(\vect{r})+\ep_0\ointl{\partial V}{}\dd^2\vect{f}_{\vect{r}}\,G(\vect{r},\vect{r}')\grad{\Psi(\vect{r})},
    }
    ahol az első tag egy konstans. 
   \end{description}
   
  \subsection{Tükörtöltések}
   
   A Poisson-egyenletet megoldását (Green-függvényét) keressük Dirichlet (fémes) határfeltétel mellett.
   A Green-függvény tartalmazza a határfeltéteteket is azzokat megkeresni általában nagyon nehéz.
   Mi most nem szeretnénk közvetlenül megoldani az egyenletet, hanem intuícióval keressük a megoldást: a fizikailag elszeparált térrészekbe pontöltéseket helyezünk úgy, hogy ezek és a tényleges töltések tere együtt kiadja a keresett potenciált a fizikai térrészben, és az a határfeltételnek is megfeleljen. 
   
   \begin{description}
    \item[Fém síklap Green-függvénye]
     Ha van egy töltés a sík felett, akkor hogy konstans legyen a potenciál az egész felületen, akkor egy azzal ellentétes, ugyanakkora töltésnek kell lennie a sík mások felületén tükörszimmetrikusan.
   Legyen a töltésünk az $\vect{r'}=(x',y',z')$ helyen.
   A tükörtöltésnek ekkor a $(x',y',-z')$ helyen kell lennie.
   A potenciál az $\vect{r}$ helyen, azaz a Green-függvény:
     \eq{
      G(\vect{r}',\vect{r}) = \frac{1}{4\pi\ep_0}
       \left[\frac{1}{\sqrt{(x-x')^2+(y-y')^2+(z-z')^2}}-\frac{1}{\sqrt{(x-x')^2+(y-y')^2+(z+z')^2}}\right].
     }
    
    \item[Fémgömb Green-függvénye]
     Legyen a gömb az $R^2=x^2+y^2+z^2$ egyenlettel megadva.
   Legyen egy egységnyi töltés az $\vect{r}=(x',0,0)$ helyen, a tükörtöltés pedig a hengerszimmetria miatt az $(\bar{x},0,0)$ helyen.
   Ennek nagysága legyen $-q$.
   Ekkor a potenciál az $\vect{r}$ helyen:
     \eq{
      \phi(\vect{r}) = \frac{1}{4\pi\ep_0}
       \left[\frac{1}{\sqrt{(x-x')^2+y^2+z^2}}-\frac{q}{\sqrt{(x-\bar{x})^2+y^2+z^2}}\right].
     }
     Azt szeretnénk, hogy $\phi(x,y,z)\equiv 0$, ha $R^2=x^2+y^2+z^2$, ehhez keresünk megfelelő $q$ és $\bar{x}$ paramétereket.
   A feltétel alapján:
     \al{
       q^2 \big[(x-x')^2+y^2+z^2\big]&=(x-\bar{x})^2+y^2+z^2 \\
      R^2&=x^2+y^2+z^2.
     }
     A második egyenletet felhasználva az első:
     \al{
      &q^2 \big[x'^2-2xx'+R^2\big]=\bar{x}^2-2x\bar{x}+R^2 &
      &\Rightarrow&
      &0=2x(q^2x'-\bar{x})+\bar{x}^2-q^2x'^2+(1-q^2)R^2,&
     }
     mely akkor igaz minden $x$-re, ha $\bar{x}=q^2x'$ és 
     \al{
      0=q^4x'^2-q^2x'^2+(1-q^2)R^2=(1-q^2)(R^2-q^2x'^2).
     }
     Az egyik megoldás a $q=1$, ám ekkor a két töltés ugyanott van, nincs tér.
   A másik, fizikai megoldás: $q=\frac{R}{x'}$, illetve $\bar{x}=\frac{R^2}{x'}$.
   Ezekkel tehát a Green-függvény:
     
     \eq{
      G(\vect{r}',\vect{r})=
      \frac{1}{4\pi\ep_0}\left[\frac{1}{\sqrt{\left(x-x'\right)^2+y^2+z^2}}-\frac{\frac{R}{x'}}{\sqrt{\left(x-\frac{R^2}{x'}\right)^2+y^2+z^2}}\right],
     }
     ami feltűnően gömbszimmetrikus és $\vect{r}$, $\vect{r}'$ szimmetrikus formában:
     \eq{
      G(\vect{r}',\vect{r})=
      \frac{1}{4\pi\ep_0}\left[\frac{1}{\abs{\vect{r}-\vect{r}'}}-\frac{1}{\sqrt{\frac{\vect{r}^2\vect{r}'^2}{R^2}-2\vect{r}\vect{r}'+R^2}}\right].
     }
   \end{description}
   
 \section{Kvantummechanika}
  
  \subsection{Schrödinger-egyenlet elektromágneses tér esetében}
   
   A Hamilton-operátort ekkor a kinetikus impulzus operátorával írhatjuk fel:
   \eq{
    \op{H}=\frac{\op{\vect{k}}^2}{2m}+\op{U}.
   }
   Itt koordinátareprezentációban $\op{\vect{k}}=\frac{\hbar}{i}\grad-q\vect{A}(\vect{r})$ és $\op{U}=q\phi(\vect{r})$, így 
   \al{
    \op{H}&=\frac{1}{2m}\left[\op{\vect{p}}-q\vect{A}\right]^2+q\phi 
     = \frac{\op{\vect{p}}^2}{2m}-\frac{q}{2m}[\op{\vect{p}}\vect{A}+\vect{A}\op{\vect{p}}]+\frac{q^2}{2m}\vect{A}^2+q\phi, 
   }
   ahol, ha Coulomb-mértéket használunk ($\divo{\vect{A}}=0$):
   \eq{
    \op{H}= \frac{\op{\vect{p}}^2}{2m}-\frac{q}{m}\vect{A}\op{\vect{p}}+\frac{q^2}{2m}\vect{A}^2+q\phi=\underbrace{-\frac{\hbar^2}{2m}\Delta+q\phi}_{\op{H}_0}+\underbrace{\frac{q\hbar i}{m}\vect{A}\grad{}}_{\op{H}\text{para}}+\underbrace{\frac{q^2}{2m}\vect{A}^2}_{\op{H}_\text{dia}}.
   }
   
   Tekintsünk egy speciális esetet: homogén esetben jó Coulomb-mértéket a szimmetrikus mérték: $\vect{A}=\frac 12\vect{B}\times\vect{r}$, illetve legyen a részecske elektron: $q=-e$.
   Ezzel a paramágneses tag:
   \eq{
    \op{H}_\text{para}
    =\frac{q\hbar i}{m}\vect{A}\grad{}
    =\frac{q\hbar i}{2m}(\vect{B}\times\vect{r})\grad{}
    =\frac{q\hbar i}{2m}(\vect{r}\times \grad{})\vect{B}
    =-\frac{q}{2m}\left(\vect{r}\times \op{\vect{p}}\right)\vect{B}
    =-\frac{q}{2m}\op{\vect{L}}\vect{B}
    =\underbrace{\frac{e\hbar}{2m}}_{\mu_\text{B}}\frac{1}{\hbar}\op{\vect{L}}\vect{B}.
   }
   Itt definiáltuk a Bohr-magnetont: $\mu_\text{B}=9.27\cdot 10^{-24}\me{J/T}$.
   A fenti Hamilton-operátort kiegészíthetjük a Zeemann-taggal, így kapjuk a Pauli paramágneses tagot:
   \eq{
    \op{H}_\text{para}=\mu_\text{B}\frac{1}{\hbar}(\op{\vect{L}}+2\op{\vect{S}})\vect{B}.
   }
   
   A Langevin diamágneses tag:
   \eq{
    \op{H}_\text{dia}
     =\frac{q^2}{2m}\vect{A}^2
     =\frac{q^2}{8m}(\vect{B}\times \vect{r})^2
     =\frac{q^2}{8m}\big(r^2B^2-(\vect{r}\vect{B})^2\big)
     =\frac{q^2B^2}{8m}\vect{r}_\perp^2,
   }
   ahol $\vect{r}_\perp$ az origón átmenő, a mágneses tér irányával párhuzamos tengelytől mért távolság. 
   
   Ezzel tehát a Hamilton-operátor nemrelativisztikus esetben, $z$ irányú mágneses térre:
   \eqn{
    \boxed{\op{H}=-\frac{\hbar^2}{2m}\Delta+q\phi+\mu_\text{B}\frac{1}{\hbar}(\op{L}^z+2\op{S}^z)B+\frac{q^2B^2}{8m}(x^2+y^2)}\label{eq:03-SCHem}
   }
   
   Vizsgáljuk meg az egyes tagok nagyságrendjeit: A Pauli-féle tag $\sim 10^{-23}\cdot 1\cdot 1=10^{-23}\me{(J)}$, míg a Langevin-féle tag: $\sim \frac{10^{-2\cdot 19} 1^2}{10^{-30}}\cdot 10^{-2\cdot 10}=10^{-28}\me{(J)}$, vagyis a diamágneses tag 5 nagyságrenddel kisebb a paramágnesesnél, annak csak akkor van jelentősége, ha a paramágneses tag nulla, vagyis ha $(\op{L}^z+2\op{S}^z)=0$.
   
   A paramágneses tag a már meglévő $\vects{\mu}_\text{para}=-\pder{H_\text{para}}{\vect{B}}=-\mu_\text{B}\frac{1}{\hbar}(\op{\vect{L}}+2\op{\vect{S}})$ momentumot akarja a mágneses tér irányába forgatni ($\op{H}_\text{para}=-\vects{\mu}_\text{para}\vect{B}$).
   A diamágneses tagot a mágneses tér indukálja: $\vects{\mu}_\text{dia}=-\frac{e^2B}{4m}(x^2+y^2)\vect{e}_z$.
   Ez arányos $B$-vel, és mindig a mágneses térrel ellentétes irányba mutat, ezért diamágneses.
   
  \subsection{Időfüggő terek, indukált emisszió, abszorpció}

   Legyen egy kvantumrendszer, melynek időbeli fejlődését a $\op{H}_0$ Hamiltonnal tudjuk leírni.
   Tegyük ezt a rendszert egy időfüggő térbe, amellyel a $\op{V}(t)$ operátoron keresztül hat kölcsön.
   Legyen a rendszer kezdetben az $i$-edik állapotában.
   Az időfüggő perturbációszámítás szerint ekkor első rendben annak a valószínűsége, hogy a rendszer a $t$ idő múlva az $f$-edik állapotában van:
   \eq{
    W^{(1)}(i\to f)=\frac{1}{\hbar^2}\abs{\intl{0}{t}\dd\tau\,\bra{f}V\ket{i}e^{\frac{i}{\hbar}(\epsilon_f-\epsilon_i)\tau}}^2
   }
   
   Alkalmazzuk ezt elektromos térre: legyen a gerjesztő jel fény, ahol $\vect{E}=E_0\vect{e}_x\sin{(\omega t)}$.
   Innen $V(t)=q\phi=eE_0x\sin{(\omega t)}$, vagyis:
   \al{
    W^{(1)}(i\to f)
     &=\frac{1}{\hbar^2}\abs{\intl{0}{t}\dd\tau\,\bra{f}eE_0x\sin{(\omega \tau)}\ket{i}e^{\frac{i}{\hbar}(\epsilon_f-\epsilon_i)\tau}}^2 
     =\frac{e^2E_0^2}{\hbar^2}\abs{\intl{0}{t}\dd\tau\,\bra{f}x\ket{i}\sin{(\omega \tau)}e^{\frac{i}{\hbar}(\epsilon_f-\epsilon_i)\tau}}^2 \\
     &=\frac{e^2E_0^2}{\hbar^2}\abs{\bra{f}x\ket{i}}^2\abs{\intl{0}{t}\dd\tau\,\sin{(\omega \tau)}e^{\frac{i}{\hbar}(\epsilon_f-\epsilon_i)\tau}}^2\\
     &=\frac{e^2E_0^2}{4\hbar^2}\abs{\bra{f}x\ket{i}}^2\abs{\intl{0}{t}\dd\tau\,\left(e^{\left[\frac{i}{\hbar}(\epsilon_f-\epsilon_i)+\omega\right]\tau}-e^{\left[\frac{i}{\hbar}(\epsilon_f-\epsilon_i)-\omega\right]\tau}\right)}^2 \\
     &=\frac{e^2E_0^2}{4\hbar^2}\abs{\bra{f}x\ket{i}}^2\abs{\frac{e^{\left[\frac{i}{\hbar}(\epsilon_f-\epsilon_i)+\omega\right]t}-1}{\frac{i}{\hbar}(\epsilon_f-\epsilon_i)+\omega}-\frac{e^{\left[\frac{i}{\hbar}(\epsilon_f-\epsilon_i)-\omega\right]t}-1}{\frac{i}{\hbar}(\epsilon_f-\epsilon_i)-\omega}}^2.
   }
   A prefaktor az elektromos tér erősségétől függ.
   Az $\abs{\bra{f}x\ket{i}}^2$ tag értékét a kiválasztási szabályok határozzák meg.
   Ha ez nem nulla, akkor megengedett az $i\to f$ átmenet.
   A kiintegrált részben a számlálók korlátos függvények, a nevezők azonban lehetnek nagyon kicsik, akkor pedig az átmeneti valószínűség nagyon nagy.
   Ez akkor történik meg, ha 
   \al{
    &\epsilon_f=\epsilon_i+\hbar\omega
    &\epsilon_f=\epsilon_i-\hbar\omega,
   }
   ami éppen megfelel a Bohr-féle frekvenciafeltételnek.
   A kezdeti és a végállapot között éppen egy $\hbar\omega$ energiájú fotont abszorbeált vagy emittál a rendszer.
