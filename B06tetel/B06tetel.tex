\chapter{Klasszikus g\'azok: ide\'alis g\'az, ekvipart\'{\i}ci\'o, Maxwell-eloszl\'as, viri\'al sorfejt\'es, van der Waals \'allapotegyenlet} 
 
 \section{Ekvipartíció tétele}
  
  Tekintsünk egy klasszikus rendszert.
   Ennek Hamilton-függvénye $H=H(q,p)$.
   Jelölje $x_i$ a Hamilton-függvény tetszőleges változóját, $q_i$-t vagy akár $p_i$-t is.
   Az $x_i$-t szabadsági foknak hívjuk, ha 
  \al{
   \lim_{x_i\to \pm\infty}H(\dots,x_i,\dots)=\infty.
  }
  Fontos, hogy mind a két határértékben divergálnia kell a Hamilton-függvénynek.
   Az ekvipartíció tétele kimondja, hogy ha $x_i$ szabadsági fok, akkor 
  \al{
   \mv{x_j\pder{H}{x_i}}=\delta_{ij}\kB T.
  }
  Bizonyítás:
  \al{
   \mv{x_j\pder{H}{x_i}}
    &=\frac{1}{Z}\intl{}{}\frac{\dd^{dN}p\dd^{dN}q}{N! h^{dN}}\,e^{-\beta H}x_j\pder{H}{x_i},
  }
  ahol végezzük el külön az $x_i$-re való integráli:
  \al{
   \intl{}{}\dd x_i\,e^{-\beta H}x_j\pder{H}{x_i}
    &=-\frac{1}{\beta}\intl{}{}\dd x_i\,x_j\pder{e^{-\beta H}}{x_i}
     =\{\text{parc. int.}\}\\
    &=-\frac{1}{\beta}\left(\left[x_j e^{-\beta H}\right]_{x_i=-\infty}^{x_i=\infty}-\intl{}{}\dd x_i\,\pder{x_j}{x_i}e^{-\beta H}\right)
     =\frac{1}{\beta}\intl{}{}\dd x_i\,\underbrace{\pder{x_j}{x_i}}_{\delta_{ij}} e^{-\beta H},
  } 
  így
  \al{
   \mv{x_j\pder{H}{x_i}}
    &=\frac{1}{Z}\frac{1}{N! h^{dN}}\intl{}{}\dd^{dN}p\dd^{dN}q\,\frac{1}{\beta}\pder{x_j}{x_i}e^{-\beta H}
     =\frac{1}{\beta}\delta_{ij}\frac{1}{Z}\underbrace{\frac{1}{N! h^{dN}}\intl{}{}\dd^{dN}p\dd^{dN}q\,e^{-\beta H}}_{=Z}
     =\kB T\delta_{ij}.
  }
  
  Még egyszer, ez csak klasszikus rendszerekben igaz.
   Alacsony hőmérsékleten nem ugyanakkora energia jut az egyes szabadsági fokokra, vannak szabadsági fokok, amelyek kifagynak.
   Pl. a Doulong--Petit-törvény szerint a harmonikus oszcillátor fajhője állandó, míg a kvantumos levezetésből alacsony hőmérsékleten $\sim T^2$ adódik. 
  
 \section{Klasszikus ideális gáz}\label{ss:B06-CID}
  
  Ideális klasszikus gáznak tekintjük azt a rendszert, amely
  \begin{itemize}
   \item klasszikus (nem kvantumos) fizikával leírható,
   \item a részecskék pontszerűnek tekinthetőek (térfogatuk elhanyagolható),
   \item a falakkal tökéletesen rugalmasan ütköznek a részecskék (energia megmarad),
   \item a részecskék közötti kölcsönhatást csak olyan értelemben vesszük figyelembe, hogy a rendszer termalizálódik, egyéb effektusokkal nem számolunk.
  \end{itemize}
  
  \subsection{Állapotszám}
  
   Legyen $N$ részecske a $d$ dimenziós, $V$ térfogatot kitöltő klasszikus ideális gázban.
   Mikrokanonikus sokaságként kezelve az állapotszám:
   \al{
    \Omega_{0}(E)
     =\intl{\suml{i=1}{N}\frac{p_i^2}{2m}<E}{}\frac{\dd^{dN}p\dd^{dN}q}{N! h^{dN}}\,
     =\frac{V^{N}}{N! h^{dN}}\underbrace{\intl{p_1^2+p_2^2+\dots<2mE}{}\,\dd^{dN}p}_{\substack{\text{$dN$ dimenziós $\sqrt{2mE}$}\\ \text{sugarú gömb térfogata}}}
     =\frac{V^{N}}{N! h^{dN}}\frac{(2mE)^{\frac{dN}{2}}\pi^{\frac{dN}{2}}}{\Gamma\left(\frac{dN}{2}+1\right)},
   }
   ahol $\Gamma$ a Gamma-függvény (a faktoriális általánosítása).
   
   Az entrópiához szükségünk van ennek a logaritmusára.
   A Stirling-formulát ($\ln n!\approx n\ln n-n$) használva, illetve a Gamma-függvényt közelítve ($\Gamma(n+1)\sim n!$): 
   \al{
    \ln{\Omega_0(E)}
     &\approx N\ln V-(N\ln N-N)-dN\ln h+\frac{dN}{2}\ln(2mE\pi)-\left(\frac{dN}{2}\ln\frac{dN}{2}-\frac{dN}{2}\right)+\mathcal{O}(\ln N)\\
     &=N\ln V-(N\ln N-N)+\frac{dN}{2}\ln\left(\frac{4mE\pi}{dNh^2}\right)+\frac{dN}{2}+\mathcal{O}(\ln N)\\
     &=\frac{2+d}{2}N+\frac{dN}{2}\ln\left(\frac{4m\pi}{dh^2}\frac{E}{N}\left(\frac{V}{N}\right)^{\frac{2}{d}}\right)+\mathcal{O}(\ln N)\\
     &=N\underbrace{\left[\frac{2+d}{2}+\frac{d}{2}\ln\left(\frac{4m\pi}{dh^2}\frac{E}{N}\left(\frac{V}{N}\right)^{\frac{2}{d}}\right)\right]}_{=\phi\left(\frac{E}{N},\frac{V}{N}\right)}+\mathcal{O}(\ln N).
   }
   Láthatjuk, hogy ez egy normál rendszer \eqaref{eq:B03-Omega--N} egyenlet alapján. 
   
  \subsection{Maxwell-féle sebességeloszlás}
   
   Írjuk le az ideális gázt kanonikus sokaságban.
   A Hamilton-függvény $H=\suml{i=1}{N}\frac{p_i^2}{2m}$.
   Mivel a Hamilton kölcsönhatásmentes, így a kanonikus állapotszám:
   \al{
    Z&=\frac{Z_1^N}{N!}\\
    Z_1&=\frac{1}{h^3}\intl{}{}\dd^3 \qv\dd^3 \pv\,e^{-\beta\frac{p^2}{2m}}
        =\frac{V}{h^3}\intl{}{}\dd^3 \pv\,e^{-\beta\frac{\pv^2}{2m}}
       =\frac{V}{h^3}\left(\frac{2m\pi}{\beta}\right)^{\frac{3}{2}}
        =\frac{V}{h^3}\left(2\pi m\kB T\right)^{\frac{3}{2}}.
   }
   Annak a valószínűsége, hogy egy részecske impulzusa $\pv$:
   \al{
    P(\pv)
     =\frac{\frac{1}{h^{3N}}\intl{}{}\dd^{3N} q\dd^{3(N-1)} p\,e^{-\beta\suml{i=1}{N}\frac{p_i^2}{2m}}}{Z}
     =\frac{\frac{V}{h^3}e^{-\beta\frac{\pv^2}{2m}}}{Z_1}
     =\frac{e^{-\beta\frac{\pv^2}{2m}}}{\left(2\pi m\kB T\right)^{\frac{3}{2}}}.
   }
   Az előző állítás nem csak kölcsönhatásmentes rendszerekre igaz, hanem olyanokra is, ahol a kölcsönhatás csak a $q$-tól függ.
   Az előző képletben a számlálóban és a nevezőben ugyanaz a térintegrál lenne, így egyszerűsíteni lehetne vele.
   
   Az impulzus eloszlásából kifejezhetjük, hogy mennyi a valószínűsége, hogy a részecske sebessége $\vv$, hiszen:
   \al{
    &P(\vv)\dd^3\vv=P(\pv)\dd^3\pv
    &\pv=m\vv
    &&P(\vv)=m^3 P(\pv)=\left(\frac{m}{2\pi \kB T}\right)^{\frac{3}{2}}e^{-\frac{m\vv^2}{2\kB T}}.
   }
   A sebesség abszolút értékének eloszlása a Maxwell-féle sebességeloszlás:
   \al{
    &\dd^3\vv=4\pi v^2\dd v
    &P(v)
     =4\pi v^2 P(\vv)
   }
   \aln{
     \boxed{P(v)=4\pi v^2\left(\frac{m}{2\pi \kB T}\right)^{\frac{3}{2}}e^{-\frac{m v^2}{2\kB T}}}.\label{eq:B06-Maxwell}
   }
   
   Az eloszlás maximumhelye:
   \al{
    P'(v)
     &=0
      =\der{}{v}\left(-\frac{m v^2}{2\kB T}+2\ln v\right)
      =-\frac{m v}{\kB T}+2\frac{1}{v}
     &v_\text{max}=\sqrt{\frac{2\kB T}{m}},
   }
   várható értéke és négyzetes közepe:
   \al{
    \mv{v}
     &=\intl{0}{\infty}\dd v\, vP(v)
      =\intl{0}{\infty}\dd v\, 4\pi v^3\left(\frac{m}{2\pi \kB T}\right)^{\frac{3}{2}}e^{-\frac{m v^2}{2\kB T}}
      =\sqrt{\frac{8\kB T}{\pi m}}\\
    \mv{v^2}
     &=\frac{2}{m}\frac{3}{2}\kB T
      =\frac{3\kB T}{m}.
   }
   Itt messze nem igaz, hogy az eloszlás éles lenne, egy részecske sebességének az eloszlása igen széles tartományban van. 
   
   Az energia szerinti eloszláshoz:
   \al{
    &\dd E=mv\dd v
    &E=\frac{1}{2}mv^2
    &&P(E)=\frac{1}{mv}P(v)
   }
   \al{
    P(E)
     &=\frac{1}{m}4\pi v\left(\frac{m}{2\pi \kB T}\right)^{\frac{3}{2}}e^{-\frac{m v^2}{2\kB T}}
      =\frac{1}{m}4\pi \sqrt{\frac{2E}{m}}\left(\frac{m}{2\pi \kB T}\right)^{\frac{3}{2}}e^{-\frac{E}{\kB T}}
      =2\pi \sqrt{E}\left(\frac{1}{\pi \kB T}\right)^{\frac{3}{2}}e^{-\frac{E}{\kB T}}.
   }
   Ennek maximumhelye és várható értéke:
   \al{
    \der{}{E}P(E)&=0
      =\der{}{E}\left(-\frac{E}{\kB T}+\frac{1}{2}\ln E\right)
      =-\frac{1}{\kB T}+\frac{1}{2E}
     & E_\text{max}=\frac{1}{2}\kB T\\
    \mv{E}&=\frac{3}{2}\kB T.
   }
   
   Az energia szórása:
   \al{
    \mv{(\Delta E)^2}
     &=\mv{E^2}-\mv{E}^2
      =\pder{}{\beta}\pder{\ln Z_1}{\beta}
      =\frac{3}{2}(\kB T)^2
      =\frac{2}{3}\mv{E}^2\\
     \frac{\sqrt{\mv{(\Delta E)^2}}}{\mv{E}}
      &=\sqrt{\frac{2}{3}}\sim\mathcal{O}(1),
   }
   ami egy részecskére igen jelentős, de az $N$ részecskéből álló gázra, ahol
   \al{
    \ln Z
     &=\ln\left(\frac{Z_1^N}{N!}\right)
      \approx -N\ln N+N+N\ln Z_1
      =-N\ln N+N+N\ln \left(\frac{V}{h^3}\left(2\pi m\kB T\right)^{\frac{3}{2}}\right)\\
    \mv{(\Delta E)^2}
     &=\pder{}{\beta}\pder{\ln Z}{\beta}
      \sim N\\
    \frac{\sqrt{\mv{(\Delta E)^2}}}{\mv{E}}&\sim\frac{1}{\sqrt{N}}.
   }
   
   Az ideális gáz állapotegyenletét megadhatjuk pl.\ az $F$ termodinamikai deriváltjával:
   \al{
    p&=-\pder{F}{V}
      =-\pder{}{V}\left(-\kB T \ln Z\right)
      =-\pder{}{V}\left(-\kB T N\ln \left(\frac{V}{N h^3}\left(2\pi m\kB T\right)^{\frac{3}{2}}\right)\right)\\
     &=\frac{1}{V}\kB T N\\
    pV&=N\kB T.
   }
   
 \section{Viriál tétel}
  
  Tegyük fel, hogy a klasszikus háromdimenziós rendszerünk Hamilton-függvénye a következő:
  \al{
   H=E_\text{kin}+U_\text{pot}=E_\text{kin}+U_\text{int}+U_\text{w},
  }
  ahol a kölcsönhatást két részre osztottuk, a részecskék egymással való kölcsönhatására és a fallal való kölcsönhatásra.
   Azt tudjuk az ekvipartíció tétele miatt, hogy $\mv{E_\text{kin}}=\frac{3}{2}\kB T$.
   Kérdés, hogy mennyi a potenciális energia átlagértéke. 
  
  A Newton-egyenletből:
  \al{
   m_i \ddot{x}_i=F_i=-\pder{U_\text{pot}}{x_i}=-\pder{H}{x_i}.
  }
  Ez beszorozva $x_i$-vel és kicsit átalakítva:
  \al{
   \der{}{t}\left(\frac{1}{2}m_ix_i\dot{x}_i\right)-\frac{1}{2}m_i\dot{x}_i^2=\underbrace{\frac{1}{2}x_i F_i}_{\text{viriál}}.
  }
  Az egyenlet időátlaga:
  \al{
   \underbrace{\lim_{\tau\to\infty}\frac{1}{\tau}\intl{0}{\tau}\dd t\,
   \der{}{t}\left(\frac{1}{2}m_ix_i\dot{x}_i\right)}_{\text{I.}}
   -\underbrace{\lim_{\tau\to\infty}\frac{1}{\tau}\intl{0}{\tau}\dd t\,\frac{1}{2}m_i\dot{x}_i^2}_{\text{II.}}
   &=\underbrace{\lim_{\tau\to\infty}\frac{1}{\tau}\intl{0}{\tau}\dd t\,\frac{1}{2}x_i F_i}_{\text{III.}}
  }
  A tagok kifejtve:
  \al{
   \text{I.}
    &=\lim_{\tau\to\infty}\frac{1}{\tau}\underbrace{\left[\frac{1}{2}m_ix_i\dot{x}_i\right]_{0}^{\tau}}_{\text{véges}}=0\\
   \text{II.}
    &=\lim_{\tau\to\infty}\frac{1}{\tau}\intl{0}{\tau}\dd t\,\frac{1}{2}m_i\dot{x}_i^2
     =\mv{E_{\text{kin},i}}\\
   \text{III.}
    &=\lim_{\tau\to\infty}\frac{1}{\tau}\intl{0}{\tau}\dd t\,\frac{1}{2}x_i F_i
     =\frac{1}{2}\mv{x_i F_i},
  }
  ahonnan a viriál tétel $i$-re való összegzéssel:
  \aln{
   \boxed{0=\mv{E_\text{kin}}+\frac{1}{2}\mv{\suml{i=1}{3N}x_i F_i}}.\label{eq:B06-virialtetel}
  }
  
  Először tekintsük csak a fallal való kölcsönhatást. 
  \al{
   U_\text{w}=\suml{i=1}{N}\ointl{\partial V}{}\dd A_\rv w(\rv_i-\rv),
  }
  ahol az integrál a fal teljes felületére megy.
   Ezzel a viriál átlagértéke:
  \al{
   \frac{1}{2}\mv{\suml{i=1}{3N}x_i F_i}
    &=-\frac{1}{2}\mv{\suml{i=1}{N}\rv_i\pder{U_\text{w}}{\rv_i}}
     =-\frac{1}{2}\mv{\suml{i=1}{N}\rv_i\ointl{\partial V}{}\dd A_\rv \grad_i w(\rv_i-\rv)}\\
    &=-\frac{1}{2}\mv{\suml{i=1}{N}\ointl{\partial V}{}\dd A_\rv \rv_i\underbrace{\grad_i w(\rv_i-\rv)}_{\text{nagyon éles}}}
     =-\frac{1}{2}\mv{\suml{i=1}{N}\ointl{\partial V}{}\dd A_\rv \rv\grad_i w(\rv_i-\rv)}\\
    &=\frac{1}{2}\mv{\suml{i=1}{N}\ointl{\partial V}{}\dd A_\rv \rv\grad_\rv w(\rv_i-\rv)}
     =\frac{1}{2}\ointl{\partial V}{}\dd A_\rv \rv\underbrace{\grad_\rv\mv{\suml{i=1}{N} w(\rv_i-\rv)}}_{\substack{\text{erő felületi sűrűsége,}\\\text{azaz a nyomás}}}\\
    &=\frac{1}{2}\ointl{\partial V}{}\dd A_\rv \rv \left(-\ev_\text{felület} p\right)
     =-\frac{p}{2}\ointl{\partial V}{}\df_\rv \rv
     =-\frac{p}{2}\intl{V}{}\drh \divo\rv
     =-\frac{3}{2}p\intl{V}{}\drh
     =-\frac{3}{2}pV.
  }
  Összefoglalva a viriál tétel:
  \al{
   \mv{E_\text{kin}}=\frac{3}{2}pV
  }
  
 \section{Viriál sorfejtés}
  
  Most vesszük figyelembe a gáz részecskéi közötti párkölcsönhatásokat:
  \al{
   U_\text{int}
    &=\suml{\mv{i,j}}{}U(\rv_i-\rv_j)
  }A viriál tétel alapján:
  \al{
   pV
    &=\frac{2}{3}\mv{E_\text{kin}}+\frac{2}{3}\frac{1}{2}\mv{\suml{i=1}{N}\rv_i\Fv_i}
     =\frac{2}{3}\mv{E_\text{kin}}-\frac{2}{3}\frac{1}{2}\mv{\suml{i=1}{N}\rv_i\suml{j(\ne i)}{}\pder{U(\rv_i-\rv_j)}{\rv_i}}\\
    &=\frac{2}{3}\mv{E_\text{kin}}-\frac{2}{3}\frac{1}{2}\frac{1}{2}\mv{\suml{i}{}\suml{j(\ne i)}{}\left(\rv_i\pder{U(\rv_i-\rv_j)}{\rv_i}+\rv_j\pder{U(\rv_i-\rv_j)}{\rv_j}\right)}\\
    &=\frac{2}{3}\mv{E_\text{kin}}-\frac{1}{6}\mv{\suml{i\ne j}{N}(\rv_i-\rv_j)\pder{U(\rv)}{\rv}}.
  }
  A jobb oldal a sorba fejthető: $p=n\kB T\big[1+b(T)n+c(T)n^2+\dots\big]$, ahol $b(T)$, $c(T)$\dots a viriál együtthatók.
  
  \subsection{Klasszikus híg gázok viriál sorfejtése}
   
   Nagykanonikus sokasággal felírva:
   \aln{
    pV=-\Phi=\kB T\ln\mathcal{Z}
     =\kB T\ln\left(\suml{N=0}{\infty}Z_Ne^{\beta\mu N}\right).\label{eq:B06-virialsf1}
   }
   Ideális gázra Láttuk, hogy 
   \al{
    &Z_1=\frac{V}{h^3}\left(2\pi m\kB T\right)^{\frac{3}{2}}
     =\frac{V}{\lambda_T^3}
    &\lambda_T=\frac{h}{\sqrt{2m\pi\kB T}},
   }
   így az állapotegyenlet
   \al{
    pV&=\kB T\ln\left(\suml{N=0}{\infty}\frac{1}{N!}(Z_1 e^{\beta\mu })^N\right)
     =\kB T Z_1 e^{\beta\mu}
     =\kB T \frac{V}{\lambda_T^3} e^{\beta\mu}
     =N\kB T,
   }
   ahonnan $e^{\beta\mu}=\lambda_T^3 n$. $\lambda_T$ értéke fix és pici, így ha $n$ kicsi, akkor $e^{\beta\mu}$ is az.
   Nem ideális gázra is azt gondoljuk, hogy ez kicsi marad, így ez a kis paraméter, ami szerint \eqaref{eq:B06-virialsf1} egyenlet jobb oldalát sorba fejtjük.
   Az állapotösszeg: $\mathcal{Z}=1+Z_1 e^{\beta\mu}+Z_2 e^{2\beta\mu}+\dots$, illetve $\ln(1+x)=x-\frac{x^2}{2}+\dots$, így
   \al{
    pV&=\kB T\ln\left(1+Z_1 e^{\beta\mu}+Z_2 e^{2\beta\mu}+\dots\right)\\
      &=\kB T\Big(Z_1 e^{\beta\mu}+Z_2 e^{2\beta\mu}+\dots-\frac{1}{2}\left(Z_1 e^{\beta\mu}+Z_2 e^{2\beta\mu}+\dots\right)^2+\dots\Big)\\
      &=\kB T\bigg(\underbrace{Z_1}_{z_1} e^{\beta\mu}+\underbrace{\left(Z_2-\frac{1}{2}Z_1^2\right)}_{z_2} e^{2\beta\mu}+\dots\bigg).
   }
   A részecskeszámot szeretnénk inkább látni az egyenlet jobb oldalán, így kifejezzük azt is
   \al{
    N=\pder{\ln\mathcal{Z}}{(\beta\mu)}
     =z_1 e^{\beta\mu}+2z_2 e^{2\beta\mu}+\dots.
   }
   Itt kicsi hibát vétünk, ha felhasználjuk az ideális gáznál kapott $e^{\mu\beta}=\frac{N}{z_1}$ összefüggést a másodrendű tagban:
   \al{
    N
     &=z_1 e^{\beta\mu}+2 z_2 \frac{N^2}{z_1^2}+\dots,
     &\Rightarrow
     &&z_1 e^{\beta\mu}\approx N-2 z_2 \frac{N^2}{z_1^2}.
   }
   Ezeket behelyettesítve a sorfejtésbe:
   \al{
    pV
     &\approx \kB T\left(z_1 e^{\mu\beta}+z_2 e^{2\mu\beta}\right)
      =\kB T\left(N-2 z_2 \frac{N^2}{z_1^2}+z_2 \frac{N^2}{z_1^2}\right)
      =\kB T\left(N-z_2 \frac{N^2}{z_1^2}\right)\\
     &=N\kB T\left(1-N\frac{Z_2-\frac{1}{2}Z_1^2}{Z_1^2}\right).
   }
   Már csak $Z_1$-et és $Z_2$-t kell kiszámolni. $Z_1$ az egyrészecskés állapotösszeg, ebben nem szerepel kölcsönhatás, így ezt ismerjük.
   A kétrészecskés állapotösszeg definíció szerint:
   \al{
    Z_2
     &=\intl{}{}\frac{\dd^3\pv_1\dd^3\pv_2\dd^3\rv_1\dd^3\rv_2}{2 h^6}\,e^{-\beta\big(\frac{\pv_1^2}{2m}+\frac{\pv_2^2}{2m}+U(\rv_1-\rv_2)\big)}
      =\frac{1}{2\lambda_T^6}\intl{}{}\dd^3\rv_1\dd^3\rv_2\,e^{-\beta U(\rv_1-\rv_2)}\\
     &=\frac{V}{2\lambda_T^6}\intl{}{}\dd^3\rv\,e^{-\beta U(\rv)}.
   }
   Mellyek a $b(T)$ viriál együttható:
   \al{
    b(T)
     &=-V\frac{Z_2-\frac{1}{2}Z_1^2}{Z_1^2}
      =-\frac{V}{2}\left(\frac{2 Z_2}{Z_1^2}-1\right)
      =-\frac{V}{2}\left(\frac{2 \lambda_T^6 Z_2}{V^2}-1\right)\\
     &=-\frac{V}{2}\left(\frac{2 \lambda_T^6}{V^2}\frac{V}{2\lambda_T^6}\intl{}{}\dd^3\rv\,e^{-\beta U(\rv)}-1\right)
      =-\frac{1}{2}\left(\intl{}{}\dd^3\rv\,e^{-\beta U(\rv)}-V\right)\\
     &=-\frac{1}{2}\intl{}{}\dd^3\rv\,\left(e^{-\beta U(\rv)}-1\right)
   }
   
   \paragraph{Példa: híg gáz, kemény mag}
    
    Legyen a kölcsönhatás ``hard ball'' potenciál:
    \al{
     U_\text{int}=\begin{cases}
                   \infty, & \text{ha }r<\sigma\\
                   0, & \text{ha }r>\sigma.
                  \end{cases}
    }
    Erre:
    \al{
     b(T)
     &=-\frac{1}{2}\intl{0}{\infty}\dd^3\rv\,\left(e^{-\beta U(\rv)}-1\right)
      =-\frac{1}{2}\intl{0}{\infty}\dd r\,4\pi r^2\left(e^{-\beta U(r)}-1\right)
      =\frac{1}{2}\intl{0}{\sigma}\dd r\,4\pi r^2
      =\frac{2\pi}{3}\sigma^3,
    }
    mellyel:
    \al{
     pV=N\kB T\left[1+\frac{N}{V}\frac{2\pi}{3}\sigma^3+\dots\right].
    }
    A nyomás tehát megnő, ami annak köszönhető, hogy a részecskék kicsit taszítják egymást. 
    
 \section{van der Waals állapotegyenlet}
  
  Célunk, hogy leírjuk a folyadák--gáz fázisátalakulást.
   A viriál sorfejtéssel kapott állapotegyenlet azonban nem lesz kielégítő semmilyen fázisátalakulás közelében, hiszen ott nemanalitikus viselkedést mutatnak a termodinamikai mennyiségek, amely sorfejtésből nem származtatható.
  
  A továbbiak \aref{ss:B09-vdW}. fejezetben.
