\chapter{Kontinuitási egyenletek a fizikában}
  
 \section{Mechanika}
  
  \subsection{Folytonos közegek alapfogalmai}
   
   A folytonos közeget nem tömegpontokból, hanem elkent tömegsűrűségből építjük fel: $\rho(\rv)$ folytonos sima függvény. A rendszer egy $V$ térfogatának a tömege így: $m=\intl{V}{}\drh\rho(\rv)$. 
   
   A folytonos közeg egy tetszőleges állapotának leírásához meg kell jelölnünk annak részeit. A $t=t_0$ kezdeti pillanatban az $\rv(t=t_0)$ helyen lévő $\rho(\rv)\drh$ tömegű részt az $\rv$ vektorral indexeljük.
   
   A rész elmozdulását az $\sv(t,\rv)$ elmozdulásmezővel írhatjuk le: $\rv(t,\rv)=\rv+\sv(t,\rv)$. Az elmozdulás sebességét ennek időderiválásából kapjuk, ez a sebességmező: 
   \eq{
    \vv(t,\rv)=\partial_t\sv(t,\rv).
    }
    
   Az elmozdulásmezőt sorba fejthetjük az $\rv$ változója szerint, azaz két közel lévő anyagdarab elmozdulását vizsgáljuk:
   \al{
    s_i(t,r_j)
     &=s_i(t,0)+\partial_{k}s_i(t,0)r_k+\dots.
   }
   Itt az elsőrendű járulék együtthatója az elmozdulástenzor, amelyet felbontunk egy szimmetrikus és egy antiszimmetrikus tenzorra: 
   \eq{
    s_{ij}
     =\partial_js_i(t,0)
     =\underbrace{\frac{\partial_js_i(t,0)+\partial_is_j(t,0)}{2}}_{=\ep_{ij}}+\underbrace{\frac{\partial_j s_i(t,0)-\partial_i s_j(t,0)}{2}}_{=a_{ij}}.
   }
   Az elmozdulásmező így egy eltolásból ($\sv(t,0)$), egy forgatásból ($a_{ij}$) és egy deformációból ($\ep_{ij}$) áll.
   
   A forgatás tagjának kifejtése: 
   \eq{
    \big[(\rot{\sv})\times\rv\big]_i
     =\ep_{ijk}\big(\ep_{jlm}\partial_l s_m\big)r_k
     =\big((\delta_{kl}\delta_{im}-\delta_{km}\delta_{il})\partial_l s_m\big)r_k
     =\big(\partial_k s_i-\partial_i s_k\big)r_k
     =2a_{ik}r_k.
   }
   Vagyis az $a_{ij}$ tenzorral való hatás egy $\phi=\frac{1}{2}\rot{\sv}$-vel való elforgatásnak felel meg. 
   
   A deformációnak megfelelő rész jelentéséhez nézzünk két egymáshoz közeli pontot: $\uv$-t és $\vv$-t. Legyen $\vv=\uv+\Delta \xv$. Elmozdulás után az új $x$ koordináták különbsége: 
   \al{
    \big[\Delta\bar \xv\big]_i
     &=\big[\bar\vv-\bar\uv\big]_i
      =\Big[\big(\uv+\Delta \xv+\sv(\uv+\Delta \xv)\big)-\big(\uv+\sv(\uv)\big)\Big]_i
      =\big[\Delta \xv+\sv(\uv+\Delta \xv)-\sv(\uv)\big]_i\\
     &=\Delta x_i+\partial_j s_i(\uv)\Delta x_j.
   }
   Ha pl. $\big[\Delta\xv\big]_i=d\delta_{1i}$, tehát $\uv$ és $\vv$ egymástól $d$ távolságra vannak $x$ irányban, akkor $\ep_{11}=\frac{\bar d-d}{d}$, vagyis az $\ep$ diagonális elemei az adott irányú relatív megnyúlásokat tartalmazzák. A nem diagonális elemek az adott pontokat összekötő szakaszainak elfordulását adják. 
   
   A térfogatelem megváltozása: 
   \eq{
    \dd \bar V=\dd x\dd y\dd z=(1+\ep_{11})\dd x(1+\ep_{22})\dd y(1+\ep_{33})\dd z\approx(1+\ep_{11}+\ep_{22}+\ep_{33})\dd V.
   }
   
   \paragraph{Erők} 
   
    A kontinuumra ható erők kétfélék lehetnek. Az egyik eset, ha az erő térfogati, azaz minden tömegpontra hat. Ilyen például a nehézségi erő vagy az elektromágneses erők. A térfogati erőkhöz bevezethető egy erősűrűség: 
    \eq{
     \Fv=\intl{V}{}\drh\fv(t,\rv). 
    }
    
    A másik eset, ha az erő csak egy felület mentén hat, akkor azt felületi erőnek hívjuk. Ilyenek pl. a molekuláris erők. Ebben az esetben az erőhöz egy feszültségtenzor vezethető be. Itt az erő iránya általános esetben nem párhuzamos a felület merőlegesével, ezért az erő és a felületmerőleges között legegyszerűbb esetben is tenzoriális a kapcsolat:
    \eq{
     F_i=\ointl{\partial V}{}\sigma_{ij}\dd^2 f_j.
    }
    
  \subsection{A szubsztanciális derivált}
   
   Egy mérhető mennyiség értékét vizsgálhatjuk egy adott $\rv$ helyen. Ekkor lokálisan ott a megváltozása $\frac{A(t+\Delta t,\rv)-A(t,\rv)}{\Delta t}\to\partial_t A(t,\rv)$. Ha azonban egy kiszemelt anyagdarabot szeretnénk követni, és annak a változását szeretnénk figyelni, akkor az anyaggal együtt kell mozognunk:
   \al{
    \frac{A(t+\Delta t,\rv+\vv\Delta t)-A(t,\rv)}{\Delta t}
     &=\frac{A(t+\Delta t,\rv)+\partial_i A(t+\Delta t,\rv)v_i\Delta t-A(t,\rv)}{\Delta t}\\
     &=\partial_t A(t,\rv)+v_i\partial_i A(t+\Delta t,\rv)
      \xrightarrow{\Delta t\to 0}\partial_t A(t,\rv)+v_i\partial_i A(t,\rv)\\
     &=\der{A(t,\rv)}{t}=\dd_t A(t,\rv),
   }
   ez a mennyiség a szubsztanciális derivált. 
   
   Egy integrális mennyiségnek is elkészíthetjük ezt a kétfajta deriváltját. Első esetben rögzítsük az integrálási tartományt, ekkor az integrális mennyiség megváltozása:
   \al{
    \frac{I(t+\Delta t,V)-I(t,V)}{\Delta t}
     &=\frac{1}{\Delta t}\left(\intl{V}{}\drh A(t+\Delta t,\rv)-\intl{V}{}\drh A(t,\rv)\right)\\
     &=\intl{V}{}\drh \underbrace{\frac{1}{\Delta t}\big(A(t+\Delta t,\rv)- A(t,\rv)\big)}_{\to\partial_t A(t,\rv)}
     \to\partial_t\intl{V}{}\drh A(t,\rv)
     =\partial_t I
   }
   A másik esetben a $V$ térfogat az anyaggal együtt mozog. Ekkor az integrális mennyiség megváltozása: 
   \aln{
   \dd_t I(t,V)
    &\xleftarrow{\Delta t\to0}
    \frac{I(t+\Delta t,V(t+\Delta t))-I(t,V)}{\Delta t}
      \nonumber\\
     &\qquad=\frac{1}{\Delta t}\left(
        \intl{V(t+\Delta t)}{}\drh A(t+\Delta t,\rv)
       -\intl{V(t)}{}\drh A(t,\rv)
       \right)\nonumber\\
     &\qquad=\frac{1}{\Delta t}\left(
        \intl{V(t)}{}\drh A(t+\Delta t,\rv)
       +\ointl{\partial V(t)}{}\df\Delta t\vv A(t+\Delta t,\rv)
       -\intl{V(t)}{}\drh A(t,\rv)
       \right)\nonumber\\
     &\qquad=\ointl{V(t)}{}\drh \partial_tA(t,\rv)+\intl{\partial V(t)}{}\df \vv A(t,\rv)
     =\intl{V(t)}{}\drh \partial_t A(t,\rv)+\intl{V(t)}{}\drh \partial_i\big(v_i A(t,\rv)\big)
     \nonumber\\
     &\qquad=\intl{V(t)}{}\drh \Big[\dd_t A(t,\rv)+A(t,\rv)\partial_i v_i \Big]\label{eq:08-szubsztancialisderivalt}
   }
   
   \paragraph{Kontinuitási egyenlet}
    
    Alkalmazzuk a fentieket az $m=\intl{V}{}\drh\rho(\rv)$ integrális mennyiségre:
    \eq{
     \dd_t m=\intl{V}{}\drh \Big[\dd_t \rho(t,\rv)+ \rho(t,\rv)\partial_i v_i \Big].
    }
    Ez a mennyiség egyenlő a $V$ térfogatba a forrásokból keletkező tömeg mennyiségével: $\dd_t m=\intl{V}{}\drh q(t,\rv)$, így
    \eqn{
     \intl{V}{}\drh q(t,\rv)=\intl{V}{}\drh \Big[\dd_t \rho(t,\rv)+ \rho(t,\rv)\partial_i v_i \Big].\label{eq:08-contint}
    }
    Az egyenlet átrendezve:
    \eq{
     \partial_t m
      =\intl{V}{}\drh q(t,\rv)-\ointl{\partial V}{}\df\rho\vv,
    }
    ahol a rögzített $V$ térfogat tömegét a $q$ források és a térfogatból kiáramló anyagmennyiség változtatja meg. 
    
    Mivel \eqaref{eq:08-contint} egyenlet minden $V$ térfogatra igaz, ezért az integrandusokra is igaznak kell lennie az összefüggésnek:
    \eqn{
     \partial_t\rho(t,\rv)+\partial_i(\rho(t,\rv)v_i)=q.\label{eq:08-contdiff}
    }
   
  \subsection{Mérlegegyenletek}
   
   Először vizsgáljunk meg egy speciális integrális mennyiséget, legyen ennek a sűrűsége $\rho(t,\rv)\cdot A(t,\rv)$. Ennek a szubsztanciális deriváltja:
   \al{
    \dd_t I(t,V)
     &=\dd_t\intl{V}{}\drh A(t,\rv)\rho(t,\rv).
      =\intl{V}{}\drh \Big[\partial_t \big(\rho(t,\rv)A(t,\rv)\big)+ \partial_i\big(v_i \rho(t,\rv)A(t,\rv)\big)\Big]\\
     &=\intl{V}{}\drh \Big[
        \partial_t \rho(t,\rv)A(t,\rv)+\rho(t,\rv)\partial_t A(t,\rv)
        + A(t,\rv)\partial_i\big(v_i \rho(t,\rv)\big)+\partial_i A(t,\rv)\big(\rho(t,\rv)v_i\big)
       \Big]\\
     &=\intl{V}{}\drh \Big[
        A(t,\rv)\Big(\partial_t \rho(t,\rv)+\partial_i\big(v_i \rho(t,\rv)\big)\Big)
        +\rho(t,\rv)\Big(\partial_tA(t,\rv)+v_i \partial_iA(t,\rv)\Big)
       \Big].
   }
   Itt az első tagban a kerek zárójel nullát ad a kontinuitási egyenlet miatt, ha nincsenek források a rendszerben. A második kerek zárójel pedig $A(t,\rv)$ szubsztanciális deriváltja:
   \eqn{
    \dd_t I(t,V)
     =\intl{V}{}\drh\rho(t,\rv)\dd_t A(t,\rv).\label{eq:08-szubsz}
   }
   
   \paragraph{Az impulzus mozgásegyenlete}
    
    Alkalmazzuk \eqaref{eq:08-szubsz} egyenletet a $A(t,\rv)=\vv_i$-re. Így $\Pv=\intl{V}{}\drh\rho(t,\rv)\vv$:
    \eq{
     \dd_t P_i 
       =\intl{V}{}\drh\rho(t,\rv)\dd_t v_i
       =\intl{V}{}\drh f_i(t,\rv)+\ointl{\partial V}{}\df_j\sigma_{ij}
    }
    ahol kihasználtuk, hogy ez Newton-egyenlet miatt egyenlő az anyagdarabra ható erővel. ami így az integrálást elhagyva a differenciális Newton-egyenletet adja:
    \eqn{
     \rho\dd_t v_i=f_i+\partial_j\sigma_{ij}.\label{eq:08-diffN}
    }
    
    Ha nem alkalmazzuk \eqaref{eq:08-szubsz} egyenletet, hanem csak a parciális deriváltakat tartalmazó alakot, akkor 
    \eq{
     \dd_t P_i
      =\intl{V(t)}{}\drh \Big[\partial_t \big(\rho(t,\rv)v_i\big)+\partial_j\big(v_j \rho(t,\rv)v_i\big)\Big]
      =\intl{V}{}\drh f_i(t,\rv)+\ointl{\partial V}{}\df_j\sigma_{ij},
    }
    mely differenciális alakban:
    \eqn{
     \partial_t \big[v_i\rho(t,\rv)\big]+\partial_j\big[v_iv_j \rho(t,\rv)-\sigma_{ij}\big]=f_i(t,\rv).
    }
    Ez a $\vv\rho(t,\rv)$-re, azaz az impulzussűrűségre felírt kontinuitási egyenlet. Az impulzussűrűség forrása az erősűrűség, az árama pedig a $\big[\jv_{p_i}\big]_j=v_iv_j \rho(t,\rv)-\sigma_{ij}$.
    
   \paragraph{Az impulzusmomentum mozgásegyenlete}
    
    A fenti gondolatmenethez hasonlóan járunk el, csak itt az $L_i=\intl{V}{}\drh\rho(t,\rv)\ep_{ijk}r_jv_k$ mennyiség szubsztanciális deriváltját számoljuk.
    \eq{
     \dd_t L_i=\intl{V}{}\drh\rho(t,\rv)\ep_{ijk}\dd_t\big(r_jv_k\big)
    }
    Fejtsük ki az integrandust:
    \al{
     \ep_{ijk}\dd_t\big(r_jv_k\big)
      &=\ep_{ijk}\Big[\partial_t\big(r_jv_k\big)+v_l\partial_l\big(r_jv_k\big)\Big]
       =\ep_{ijk}\Big[0+r_j\partial_tv_k+v_l\underbrace{\partial_lr_j}_{\delta_{lj}}v_k+v_lr_j\partial_lv_k\Big]\\
      &=\ep_{ijk}\Big[r_j\dd_tv_k+v_jv_k\Big]
       =\ep_{ijk}r_j\dd_tv_k.
    }
    Ezt visszahelyettesítve, illetve felhasználva \eqaref{eq:08-diffN} egyenletet:
    \al{
     \dd_t L_i
      &=\intl{V}{}\drh\rho(t,\rv)\ep_{ijk}r_j\dd_tv_k
       =\intl{V}{}\drh\ep_{ijk}r_j\big(f_k+\partial_l\sigma_{kl}\big)\\
      &=\intl{V}{}\drh\ep_{ijk}r_jf_k
        +\ointl{\partial V}{}\df_l\ep_{ijk}r_j\sigma_{kl}
        -\underbrace{\intl{V}{}\drh\ep_{ijk}\sigma_{kj}}_{=0 (!)}.
    }
    Itt az első az $\fv$ erősűrűség forgatónyomatéka, a második tag a $\sigma_{kl}$ felületi feszültséghez tartozó forgatónyomaték. Az utolsó tag az differenciálás átírásakor jelenik meg. Tudjuk, hogy ha nincs forgatónyomaték, akkor nincs perdületváltozás, így az utolsó tagnak nullának kell lennie. Ez feltételt szab a $\sigma_{kj}$ tenzorra. A tag csak akkor tűnik el, ha a feszültségtenzor teljesen szimmetrikus:
    \eq{
     \sigma_{ij}=\sigma_{ji}.
    }
    
    
   \paragraph{A kinetikus energia mozgásegyenlete}
    
    A kinetikus energia sűrűséggel kifejezve:
    \eq{
     K=\intl{V}{}\drh\frac{1}{2}\rho(t,\rv)v_iv_i.
    } 
    Készítsük el ennek is a szubsztanciális deriváltját \eqaref{eq:08-szubsz} egyenletnek megfelelően:
    \al{
     \dd_t K
      &=\intl{V}{}\drh\frac{1}{2}\rho(t,\rv)\dd_t\big[v_iv_i\big]
       =\intl{V}{}\drh\frac{1}{2}\rho(t,\rv)\Big(\partial_t\big[v_iv_i\big]+v_j\partial_j\big[v_iv_i\big]\Big)\\
      &=\intl{V}{}\drh\frac{1}{2}\rho(t,\rv)\Big(2v_i\partial_tv_i+2v_iv_j\partial_jv_i\Big)
       =\intl{V}{}\drh v_i\rho(t,\rv)\Big(\partial_tv_i+v_j\partial_jv_i\Big)\\
      &=\intl{V}{}\drh v_i\rho(t,\rv)\dd_tv_i.
    }
    Itt is felhasználjuk \eqaref{eq:08-diffN} egyenletet:
    \al{
     \dd_t K
      &=\intl{V}{}\drh v_i\big(f_i+\partial_j\sigma_{ij}\big)
       =\intl{V}{}\drh v_if_i
        +\intl{V}{}\drh \partial_j\big(v_i\sigma_{ij}\big)
        -\intl{V}{}\drh \sigma_{ij}\partial_jv_i\\
      &=\intl{V}{}\drh v_if_i
        +\ointl{\partial V}{} \df_j v_i\sigma_{ij}
        -\intl{V}{}\drh \sigma_{ij}\partial_jv_i
    }
    Vizsgáljuk meg itt a harmadik tag integrandusát: 
    \eq{
     \sigma_{ij}\partial_jv_i
      =\sigma_{ij}\partial_j\partial_ts_i
      =\sigma_{ij}\partial_t\partial_js_i
      =\sigma_{ij}\partial_t s_{ij}.
    }
    Itt mivel $\sigma$ szimmetrikus, ezért $s_{ij}$-nek is csak a szimmetrikus része ad járulékot ($\ep_{ij}$), így:
    \eq{
     \dd_t K
      =\intl{V}{}\drh v_if_i
       +\ointl{\partial V}{} v_i\sigma_{ij}
       -\intl{V}{}\drh \sigma_{ij}\partial_t\ep_{ij}.
    }
    A $\dd_tK$-t itt is ki tudjuk fejteni parciális deriváltakkal is:
    \eq{
     \dd_t K
      =\intl{V}{}\drh\left[\partial_t\left(\frac{1}{2}\rho(t,\rv)v_iv_i\right)+\partial_j\left(\frac{1}{2}\rho(t,\rv)v_iv_iv_j\right)\right].
    } 
    Az előző két egyenletet megfelelő oldalait egyenlővé téve és átrendezve a kinetikus energiasűrűségre vonatkozó kontinuitási egyenlete kapjuk:
    \eqn{
     \intl{V}{}\drh\left[\partial_t\left(\frac{1}{2}\rho(t,\rv)v_iv_i\right)+\partial_j\left(\frac{1}{2}\rho(t,\rv)v_iv_iv_j-v_i\sigma_{ij}\right)\right]
      =\intl{V}{}\drh\bigg[ v_if_i
      -\sigma_{ij}\partial_t\ep_{ij}\bigg].\label{eq:08-kinint}
    }
    Itt az első tag az adott térfogatban a kinetikus energia időbeli megváltozása, a második tagban a deriváláson belül a kinetikus energia árama, az egyenlőség jobb oldalán pedig a a kinetikus energia forrása található. 
    
    Ennek lokális alakja a kinetikus energiája vonatkozó kontinuitási egyenlet:
    \eqn{
     \partial_t\left(\frac{1}{2}\rho(t,\rv)v_iv_i\right)+\partial_j\left(\frac{1}{2}\rho(t,\rv)v_iv_iv_j-v_i\sigma_{ij}\right)
      =v_if_i-\sigma_{ij}\partial_t\ep_{ij},\label{eq:08-kindiff}
    }
    ahol az első zárójelben a kinetikus energia sűrűsége áll, a másodikban az energiaáram-sűrűség, a jobb oldalon pedig az energiasűrűség forrása, vagyis a munka sűrűsége áll. 
    
   
  \subsection{Navier--Stokes-egyenlet}
   
   Láttuk, hogy a mozgásegyenlet \eqaref{eq:08-diffN} alakban írhatjuk fel lokálisan. Alakítsuk át a $\sigma_{ij}$ feszültségtenzort, válasszuk le annak trace-ét (a nyomást):
   \al{
    &\pi=-\frac{1}{3}\sigma_{ii}
    &\sigma_{ij}=-\delta_{ij}\pi + T_{ij}.
   }
   Ezután a mozgásegyenlet:
   \eq{
    \rho\dd_t v_i=f_i+\partial_jT_{ij}-\partial_j\pi\delta_{ij}.
   }
   Ebben az egyenletben a feszültségtenzor elemeire szükséges valamilyen fizikai megfontolásokat tenni. Ezek a megfontolások attól függnek, hogy milyen anyagra és milyen körülmények között akarjuk alkalmazni az egyenletet. A mi esetünk a következő: folytonos, nemrelativisztikus, Newtoni folyadékokkal foglalkozunk.
   
   A feltételezések a következőek: az anyag izotrop, nyugalomban lévő anyagban $\partial_i T_{ij}=0$, hogy a hidrosztatikus esetet kapjuk, illetve hogy a feszültségtenzor arányos a deformációsebességgel. Ezek alapján a feszültségtenzor és a nyomás:
   \al{
    &\pi=p-\eta'\partial_iv_i
    &T_{ij}=\eta\left(\partial_iv_j+\partial_jv_i-2\delta_{ij}\frac{1}{3}\partial_kv_k\right).
   }
   Itt az $\eta$ a hagyományos viszkozitás, $\eta'$ pedig csak akkor jelenik meg, ha a közeg összenyomható. Ekkor a viszkozitás miatt a nyomás is megváltozik. A $p$ a termodinamikai módszerekkel számolható nyomásnak felel meg. Ezeket behelyettesítve:
   \al{
    \rho\dd_t v_i
     &=f_i+\eta\partial_j\left(\partial_iv_j+\partial_jv_i-2\delta_{ij}\frac{1}{3}\partial_kv_k\right)-\partial_j\left[p-\eta'\partial_kv_k\right]\delta_{ij}\\
     &=f_i+\eta\partial_i\partial_jv_j+\eta\partial_j\partial_jv_i-2\eta\delta_{ij}\frac{1}{3}\partial_j\partial_kv_k-\partial_ip+\eta'\partial_i\partial_kv_k\\
     &=f_i-\partial_ip+\left(\frac 13\eta+\eta'\right)\partial_i\partial_jv_j+\eta\partial_j\partial_jv_i.
   }
   Ez vektoros alakban:
   \eqn{
    \rho\dd_t\vv
     =\fv-\grad{p}+\left(\frac 13\eta+\eta'\right)(\divo\circ\divo)\vv+\eta\Delta\vv.\label{eq:08:NavierStokes}
   }
    
 \section{Elektrodinamika}
 
  \paragraph{Ponttöltések és töltéssűrűségek}
   
   A ponttöltések helyén sűrűség divergál, így ezeket a függvény értelemben nem lehet a töltéssűrűséghez hozzávenni. Disztribúció értelemben azonban igen, ekkor egy $\rv_0$ helyen lévő ponttöltéshez tartozó sűrűség: $\rho(\rv)=q\delta(\rv-\rv_0)$. 
   
  \paragraph{Áramsűrűség, töltésmegmaradás}
   
   Az elektromágneses töltéseket vizsgálva is megállapíthatunk egy kontinuitási egyenletet. A $\rho(t,\rv)$ sűrűség itt a töltéssűrűség, és a $\vv$ sebességmező pedig a töltéshordozók mozgása. Az elektromos áramsűrűség: $\Jv(t,\rv)=\rho(t,\rv)\vv$. Az elektromos áram ennek egy adott felületre vett integrálja:
   \eq{
    I=\intl{A}{}\df \Jv(t,\rv).
   }
   
   Feltesszük, hogy itt töltések nem keletkezhetnek, és nem is veszhetnek el. Emiatt az együttmozgó $V$ térfogatban a töltések megmaradnak:
   \aln{
    &\dd_t\intl{V}{}\drh\rho(t,\rv)=0
    &\partial_t\rho(t,\rv)+\divo{\Jv(t,\rv)}=0.\label{eq:08-tolteskont}
   }
   
  \paragraph{Ohm-törvény}
   
   Az Ohm-törvény egy anyagi egyenlet, azt írja le, hogy adott térerősségre egy fémes vezetőben mekkora áramsűrűség jön létre. Az összefüggés egyszerű arányosság:
   \eq{
    \Jv(t,\rv)=\sigma\Ev(t,\rv).
   }
   
   Ezzel az egyenlettel az egyéb anyagi állandók ($\ep$, $\mu$) ismeretében meg lehet határozni az anyagdarab elektromágneses viselkedését. Ezek közül az egyik legegyszerűbb eset a kvázistacionárius  viselkedés.
   
  \subsection{Kvázistacionárius dinamika}\label{ss:08-kvazistacdin}
   
   Oldjuk meg a 
   \al{
    \divo{\vect{D}(t,\vect{r})}&=\rho(t,\vect{r}) &
     \divo\vect{B}(t,\vect{r})&=0 \\
     \rot{\vect{E}(t,\vect{r})}&=-\partial_t\vect{B}(t,\vect{r}) &
     \rot{\vect{H}(t,\vect{r})}&=\vect{J}(t,\vect{r}) 
    }
   egyenletrendszert, annak tudatában, hogy nincs szabad töltés ($\rho(\rv)=0$), $\Hv(t,\rv)=\Bv(t,\rv)/\mu$, $\Dv(t,\rv)=\ep\Ev(t,\rv)$, illetve $\Jv(t,\rv)=\sigma\Ev(t,\rv)$.
   
   A rotációt tartalmazó egyenletekre még egyszer rotációt alkalmazva, majd átalakítva az $\Ev(t,\rv)$ és a $\Bv(t,\rv)$-re is hasonló egyenletet kapunk:
   \al{
    &\Delta\Ev(t,\rv)=\mu\sigma\partial_t\Ev(t,\rv)
    &\Delta\Bv(t,\rv)=\mu\sigma\partial_t\Bv(t,\rv)
   }
   \paragraph{Töltés szennyezés}
    Oldjuk meg az $\Ev(t,\rv)$-re vonatkozó egyenletet homogén közegben az időben állandó $\Ev(t,\rv)=\Ev_0(\rv)$ peremfeltétel mellett. Fourier-transzformáljuk az egyenletet a térkoordináta szerint:
    \al{
     &\Delta\Ev(t,\rv)=\mu\sigma\partial_t\Ev(t,\rv)&
     &\Rightarrow&
     &-\kv^2\Ev(t,\kv)=\mu\sigma\partial_t\Ev(t,\kv),&
    }
    melynek megoldása:
    \eq{
     \Ev(t,\kv)=\Ev_0(\kv)e^{-\frac{\kv^2t}{\mu\sigma}}.
    }
    Láthatjuk, hogy időben exponenciális lefutású a probléma, míg térben a különböző Fourier-kom\-po\-nen\-sek más más időállandóval csengenek le, az élesebb csúcsok ($\abs{\kv}>>1$) gyorsabban, az sima alakok pedig lassabban. 
    
  \subsection{Stacionárius áramlási terek}
   
   Stacionárius áramlási terek esetében a töltéssűrűség nem változik: $\partial_t\rho(t,\rv)=0$, így a kontinuitási egyenlet miatt $\divo{\Jv(\rv)}=0$. Ennek következménye, hogy stacionárius áramlások és az elektrosztatika összefüggései között igen erős analógiát lehet találni.
   \begin{table}[h!]
    \centering
    \begin{tabular}{r||c|c}
     Kapcsolat & Elektrosztatika & Kvázistacioner áramlások \\ \hline
     $\Ev(\rv)$ & $\Ev(\rv)=-\grad{\phi(\rv)}$ & $\Ev(\rv)=-\grad{\phi(\rv)}$ \\
     $\Dv(\rv)$ -- $\Jv(\rv)$ & $\Dv(\rv)=\ep\Ev(\rv)$ & $\Jv(\rv)=\sigma\Ev(\rv)$ \\
      & $\divo{\Dv(\rv)}=0$ & $\divo{\Jv(\rv)}=0$ \\
     $Q$ -- $I$ & $Q=\ointl{}{}\df\Dv(\rv)$ & $I=\intl{}{}\df\Jv(\rv)$ \\
     $C$ -- $\sigma$ & $C=\frac{Q}{\abs{\phi_1-\phi_2}}$ & $\sigma=\frac{I}{\abs{\phi_1-\phi_2}}$. \\
    \end{tabular}
    \caption{A stacionárius áramlások és az elektrosztatika összefüggései és az azok közötti párhuzam.}\label{tabl:08-stacstat}
   \end{table}
   
 \section{Kvantummechanika}
  
  \subsection{Kontinuitási egyenletek a kvantummechanikában}\label{ss:A08-kv}
   
   \paragraph{Schrödinger-egyenlet}
    
    A Schrödinger-egyenletnek készítsük el a konjugáltját, majd bővítsük:
    \al{
     i\hbar\partial_t\Psi(t,\rv)&=\left(-\frac{\hbar^2}{2m}\Delta+V(\rv)\right)\Psi(t,\rv)\\
     &\Rightarrow\qquad\qquad
     i\hbar\Psi^*(t,\rv)\big(\partial_t\Psi(t,\rv)\big)=\Psi^*(t,\rv)\left[\left(-\frac{\hbar^2}{2m}\Delta+V(\rv)\right)\Psi(t,\rv)\right]
     \\
     -i\hbar\partial_t\Psi^*(t,\rv) &=\left(-\frac{\hbar^2}{2m}\Delta+V(\rv)\right)\Psi^*(t,\rv)&\\
     &\Rightarrow\qquad\qquad
     -i\hbar\partial_t\Psi^*(t,\rv)\Psi(t,\rv)=\left[\left(-\frac{\hbar^2}{2m}\Delta+V(\rv)\right)\Psi^*(t,\rv)\right]\Psi(t,\rv).
    }
    Vonjuk ki a két egyenletet egymásból, majd rendezzük:
    \al{
      0&=i\hbar\big[\Psi(t,\rv)^*\partial_t\Psi(t,\rv)+\partial_t\Psi^*(t,\rv)\Psi(t,\rv)\big]+
      \frac{\hbar^2}{2m}\left[\Psi^*(t,\rv)\Delta\Psi(t,\rv)-\Delta\Psi^*(t,\rv)\Psi(t,\rv)\right]
     \\
      0&=i\hbar\partial_t\big[\Psi^*(t,\rv)\Psi(t,\rv)\big]+
      \frac{\hbar^2}{2m}\divo{\left[\Psi^*(t,\rv)\grad{\Psi(t,\rv)}-\grad{\Psi^*(t,\rv)}\Psi(t,\rv)\right]}
     \\
     0&=\partial_t\big[\Psi^*(t,\rv)\Psi(t,\rv)\big]+
      \divo{\left[\frac{i\hbar}{2m}\left(\Psi(t,\rv)\grad{\Psi^*(t,\rv)}-\grad{\Psi(t,\rv)}\Psi^*(t,\rv)\right)\right]}
     \\
     0&=\partial_t\big[\underbrace{\Psi^*(t,\rv)\Psi(t,\rv)}_{\rho(t,\rv)}\big]+
      \divo{\bigg(\frac{1}{m}\underbrace{\Re\big[\Psi^*(t,\rv)\oppv{\Psi(t,\rv)}\big]}_{\jv(t,\rv)}\bigg)}.
    }
    A Schrödinger-egyenlet alakjából azonnal következik a fenti kontinuitási egyenlet. Itt az első tag mint valamilyen sűrűség, a második pedig mint az ahhoz tartozó áramsűrűség jelenik meg. Mivel $\rho(t,\rv)$ pozitív definit és normált, ezért valóban tekinthető valószínűségi-sűrűségnek. Emellett a valószínűségi-áramsűrűség is valós mennyiség.
    
   \paragraph{Pauli--Schrödinger-egyenlet}
    
    Tekintsük most az elektromágneses teret tartalmazó Ha\-mil\-ton-o\-pe\-rá\-tort  abban az esetben, mikor kétkomponensűek a hullámfüggvények. A Pauli-egyenlet és annak konjugáltja:
    \al{
     i\hbar\partial_t\Psi&=\left(-\frac{\hbar^2}{2m}\Delta+\frac{i\hbar q}{2m}\big(\nabla\Av+\Av\nabla\big)+q\phi \right)\Psi+\mu_\text{B}\Bv\sigv\Psi\\
     &\Rightarrow\qquad\qquad
     i\hbar\Psi^\dagger\partial_t\Psi=\Psi^\dagger\left(-\frac{\hbar^2}{2m}\Delta+\frac{i\hbar q}{2m}\big(\nabla\Av+\Av\nabla\big)+q\phi \right)\Psi+\mu_\text{B}\Bv\Psi^\dagger\sigv\Psi
     \\
     -i\hbar\partial_t\Psi^\dagger&=\left(-\frac{\hbar^2}{2m}\Delta-\frac{i\hbar q}{2m}\big(\nabla\Av+\Av\nabla\big)+q\phi \right)\Psi^\dagger+\mu_\text{B}\Bv\Psi^\dagger\sigv\\
     &\Rightarrow\qquad\qquad
     -i\hbar\partial_t\Psi^\dagger\Psi=\left(-\frac{\hbar^2}{2m}\Delta-\frac{i\hbar q}{2m}\big(\nabla\Av+\Av\nabla\big)+q\phi \right)\Psi^\dagger\Psi+\mu_\text{B}\Bv\Psi^\dagger\sigv\Psi
    }
    Vonjuk ki a két egyenletet egymásból, majd rendezzük:
    \al{
     0
      &=i\hbar\partial_t\big(\Psi^\dagger\Psi\big)
        +\frac{\hbar^2}{2m}\big(\Psi^\dagger\Delta\Psi-\Delta\Psi^\dagger\Psi\big)
        -\frac{i\hbar q}{2m}\Big(\Psi^\dagger\big(\nabla\Av+\Av\nabla\big)\Psi+\big(\nabla\Av+\Av\nabla\big)\Psi^\dagger\Psi\Big).
    }
    A második tagot a szokásos módon alakítjuk át:
    \eq{
     \Psi^\dagger\Delta\Psi-\Delta\Psi^\dagger\Psi=\divo{\big(\Psi^\dagger\grad\Psi-\grad\Psi^\dagger\Psi\big)},
    } 
    az utolsónál pedig elvégezzük a deriválásokat:
    \al{
     \Psi^\dagger\big(\nabla\Av+\Av\nabla\big)\Psi+\big(\nabla\Av+\Av\nabla\big)\Psi^\dagger\Psi
      &=\divo{\Av}\Psi^\dagger\Psi+2\Av\Psi^\dagger\grad\Psi+\divo{\Av}\Psi^\dagger\Psi+2\Av\grad\Psi^\dagger\Psi\\
      &=2\divo{\Av}\Psi^\dagger\Psi+2\Av\grad\Psi^\dagger\Psi+2\Av\Psi^\dagger\grad\Psi\\
      &=2\divo{\big(\Av\Psi^\dagger\Psi\big)}.
    }
    Így:
    \al{
     0&=\partial_t\big(\Psi^\dagger\Psi\big)+\divo{\left(\frac{\hbar}{2mi}\big(\Psi^\dagger\grad\Psi-\grad\Psi^\dagger\Psi\big)-\frac{q}{m}\Av\Psi^\dagger\Psi\right)}\\
      &=\partial_t\big(\Psi^\dagger\Psi\big)+\divo{\left(\frac{1}{m}\Re\big[\Psi^\dagger\big(\oppv-q\Av\big)\Psi\big]\right)}
      =\partial_t\big(\Psi^\dagger\Psi\big)+\divo{\left(\frac{1}{m}\Re\big[\Psi^\dagger\opkv\Psi\big]\right)}.
    }
    Az áramsűrűséghez itt nem a $\oppv$, hanem a kinetikus impulzus kapcsolódik.
    
    
   \paragraph{Klein--Gordon-egyenlet}
    
    A fentihez teljesen hasonló átalakítással a
    \eq{
     0=\Psi(t,\rv)\square{\Psi^*(t,\rv)}-\Psi^*(t,\rv)\square{\Psi(t,\rv)}
    }
    egyenletre jutunk. Ez kicsit megdolgozva:
    \al{
     0
      &=
       \big(\Psi(t,\rv)\Delta{\Psi^*(t,\rv)}-\Psi^*(t,\rv)\Delta{\Psi(t,\rv)}\big)
       -\frac{1}{c^2}\big(\Psi(t,\rv)\partial_t^2{\Psi^*(t,\rv)}-\Psi^*(t,\rv)\partial_t^2{\Psi(t,\rv)}\big)\\
     0&=
       \divo{\big(\Psi(t,\rv)\grad{\Psi^*(t,\rv)}-\Psi^*(t,\rv)\grad{\Psi(t,\rv)}\big)}
       -\frac{1}{c^2}\partial_t\big(\Psi(t,\rv)\partial_t{\Psi^*(t,\rv)}-\Psi^*(t,\rv)\partial_t{\Psi(t,\rv)}\big)\\
     0&=
       -\partial_t\bigg(\frac{1}{c^2}2i\Im\big[\Psi(t,\rv)\partial_t{\Psi^*(t,\rv)}\big]\bigg)+\divo{\bigg(2i\Re\big[\Psi^*(t,\rv)i\grad{\Psi(t,\rv)}\big]\bigg)}
       \\
     0&=
       \partial_t\bigg(\frac{\hbar}{mc^2}\Im\big[\Psi(t,\rv)\partial_t{\Psi^*(t,\rv)}\big]\bigg)+\divo{\bigg(\frac{1}{m}\Re\big[\Psi^*(t,\rv)\oppv{\Psi(t,\rv)}\big]\bigg)}.
    }
    Az áramsűrűségnek megfelelő tag megegyezik a klasszikus esetnél lévővel, azonban a sűrűségnek megfelelő tag más, és az itt szereplővel a probléma az, hogy nem pozitív definit. A Klein--Gordon-egyenlet másodrendű, így két kezdeti feltétel szükséges a megoldásához. Az egyik a kezdeti elrendezés, $\Psi(t=t_0,\rv)$, de ettől teljesen független $\partial_t\Psi(t=t_0,\rv)$, aminek nincsen fizikai jelentése. Emellett a spin értelmezését sem tartalmazza az egyenlet.
    
   \paragraph{Dirac-egyenlet}
    
    A Dirac-egyenletet is felírjuk a hagyományos és a konjugált alakban is:
    \al{
     i\hbar\partial_t\Psi&=c\alv(\oppv-q\Av)\Psi+q\phi\Psi+\beta mc^2\Psi\\
     &\Rightarrow\qquad\qquad
     i\hbar\Psi^\dagger\partial_t\Psi=c\Psi^\dagger\alv(\oppv-q\Av)\Psi+q\phi\Psi^\dagger\Psi+mc^2\Psi^\dagger\beta \Psi
     \\
     -i\hbar\partial_t\Psi^\dagger&=c(-\oppv-q\Av)\Psi^\dagger\alv+q\phi\Psi^\dagger+mc^2\Psi^\dagger\beta\\
     &\Rightarrow\qquad\qquad
     -i\hbar\partial_t\Psi^\dagger\Psi=c(-\oppv-q\Av)\Psi^\dagger\alv\Psi+q\phi\Psi^\dagger\Psi+mc^2\Psi^\dagger\beta\Psi.
    }
    Végezzük el a szokásos kivonást:
    \al{
     0
      &=
       i\hbar\partial_t\big(\Psi^\dagger\Psi\big)-c\Big(\Psi^\dagger\alv\oppv\Psi+\oppv\Psi^\dagger\alv\Psi\Big)\\
     0&=
       \partial_t\big(\Psi^\dagger\Psi\big)+\divo{\Big(c\Psi^\dagger\alv\Psi\Big)}.
    }
    Ezekkel a valószínűségi-sűrűség és a valószínűségi áramsűrűség:
    \al{
     &\rho=\Psi^\dagger\Psi
     &\Jv=c\Psi^\dagger\alv\Psi.
    }
    
    A kontinuitási egyenlet kovariáns alakja: 
    \eq{
     \partial_\mu\minv{J}^\mu=0.
    }
    
  \subsection{Alagúteffektus 1D-ban}
   
   A Schrödinger-egyenletet oldjuk meg 1D-ban, ahol: 
   \eq{
    V(x)=\begin{cases}
          V_0 &\text{ha } x\in[-a,0]\\
          0 & \text{ha } x\notin[-a,0]
         \end{cases}
   }
   Osszuk a tartományt 3 részre, legyen $x\in[-\infty,-a]$ az I., $x\in[-a,0]$ a II. és $x\in[0,\infty]$ a III. tartomány. Ahol a $V$ nulla, ott szabad hullám (vagy azok lineáris kombinációja) a megoldás:
   \al{
    &-\frac{\hbar^2}{2m}\der{^2}{x^2}\psi(x)=E\psi(x)&
    &\Rightarrow&
    &\psi(x)=\text{Const}\cdot e^{ikx}&
    &k=\sqrt{\frac{2mE}{\hbar^2}},&
   }
   ahol pedig van potenciál, ott:
   \al{
    &\left(-\frac{\hbar^2}{2m}\der{^2}{x^2}+V_0\right)\psi(x)=E\psi(x)&
    &\Rightarrow&
    &\psi(x)=\text{Const}\cdot e^{i\alpha x}&
    &\alpha=\sqrt{\frac{2m(E-V_0)}{\hbar^2}}.&
   }
   
   Az I. tartományon van egy jobbra haladó beérkező hullám ($A$), illetve egy visszavert ($B$) hullám. A III. tartományon csak transzmittált hullám található ($C$). A II. tartományban az I.-hez hasonlóan előre és hátrafelé haladó hullám is van. A próbafüggvények ezért legyenek:
   \al{
    \psi_\text{I.}&=Ae^{ikx}+Be^{-ikx}\\
    \psi_\text{II.}&=Fe^{i\alpha x}+Ge^{-i\alpha x}\\
    \psi_\text{III.}&=Ce^{ikx},
   }
   a feladat megoldása pedig akkor sikeres, ha megadtuk a próbafüggvényekben szereplő paramétereket. 
   
   Az $A$ paraméterrel tudjuk skálázni a többi paraméter nagyságát, úgyhogy a többi paraméter ehhez viszonyított értékét tudjuk csak kiszámolni. Tekintsük először az áramsűrűség:
   \al{
    J_\text{I.}
     &=\frac{i\hbar}{2m}\left(\psi\der{}{x}\psi^*-\psi^*\der{}{x}\psi\right)\\
     &=\frac{i\hbar}{2m}\left[\big(Ae^{ikx}+Be^{-ikx}\big)\der{}{x}\big(A^*e^{-ikx}+B^*e^{ikx}\big)-\big(A^*e^{-ikx}+B^*e^{ikx}\big)\der{}{x}\big(Ae^{ikx}+Be^{-ikx}\big)\right]\\
     &=\frac{i\hbar}{2m}\bigg[\big(Ae^{ikx}+Be^{-ikx}\big)\big(-ikA^*e^{-ikx}+ikB^*e^{ikx}\big)\\
     &\qquad\qquad\qquad-\big(A^*e^{-ikx}+B^*e^{ikx}\big)\big(ikAe^{ikx}-ikBe^{-ikx}\big)\bigg]\\
     &=\underbrace{\frac{\hbar k}{m}\abs{A}^2}_{J_\text{be}}-\underbrace{\frac{\hbar k}{m}\abs{B}^2}_{J_\text{vissza}},
   }
   innen pedig látszik, hogy a reflexió: $R=\frac{\abs{B}^2}{\abs{A}^2}$ és a transzmisszió $T=\frac{\abs{C}^2}{\abs{A}^2}$. Az anyagmegmaradás miatt:
   \eq{1=R+T.}
   
   Vegyük sorra, hogy milyen határfeltételeink vannak a hullámfüggvényre:
   \al{
    \psi_\text{I.}(x=-a)&=\psi_\text{II.}(x=-a)\\
    \psi_\text{II.}(x=0)&=\psi_\text{III.}(x=0)\\
    \der{}{x}\psi_\text{I.}(x=-a)&=\der{}{x}\psi_\text{II.}(x=-a)\\
    \der{}{x}\psi_\text{II.}(x=0)&=\der{}{x}\psi_\text{III.}(x=0)
   }
   Ez a négy összefüggés behelyettesítve:
   \aln{
    Ae^{-ika}+Be^{ika}&=Fe^{-i\alpha a}+Ge^{i\alpha a}\label{eq:08-p1}\\
    F+G&=C\label{eq:08-p2}\\
    k\big(Ae^{-ika}-Be^{ika}\big)&=\alpha\big(Fe^{-i\alpha a}-Ge^{i\alpha a}\big)\label{eq:08-p3}\\
    \alpha\big(F-G\big)&=kC.\label{eq:08-p4}
   }
   \Eqaref{eq:08-p2} és \eqaref{eq:08-p4} egyenletből: 
   \al{
    &F=\frac{1}{2}\left(1+\frac{k}{\alpha}\right)C
    &G=\frac{1}{2}\left(1-\frac{k}{\alpha}\right)C,
   }
   melyeket visszahelyettesítve a maradék két egyenletbe:
   \aln{
    Ae^{-ika}+Be^{ika}&=\frac{C}{2\alpha}\left[\left(\alpha+k\right)e^{-i\alpha a}+\left(\alpha-k\right)e^{i\alpha a}\right]\label{eq:08-b1}\\
    Ae^{-ika}-Be^{ika}&=\frac{C}{2k}\left[\left(\alpha+k\right)e^{-i\alpha a}-\left(\alpha-k\right)e^{i\alpha a}\right].\label{eq:08-b2}
   }
   $k$-val elosztva a másodikat, majd a kettőt egymásból kivonva, illetve összeadva megkaphatjuk a keresett $A$--$B$ és $A$--$C$ arányt:
   \al{
    2Ae^{-ika}
     &=\left(\frac{C}{2\alpha}+\frac{C}{2k}\right)\left(\alpha+k\right)e^{-i\alpha a}
      +\left(\frac{C}{2\alpha}-\frac{C}{2k}\right)\left(\alpha-k\right)e^{i\alpha a}\\
    \frac{C}{A}
     &=\frac
       {4e^{-ika}}
       {\left(\frac{1}{\alpha}+\frac{1}{k}\right)\left(\alpha+k\right)e^{-i\alpha a}+\left(\frac{1}{\alpha}-\frac{1}{k}\right)\left(\alpha-k\right)e^{i\alpha a}}
       \\
     &=\frac
       {4\alpha k e^{i(\alpha-k)a}}
       {\left(\alpha+k\right)^2-\left(\alpha-k\right)^2e^{2i\alpha a}}.
   }
   Ehhez teljesen hasonlóan a különbségből a másik arány:
   \eq{
    \frac{B}{A}
     =\frac
      {(k^2-\alpha^2)\big(1-e^{i(\alpha-k)a}\big)}
      {\left(\alpha+k\right)^2-\left(\alpha-k\right)^2e^{2i\alpha a}}.
   }
   Ezek abszolút érték négyzetéből megkaphatjuk a $T$-t és az $R$-et:
   \aln{
    &R=\frac{1}{1+\frac{4E(E-V_0)}{V_0^2\sin^2(\alpha a)}}
    &T=\frac{1}{1+\frac{V_0^2\sin^2(\alpha a)}{4E(E-V_0)}}.\label{eq:08-ale}
   }
   
   Vizsgáljuk a transzmissziós együtthatót. Láthatjuk, hogy vonzó potenciálra sem 1 a transzmisszió, csak speciális rezonáns állapotokban ($\alpha a=\pi n|n\in\mathbb{Z}$). Az áthaladás nulla, ha $m\to\infty$, $V_0\to\infty$ vagy ha $a\to\infty$. Abban az esetben, ha $E\le V_0$, akkor a fenti formula kicsit módosul ($\alpha=i\beta$):
   \eq{
    T=\frac{1}{1+\frac{V_0^2\sinh^2(\beta a)}{4E\abs{E-V_0}}}
     \xrightarrow{E\ll V_0\text{ vagy }\beta a\gg 1}
     \frac{16E(V_0-E)}{V_0^2}e^{-2\beta a}.
   }
   Láthatjuk, hogy akkor is van transzmisszió, ha a hullám energiája kisebb, mint a potenciálfal magassága. Ez az alagúteffektus.