\chapter{A klasszikus fizika \'es a nemrelativisztikus kvantummechanika alapegyenletei}\label{1tetel}
 
 \section{Mechanika}
  
  \subsection{Newton axiómák}
   
   Tér lokálisan euklideszi.
   Ebben választunk egy kitüntetett pontot (origó) és egy koordináta-rendszert, ez egy vonatkoztatási rendszer.
   Tömegpontok  helyzetének leírása vektorokkal: $\vect{r}(t)$, itt $t$ az idő, paraméter.
   Kinematika: mozgás leírása: $\frac{\dd\vect{r}}{\dd t}= \vect{v}(t)$, $\frac{\dd\vect{v}}{\dd t}= \vect{a}(t)$\dots. 
   
   Newton axiómái tömegpontokra:
   
   {\bf Newton I. axiómája:} Minden tömegpont nyugalomban marad vagy egyenes vonalú egyenletes mozgást végez mindaddig, míg ezt az állapotot egy másik tömegpont vagy mező meg nem változtatja.
   Egy vonatkoztatási rendszer akkor inerciarendszer, ha ez a törvény minden esetben teljesül.
   
   Következmény, ha egy tömegpontnak változik a mozgásállapota, akkor van gyorsulása, aminek az oka egy másik tömegponttal való kölcsönhatás: erőhatás.
   Mozgásállapot jellemző mennyisége az impulzus: $\vect{p}=m\vect{v}$.
   Ahol $m$ a tömeg.
   Tömeg definíciója: a) {\it dinamikai tömeg}: egy álló és egy mozgó test tökéletesen rugalmas ütközésénél: $m_1 \Delta v_1=m_2 \Delta v_2$.
   Tömeg egysége definíció alapján (Párizs, platina-irídium), b) {\it súlyos tömeg}: Nehézségi erő által előidézett gyorsulásváltozás: $\vect{G}=m_s g$.
   Ezek egyenlőek Eötvös ingakísérletei alapján.
   
   {\bf Newton II. axiómája:} Egy tömegpont lendületének (impulzusának) egyenlő a testre ható $\vect{F}$ erővel. $\vect{F}=\frac{\dd \vect{p}}{\dd t}$. 
   
   {\bf Newton III. axiómája:} Két tömegpont kölcsönhatása során mindkét tömegpont azonos nagyságú, egymással ellentétes irányú erő hat.
   Ez a hatás-ellenhatás törvénye. 
   
   {\bf Newton IV. törvénye:} Több erő együttes hatása megegyezik az erők eredőjének (vektori összegének) hatásával. 
   
  \subsection{A virtuális munka elve (statika)}
   
   Tekintsünk $n$ darab tömegpontból álló pontrendszert.
   A pontokra ható erők lehetnek szabad- és kényszererők.
   Kényszererő: olyan erő, amelynek irányában nem lehetséges elmozdulás.
   A pontrendszer egyensúlyban van, ha a tömegpontokra ható erők eredője nulla tömegpontonként: $\forall i\in [1,n]:\; \vect{F}_i=0$.
   Ezzel ekvivalens a 
   \eqn{
    \delta A:=\sum\limits_{i=1}^{n}\vect{F}_i\delta\vect{r}_i =0\label{eq:01vme}
   }
   kifejezés, ahol $\delta A$ a virtuális munka, $\vect{F}_i$ az $i$-edik tömegpontra ható erők eredője, $\delta\vect{r}_i$ pedig az $i$-edik tömegpont virtuális elmozdulása: nagysága infinitezimálisan kicsi és bármerre mutathat, amerre a test el tud mozdulni.
   Ebből következik, hogy csak a szabad erőket kell \eqaref{eq:01vme} egyenletbe írni. 
   
   $\delta A=0$ $\Rightarrow$ $\forall i\in [1,n]:\; \vect{F}_i=0$, így az egyensúly akkor és csak akkor állhat fenn, ha $\delta A=0$.
   
   Áttérés egységes jelölésre: $[\vect{r}_1]_1=x_1$, $[\vect{r}_1]_2=x_2$,\dots,$[\vect{r}_n]_3=x_{3n}$; $[\vect{F}_1]_1=X_1$, $[\vect{F}_1]_2=X_2$,\dots, $[\vect{F}_n]_1=X_{3n}$:
   \eqn{
    \delta A:=\sum\limits_{i=1}^{3n}X_i\delta x_i. \label{eq:01vme-alt}
   }
   Kényszerfeltételek egyenletek formájában\footnote{Itt csak a holonom kényszerekkel foglalkozunk.
   Továbbiak \aref{ss3:kenyszerfeletetelek}. fejezetben.}: 
   \eqn{
    \phi_j(x_1,x_2,\dots,x_{3n},t)=0\qquad\forall j=1\dots s.\label{eq:01-kenyszerek}
   }
   Ezek az egyenletek a mozgás során végig igazak, ezeket idő szerinte deriválva:
   \al{
    \der{\phi_j}{t}=\suml{i=1}{3n}\pder{\phi_j}{x_i}\der{x_i}{t}+ \pder{\phi_j}{t}&=0 \qquad \forall j=1\dots s \\
    \dd\phi_j=\suml{i=1}{3n}\pder{\phi_j}{x_i}\dd x_i+ \pder{\phi_j}{t}\dd t&=0 
   }
   A virtuális elmozdulások egy időpillanatban történnek meg, így a köztük lévő kapcsolatot a $\dd t=0$-ban keressük:
   \eqn{
    \delta\phi_j=\suml{i=1}{3n}\pder{\phi_j}{x_i}\delta x_i=0 \label{eq:01-kenyszer}
   }
   Ez az egyenlet a virtuális elmozdulások közötti kapcsolatokat fejezi ki.  Ezeket a kényszerfeltételeket be kell foglalni a virtuális munka elvébe.
   Ezt megtehetjük Lagrange-multiplikátorokkal: 
   \eq{
    \delta A:
    =\sum\limits_{i=1}^{3n}X_i\delta x_i + \suml{j=1}{s}\lambda_j\delta\phi_j
    =\sum\limits_{i=1}^{3n} X_i\delta x_i + \suml{j=1}{s}\lambda_j\suml{i=1}{3n} \pder{\phi_j}{x_i}\delta x_i
    =\sum\limits_{i=1}^{3n} \left(X_i + \suml{j=1}{s}\lambda_j\pder{\phi_j}{x_i}\right)\delta x_i=0
   }
   Ebből pedig, mivel a $\delta x_i$-k függetlenek, az egyensúlyi egyenletek adódnak:
   \eqn{
    \boxed{X_i + \suml{j=1}{s}\lambda_j\pder{\phi_j}{x_i}=0 \qquad\forall i=1\dots 3n}\label{eq:01-ee}
   }
   $3n$ egyenlet, $3n+s$ ismeretlen, de hozzávesszük, \eqaref{eq:01-kenyszerek} egyenleteket, akkor a feladat jól definiált. 
  
  \subsection{A d'Alembert-elv (dinamika)}\label{ss1:dalembert}
   
   Ugyanaz, mint a virtuális munka elvénél, csak nem $\vect{F}_i$-re, hanem $X_i-m_i\ddot{x}_i$-re írjuk fel az összefüggést: $\delta A = \suml{i=1}{3n}(X_i-m_i\ddot{x}_i)\delta x_i=0$.
   Ez akkor és csak akkor nulla, ha teljesülnek a mozgásegyenletek.
   Kényszerfeltételek esetében:
   \eq{
    \delta A = \suml{i=1}{3n}(X_i-m_i\ddot{x}_i)\delta x_i+\suml{j=1}{s}\lambda_j\delta\phi_j=0,
   }
   ahonnan hasonlóan az következik, hogy 
   \eqn{
    \boxed{m_i\ddot{x}_i=X_i+\suml{j=1}{s}\lambda_j\pder{\phi_j}{x_i}}.\label{eq:01-LIE}
   }
   Ezek a Lagrange-féle elsőfajú egyenletek. 
   
   Kényszerek fajtái, osztályozása \aref{ss3:kenyszerfeletetelek}. fejezetben.

  \subsection{Általánosított koordináták, Lagrange-formalizmus}\label{ss1:lagrange2}
   
   Praktikus okok: Descartes-féle koordináta-rendszer nem mindig kényelmes.
   Holonom kényszerek esetében a független változók száma: $3n-s=f$, ez a szabadsági fokok száma, ennyi mennyiséggel le lehet írni a rendszert: ezek az általánosított koordináták: $q_1$, $q_2$,\dots, $q_f$. 
   
   Koordinátatranszformáció:
   \eq{
    x_i=x_i(q_1, q_2,\dots,q_f,t)\qquad\forall i=1\dots 3n
   }
   Transzformáljuk a d'Alembert-elvet az általánosított koordináta-rendszerbe:
   \al{
    0 &= \suml{i=1}{3n}(X_i-m_i\ddot{x}_i)\delta x_i 
      = \suml{i=1}{3n}\left( X_i-m_i\ddot{x}_i \right)\left(\suml{j=1}{f}\pder{x_i}{q_j}\delta q_j\right)
      = \suml{j=1}{f}\left[\underbrace{\suml{i=1}{3n} X_i\pder{x_i}{q_j}}_{:=Q_j}-\underbrace{\suml{i=1}{3n}m_i\ddot{x}_i\pder{x_i}{q_j}}_{\text{II.}}\right]\delta q_j
   }
   A második tagot átalakítjuk.
   Ehhez az alábbi azonosságokra van szükség: a koordinátatranszformáció definíciójába deriválva:
   \aln{
    \dtx_i&=\suml{k=1}{f}\pder{x_i}{q_k}\dtq_k+\pder{x_i}{t} & \Rightarrow & & \pder{\dtx_i}{\dtq_j}&=\pder{x_i}{q_j}, \label{eq:01-altd}
   }
   illetve az alábbi átalakításokat elvégezve:
   \eqn{
    \der{}{t}\pder{x_i}{q_j}
    = \suml{k=1}{f}\pder{}{q_k}\pder{x_i}{q_j}\dtq_k + \pder{}{t}\pder{x_i}{q_j}
     = \pder{}{q_j}\left[\suml{k=1}{f}\pder{x_i}{q_k}\dtq_k + \pder{}{t}x_i\right ] 
     = \pder{}{q_j}\left[\der{x_i}{t}\right ] 
     = \pder{\dtx_i}{q_j}. \label{eq:01-dtdqcsere}
   }
   
   Ezek felhasználásával a nagy zárójel második tagja:
   \eq{
    \text{II.}=\suml{i=1}{3n}m_i\ddot{x}_i\pder{x_i}{q_j}
    = \suml{i=1}{3n}m_i\left[\der{}{t}\left(\dtx_i\pder{x_i}{q_j}\right)-\dtx_i\der{}{t}\left(\pder{x_i}{q_j}\right)\right],
   }
   ahol az első részben \eqaref{eq:01-altd} egyeneltet a másodikban pedig \eqaref{eq:01-dtdqcsere} egyenletet használjuk fel:
   \eq{
    \text{II.}
     = \suml{i=1}{3n}m_i\left[\der{}{t}\left(\dtx_i\pder{\dtx_i}{\dtq_j}\right)-\dtx_i\pder{\dtx_i}{q_j}\right]
     = \suml{i=1}{3n}\frac{m_i}{2}\left[\der{}{t}\pder{\dtx_i^2}{\dtq_j}-\pder{\dtx_i^2}{q_j}\right]
   }
   A kinetikus energia definícióját $\left(K=\suml{i=1}{3n}\frac{m_i}{2}\dtx_i^2\right)$ felhasználva:
   \eq{
    \text{II.}
     = \der{}{t}\pder{K}{\dtq_j}-\pder{K}{q_j}.
   }
   Így az általánosított d'Alembert-elv a
   \eqn{
    \suml{j=1}{f}\left[Q_j-\der{}{t}\pder{K}{\dtq_j}+\pder{K}{q_j}\right]\delta q_j=0
   }
   alakot ölti, mely akkor és csak akkor teljesül, ha a szögletes zárójelben szereplő mennyiség minden $j$-re eltűnik:
   \eq{
    Q_j=\der{}{t}\pder{K}{\dtq_j}-\pder{K}{q_j}
   }
   Tegyük fel, hogy a pontrendszerre ható erők potenciálosak, akkor $Q_j$-k átalakíthatóak:
   \eq{
    Q_j=\suml{i=1}{3n} X_i\pder{x_i}{q_j}=-\suml{i=1}{3n} \pder{U}{x_i}\pder{x_i}{q_j}=-\pder{U}{q_j}.
   }
   Az előző összefüggés átalakításához kihasználjuk azt is, hogy a potenciálok $\dtq_i$-ktől nem függnek:
   \al{
    \der{}{t}\pder{K}{\dtq_j}-\pder{K}{q_j}&=-\pder{U}{q_j} \\
    \der{}{t}\pder{(K-U)}{\dtq_j}-\pder{K}{q_j}&=-\pder{U}{q_j} \\
    \der{}{t}\pder{(K-U)}{\dtq_j}-\pder{(K-U)}{q_j}&=0. \\
   }
   Itt bevezetve az Lagrange-függvényt, melynek független változói $q$, $\dtq$ és $t$:
   \eqn{
    L=L(q,\dtq,t)=K-U,
   }
   a másodfajú Lagrange-egyenleteket kapjuk:
   \eqn{
    \boxed{0=\der{}{t}\pder{L}{\dtq_j}-\pder{L}{q_j} \qquad\forall j=1\dots f.}
   }
   
  \subsection{Hamilton-elv}\label{ss1:hamiltonelv}
   
   A Lagrange-egyenletet más formalizmussal is le lehet vezetni.
   Tekintsük az 
   \eqn{
    S=\intl{t_1}{t_2}\dd t L = \intl{t_1}{t_2}\dd t L(q,\dtq,t)
   }
   mennyiséget.
   Ez a hatás.
   A Hamilton-elv kimondja, hogy a hatás variációja akkor és csak akkor tűnik el, ha a rendszer a mozgásegyenletek szerint mozog, azaz teljesülnek a másodrendű Lagrange-egyenletek.
   A variálás a pályák ($q$) szerint zajlik.
   
   Készítsük el a hatás variációját: 
   \al{
    \delta S 
     &= \delta \intl{t_1}{t_2}\dd t L(q_1,\dots,q_f,\dtq_1,\dots,\dtq_f,t) \\
     &= \intl{t_1}{t_2}\dd t L(q_1+\delta q_1,\dots,q_f+\delta q_f,\dtq_1+\delta \dtq_1,\dots,\dtq_f+\delta \dtq_f,t)-\intl{t_1}{t_2}\dd t L(q_1,\dots,q_f,\dtq_1,\dots,\dtq_f,t) \\
     &= \intl{t_1}{t_2}\dd t \delta L 
      = \intl{t_1}{t_2}\dd t \left[\suml{i=1}{f}\left(\pder{L}{q_i}\delta q_i+\pder{L}{\dtq_i}\delta \dtq_i\right)\right]
      = \suml{i=1}{f}\intl{t_1}{t_2}\dd t \left(\pder{L}{q_i}\delta q_i+\pder{L}{\dtq_i}\delta \dtq_i\right) \\
     &= \suml{i=1}{f}\left[\intl{t_1}{t_2}\dd t \pder{L}{q_i}\delta q_i+\underbrace{\intl{t_1}{t_2}\dd t\pder{L}{\dtq_i}\delta \dtq_i}_{\text{II.}}\right]
   }
   Felhasználjuk, hogy a deriválás idő szerint és a variálás felcserélhető, hiszen $t$-ben nem variálunk. Így a második tag:
   \eq{
   \text{II.}=\intl{t_1}{t_2}\dd t \pder{L}{\dtq_i}\der{\delta q_i}{t} = \{\text{parc.int}\}
    = \underbrace{\left[\pder{L}{\dtq_i}\delta q_i\right]_{t_1}^{t_2}}_{=0} - \intl{t_1}{t_2}\dd t \der{}{t}\pder{L}{\dtq_i}\delta q_i,
   }
   vagyis a variáció:
   \eq{
    \delta S = \intl{t_1}{t_2}\dd t\suml{i=1}{f} \left[\pder{L}{q_i}-\der{}{t}\pder{L}{\dtq_i}\right]\delta q_i
   }
   
   Ez akkor és csak akkor lehet nulla, ha minden $i$-re eltűnik a zárójelben szereplő tag külön-külön. Így:
   \eq{
    \pder{L}{q_i}-\der{}{t}\pder{L}{\dtq_i}=0 \qquad\forall i=1,\dots, f,
   }
   vagyis megkaptuk a másodfajú Lagrange-egyenleteket. 
  
  \subsection{Kanonikus-formalizmus}
   
   Definiáljuk a Hamilton-függvényt és a kanonikus impulzust:
   \aln{
    H&=H(p,q,t)=\suml{i=1}{f}p_i\dtq_i-L, & \text{ahol} & & p_i=\pder{L}{\dtq_i}.
   }
   A Hamilton-függvény saját változói a $q$ a $p$ és $t$.
   Készítsük el a Hamilton-függvény deriváltjait:
   \eq{
    \pder{H}{q_k} = \suml{i=1}{f}p_i\pder{\dtq_i}{q_k}-\suml{i=1}{f}\pder{L}{\dtq_i}\pder{\dtq_i}{q_k}-\pder{L}{q_k} = -\pder{L}{q_k} = -\der{}{t}\pder{L}{\dtq_k} = -\dtp_k
   }
   \eq{
    \pder{H}{p_k} = \dtq_k +\suml{i=1}{f}p_i\pder{\dtq_i}{p_k}-\suml{i=1}{f}\pder{L}{\dtq_i}\pder{\dtq_i}{p_k} = \dtq_k
   }
   \eq{
    \pder{H}{t}=\suml{i=1}{f}p_i\pder{\dtq_i}{t}-\suml{i=1}{f}\pder{L}{\dtq_i}\pder{\dtq_i}{t} -\pder{L}{t}= -\pder{L}{t}
   }
   
   Összefoglalva a három kanonikus-egyenlet:
   \aln{
    &\boxed{\pder{H}{q_k} =-\dtp_k} & &\boxed{\pder{H}{p_k} =\dtq_k }& \boxed{\pder{H}{t}=-\pder{L}{t}} \label{eq:01-kanonikus}
   }
   
   \subsection{Módosított Hamilton-elv}
   
   A Hamilton-elvnél az $S=\intl{t_1}{t_2}\dd t\,L(q,\dtq,t)$ hatás variációit számoltuk.
   Itt a variálás a $q_i$, $i=1\dots f$ szerint történt.
   Azonban a kanonikus formalizmusban a $q_i$-ket és a $p_i$-ket független koordinátánként kezeljük.
   Kérdés, hogy így is igaz-e a Hamilton-elv.
   Tekintsük az $L=\suml{i=1}{f}p_i\dtq_i-H(q,p,t)$ Lagrange-függvényt, és nézzük meg, hogy az ezzel felírt hatás variációja mikor tűnik el:
   \al{
    \delta S
     &=\intl{t_1}{t_2}\dd t\,\delta\left(\suml{i=1}{f}p_i\dtq_i-H(q,p,t)\right)
      =\intl{t_1}{t_2}\dd t\,\suml{i=1}{f}\left(\delta p_i\dtq_i+p_i\delta\dtq_i-\pder{H}{q_i}\delta q_i-\pder{H}{p_i}\delta p_i\right) \\
     &=\intl{t_1}{t_2}\dd t\,\suml{i=1}{f}\left(\delta p_i\dtq_i-\pder{H}{q_i}\delta q_i-\pder{H}{p_i}\delta p_i\right)
       +\suml{i=1}{f}\intl{t_1}{t_2}\dd t\,p_i\delta\dtq_i \\
     &=\intl{t_1}{t_2}\dd t\,\suml{i=1}{f}\left(\delta p_i\dtq_i-\pder{H}{q_i}\delta q_i-\pder{H}{p_i}\delta p_i\right)
       +\suml{i=1}{f}\left(\underbrace{\left[p_i\delta q_i\right]_{t_1}^{t_2}}_{=0}-\intl{t_1}{t_2}\dd t\,\dtp_i\delta q_i \right)\\
     &=\intl{t_1}{t_2}\dd t\,\suml{i=1}{f}\left[
     \left(\dtq_i-\pder{H}{p_i}\right)\delta p_i-\left(\dtp_i+\pder{H}{q_i}\right)\delta q_i
     \right]
   }
   
   Ennek megfelelően itt a hatás egy $2f$ dimenziós térben az $S=\intl{t_1}{t_2}\dd t\,L(q,p,t)$ vonalintegrál.
   Ebben a térben a $\big\{\{q_i\}_{i=1}^{f},\{p_i\}_{i=1}^{f}\big\}$ koordináták lineárisan függetlenek.
   A módosított Hamilton-elv pedig azt mondja ki, hogy az ebben a térben végzett variáció is pontosan akkor tűnik el, ha a rendszer a mozgásegyenlet szerint mozog.
   
  \subsection{Hamilton--Jacobi-egyenlet}
   
   A cél, hogy a kanonikus egyenleteket a legegyszerűbb alakra hozzuk, vagyis:
   \al{
    &\pder{H'}{P_i}=\dtQ_i=0 &\pder{H'}{Q_i}=-\dtP_i=0
   }
   A Hamilton--Jacobi-egyenlethez egy 2-es típusú kanonikus transzformációval tudunk eljutni.
   Ennek alkotófüggvénye legyen $S=S(q_i,P_i,t)$ alakú, a transzformációs szabályok:
   \aln{
    &p_i=\pder{S}{q_i}
    &Q_i=\pder{S}{P_i}&
    &H'=H+\pder{S}{t}.
   }
   Akkor lesznek triviálisak az új kanonikus egyenletek, ha $H'=0$, vagyis 
   \al{
    H(q,p,t)+\pder{S(q,P,t)}{t}=0.
   }
   Mivel az új impulzusok ciklikusak, ezért $P_i=\alpha_i=\text{állandó}$.
   Az áttérésre vonatkozó egyenletek alapján:
   \aln{
    0=H\left(q,\pder{S}{q},t\right)+\pder{S(q,P,t)}{t},\label{eq:01-hatas}
   }
   ami $S$-re egy parciális differenciálegyenlet.
   Készítsük el $S$ deriváltját:
   \al{
    \der{S}{t}=\suml{i=1}{f}\pder{S}{q_i}\dtq_i+\pder{S}{t}
     =\suml{i=1}{f}p_i\dtq_i-H
     =L,
   }
   vagyis $S(t)=S(0)+\intl{0}{t}\dd t\,L$, így $S$ maga a hatás.
   Ha sikerül megoldani \eqaref{eq:01-hatas} egyenletet $S$-re, akkor a kanonikus egyenletek megoldása már triviális.
   Tudjuk, hogy $Q_i=\beta_i=\text{állandó}$ és $P_i=\alpha_i=\text{állandó}$, és $S$ ismeretében azt is tudjuk, hogy hogyan lehet áttérni a $Q,P$-től $q,p$-re:
   \al{
    \beta_i=\pder{S}{\alpha_i},
   }
   ami egy algebrai egyenletrendszer ($i=1\dots f$), melyet meg lehet oldani $q_i$-ra.
   
 \section{Elektrodinamika}
  
  \subsection{Elektrosztatika vákuumban}\label{ss:01-CoulombMaxwell}
   
   Elektrodinamika alapfogalma a mező: a fizikai tér minden pontjában minden időpillanatban létező mennyiség.
   A testek a mezővel hatnak kölcsön.
   Megfigyelés: töltések közötti erő arányos a töltéssel, távolság reciproknégyzetével $\vect{F}\sim\frac{q_1q_2}{r^2}$, iránya a töltésekre fektetett egyenesre esik.  Egységválasztás: $k=9\cdot 10^9\me{\frac{Nm^2}{C^2}}$
   \footnote{SI-ben az alapmennyiség az Amper.
   Ennek definíciója: két egyenes vezető egymástól 1 méterre, 1 Ampert áthajtva $2\cdot 10^{-7}\me{N}$ nagyságú erővel hatnak egymásra.
   A töltés definíciója az Amper alapján: C$=$As.
   Ebből már következik az $\ep_0$ értékének választása.}
   .
   A Coulomb-törvény ponttöltések által létrehozott térerősségre mint mezőre felírva:
   \eq{
    \vect{E}(\vect{r}):=\suml{i}{}\frac{q_i}{4\pi\ep_0}\frac{\vect{r}-\vect{r}_i}{\abs{\vect{r}-\vect{r}_i}^3}.
   }
   Ha a töltéseloszlás folytonos, vagy jó közelítéssel annak tekinthető, akkor :
   \eq{
    \vect{E}(\vect{r}):=\int\drkh\frac{q_i}{4\pi\ep_0}\frac{\vect{r}-\vect{r'}}{\abs{\vect{r}-\vect{r'}}^3}.
   }
   
   A ponttöltés tere zárt felületre integrálva:
   \eq{
    \oint\df\vect{E}
     = \oint\dd f\; \vect{n}\vect{E} 
      = \int\dd \Omega\; \underbrace{\frac{r^2}{\cos\vartheta}}_\text{Jacobi}\frac{q}{4\pi\ep_0}\frac{1}{r^2}\underbrace{\cos\vartheta}_\text{vetület} = \frac{q}{4\pi\ep_0}\int\dd \Omega = \begin{cases}
                                   \frac{q}{\ep_0},\quad\text{ha }q\in V \\
                                   0,\quad\text{ha }q\notin V
                                  \end{cases}
   }
   Kiterjedt töltéseloszlásra:
   \eq{
    \oint\limits_{\partial V}\df\vect{E}(\vect{r}) = \frac{1}{\ep_0}\intl{V}{}\drh \rho(\vect{r}).
   }
   Ez a Stokes-tétellel átalakítva az alábbi lokális formát veszi fel:
   \eqn{
    \divo{\vect{E}(\vect{r})}=\frac{1}{\ep_0}\rho{(\vect{r})}.\label{eq:01-MX1s}
   }
   
   A ponttöltés tere felírható egy $\phi(\vect{r})$ skalárfüggvény negatív gradienseként, ez a potenciálfüggvény:
   \al{
    &\phi(\vect{r})=\frac{1}{4\pi\ep_0}\frac{1}{\abs{\vect{r}-\vect{r}_0}} & \vect{E}(\vect{r})=-\grad{\phi(\vect{r})}
   }
   
   Kiterjedt töltéseloszlásra hasonlóan:
   \aln{
    \phi(\vect{r})=\frac{1}{4\pi\ep_0}\int\drkh\frac{\rho(\vect{r}')}{\abs{\vect{r}-\vect{r}'}}.\label{eq:01-pot}
   }
   
   A potenciál definícióját behelyettesítve \eqaref{eq:01-MX1s}. egyenletbe, a Poisson-egyenletet kapjuk:
   \eqn{
    \Delta\phi( \vect{r})=-\frac{1}{\ep}\rho(\vect{r}).\label{eq:01-Poi}
   }
   
   Mivel a térerősség gradiensként áll elő, azért annak rotációja mindenképp eltűnik:
   \eqn{
    \rot{\vect{E}(\vect{r})}=0.\label{eq:MX3s}
   }
   
  \subsection{Elektrosztatika anyagban}\label{ss1:elsztat}
   
   Ha az anyagban vannak szabadon elmozdítható töltéshordozók, akkor azok addig áramlanak, amíg hat rájuk erő, vagyis míg $\vect{E}(\vect{r})\ne 0$ az anyagban.
   Ha nincsenek szabadon elmozduló töltések, akkor az anyagot az elektromos tér csak polarizálni tudja.
   A polarizáció során elemi dipólmomentumok jönnek lére az anyagban. 
   
   Egy $\rho(\vect{r})$ töltéseloszlás  hatására $\vect{P}(\vect{r})$ dipólmomentum-sűrűség (\ref{ss:A06-dipol}. fejezet) jön létre.
   Ezek együttes potenciálja:
   \eq{
    \phi(\vect{r})=\frac{1}{4\pi\ep_0}\int\drkh\frac{\rho(\vect{r}')}{\abs{\vect{r}-\vect{r}'}} + \frac{1}{4\pi\ep_0}\int\drkh\left(-\vect{P}(\vect{r}')\grad\frac{1}{\abs{\vect{r}-\vect{r}'}}\right).
   }
   A második tagban a deriválásban áttérünk az $\vect{r}'$-re, majd parciálisan integráljuk és elhagyjuk a felületi tagot, hiszen $\vect{P}(\vect{r})=0$, ha $r\to\infty$:
   \eq{
   \phi(\vect{r})=\frac{1}{4\pi\ep_0}\int\drkh\left(\frac{\rho(\vect{r}')-\grad{\vect{P}}(\vect{r}')}{\abs{\vect{r}-\vect{r}'}}\right).
   }
   
   Így tehát nem az eredeti $\rho(\vect{r})$ töltéssűrűséget látjuk, hanem az leárnyékolódik: 
   \aln{
   & \rho_\text{tot}(\vect{r})=\rho(\vect{r})+\rho_\text{ind}(\vect{r}) & \rho_\text{ind}(\vect{r})=-\divo{\vect{P}(\vect{r})},\label{eq:01-toltsuru}
   }
   $\rho_\text{ind}(\vect{r})$ töltéssűrűség indukálódik. 
   Az új töltéssűrűséget behelyettesítve \eqaref{eq:01-MX1s} egyenletbe, és definiálva a dielektromos eltolást ($\vect{D}(\vect{r})$):
   \al{
    & \vect{D}(\vect{r})=\ep_0\vect{E}(\vect{r})+\vect{P}(\vect{r}) & \divo{\vect{D}}(\vect{r})=\rho(\vect{r}).
   }
   
   Az anyag viselkedéséhez szükséges megadni a $\vect{D}(\vect{r})$, a $\vect{P}(\vect{r})$ és az $\vect{E}(\vect{r})$ mennyiségek között egy összefüggést.
   Ez az anyagi egyenlet.
   Lineáris anyagokra: $\vect{P}(\vect{r})=\ep_0\mat{\chi}\vect{E}(\vect{r})$.
   Ha az anyag általános, akkor $\mat\chi$ egy $3\times 3$-as tenzor.
   Ha az anyag izotrop, akkor $\mat\chi=\chi\mat 1$, vagyis $\vect{P}(\vect{r})=\ep_0\chi\vect{E}(\vect{r})$.
   Ekkor definiálhatjuk egy anyag relatív dielektromos állandóját: $\ep_r=1+\chi$, amivel $\vect{D}(\vect{r})=\ep_0\ep_r\vect{E}(\vect{r})$. 
   
   A térerősségre és a dielektromos eltolásra vonatkozó határfeltételeket a megfelelő tartományra való integrálással kaphatjuk.
   Különböző anyagi tulajdonsággal rendelkező tartományok határfelületén a dielektromos eltolás normális, illetve a térerősség tangenciális komponense halad át változatlanul. 
   
   A potenciálra vonatkozó határfeltételek: mivel a térerősség véges, ezért a potenciálnak folytonosnak kell lennie. $E_n$ ugrása a potenciál felületre merőleges gradiensének ugrását implikálja: $\ep_1\pder{\phi}{\vect{n}_1}=\ep_2\pder{\phi}{\vect{n}_2}$. 
   
  \subsection{Magnetosztatika vákuumban}
   
   Tekintsünk egy $\rho(\vect{r})$ töltéssűrűséget, amely $\vect{v}(\vect{r})$ sebességmezővel leírható áramlást végez.
   Ekkor ez az áramlás mágneses mezőt hoz létre (Biot--Savart-törvény):
   \al{
    &\vect{B}(\vect{r})=\frac{\mu_0}{4\pi}\int\drkh\frac{\vect{J}(\vect{r}')\times(\vect{r}-\vect{r}')}{\abs{\vect{r}-\vect{r}'}^3}
    &\vect{B}(\vect{r})=\frac{\mu_0 I}{4\pi}\int\frac{\dd\vect{s}\times(\vect{r}-\vect{s})}{\abs{\vect{r}-\vect{s}}^3}.
   }
   A $\mu_0=4\pi\cdot 10^{-7}\me{\frac{N}{A^2}}$.
   A jobb oldali összefüggés vékony vezetőkre alkalmazható, ahol $\drh\,\vect{J}(\vect{r})\leftrightarrow I\,\dd \vect{s}$.
   
   A $\vect{v}(\vect{r})$ sebességgel mozgó töltésekre ható erő és forgatónyomaték:
   \al{
    &\vect{F}=\int\drh\vect{J}(\vect{r})\times\vect{B}(\vect{r}) &\vect{N}=\int\drh \vect{r}\times\big(\vect{J}(\vect{r})\times\vect{B}(\vect{r}) \big).
   }
   
   A Biot--Savart-törvénynek is van lokális alakja, felhasználva, hogy:
   \eq{
    \frac{\vect{J}(\vect{r}')\times(\vect{r}-\vect{r}')}{\abs{\vect{r}-\vect{r}'}^3}
     = -\vect{J}(\vect{r}')\times \grad_\vect{r}\left(\frac{1}{\abs{\vect{r}-\vect{r}'}}\right)
     =\rot_\vect{r}\left(\frac{\vect{J}(\vect{r}')}{\abs{\vect{r}-\vect{r}'}}\right).
   }
   Így bevezethető egy vektorpotenciál, hogy 
   \aln{
    &\vect{A}(\vect{r})=\frac{\mu_0}{4\pi}\int\drkh\frac{\vect{J}(\vect{r}')}{\abs{\vect{r}-\vect{r}'}}
    &\vect{B}(\vect{r})=\rot{\vect{A}(\vect{r})}
    & &\Rightarrow &
    &\divo{\vect{B}(\vect{r})}=0.\label{01-MX2s}
   }
   
   Kiszámolhatjuk $\vect{B}(\vect{r})$ rotációját is:
   \eq{
    \rot{\vect{B}(\vect{r})}=\rot(\rot\vect{A}(\vect{r}))=\grad\divo\vect{A}(\vect{r})-\Delta\vect{A}(\vect{r}).
   }
   Itt $\divo\vect{A}(\vect{r})=0$, hiszen a div csak az $1/x$-re hat, majd parciálisan integrálva a felületen kellene $\partial_i J_i$-t számolni, ami úgyis nulla.
   A második tagban:$\Delta\frac{1}{\abs{\vect{r}-\vect{r}'}} =4\pi\delta(\vect{r}-\vect{r}')$, vagyis:
   \eqn{
    \rot{\vect{B}(\vect{r})}=\mu_0\vect{J}(\vect{r}). \label{eq:01-MX4s}
   }
   
  \subsection{Magnetosztatika anyagban}\label{ss1:magnetosztatika}
   
   Hasonlóan az elektromos jelenségekhez is, itt is a külső mágneses tér hatására az anyag polarizálódik, és $\vect{M}(\vect{r})$ mágnesezettségi sűrűség alakul ki.
   A külső tér és ennek az együttes vektorpotenciálja:
   \eq{
    \vect{A}(\vect{r})=\frac{\mu_0}{4\pi}\int\drkh\frac{\vect{J}(\vect{r}')}{\abs{\vect{r}-\vect{r}'}}+\frac{\mu_0}{4\pi}\int\drkh\frac{\vect{M}(\vect{r}')\times(\vect{r}-\vect{r}')}{\abs{\vect{r}-\vect{r}'}^3}
   }
   A második tagot átalakíthatjuk: a törtet felírjuk gradiensként, a deriválást áthárítjuk $\vect{r}'$-re, parciálisan integrálunk és eldobjuk a felületi tagot, így:
   \eq{
    \vect{A}(\vect{r})=\frac{\mu_0}{4\pi}\int\drkh\frac{\vect{J}(\vect{r}')+\rot{\vect{M}(\vect{r}')}}{\abs{\vect{r}-\vect{r}'}}.
   }
   Itt is hasonló eredményt kaptunk, mint az elektromos esetben: 
   \eq{
    \vect{J}_\text{tot}(\vect{r})=\vect{J}(\vect{r})+\rot\vect{M}(\vect{r})
   }
   Ezt behelyettesítve \eqaref{eq:01-MX4s} egyenletbe:
   \al{
    &\rot{\vect{B}(\vect{r})}=\mu_0(\vect{J}(\vect{r})+\rot\vect{M}(\vect{r}))
    &\vect{H}(\vect{r})=\frac{1}{\mu_0}\vect{B}(\vect{r})-\vect{M}(\vect{r})
    & &\rot{\vect{H}(\vect{r})}=\vect{J}(\vect{r})
   }
   
   Itt is szükség van egy anyagi egyenletre ($\vect{B}(\vect{H})$ vagy $\vect{H}(\vect{M})$ vagy $\vect{B}(\vect{M})$), hogy az egyenletrendszer zárt legyen.
   Lineáris anyagoknál a mágnesezettség a mágneses tér lineáris függvénye: $\vect{M}(\vect{r})=\mat{\chi}\vect{H}(\vect{r})$.
   Itt $\mat{\chi}$ a mágneses szuszceptibilitás tenzor.
   Ha az anyag izotrop, akkor ez helyettesíthető egy skalárral: $\vect{M}(\vect{r})=\chi\vect{H}(\vect{r})$.
   Ekkor $\vect{B}(\vect{r})=\mu_0(1+\chi)\vect{H}(\vect{r})=\mu\vect{H}(\vect{r})$. 
   
   \begin{description}
    \item[Diamágneses] anyagoknál $-1\leq\chi<0$, vagyis $\mu<\mu_0$.
   A generálódó mágnesezettség csökkenteni kívánja a külső teret.
   Akkor áll fenn általában, ha az elemi összetevőknek nincsen mágneses momentumuk kezdetben.
    \item[Paramágneses] anyagoknál az elemi összetevők kezdetben mágnesesek, a külső tér ezekre forgatónyomatékkal hat, a saját irányába akarja forgatni: $0<\chi$, vagyis $\mu>\mu_0$. 
    \item[Ferromágneses] anyagoknál az elemi részek közötti mágneses kölcsönhatás erős.
   Külső tér nélkül is van eredő mágnesezettség, energia csökkentésére doménszerkezet alakul ki.
   Nem lineáris: $\vect{B}=\mu(\vect{H})\vect{H}$. 
   \end{description}
   Dia- és paramágneses esetben $\abs{\chi}\sim 10^{-5}\ll 1$, így általában elhanyagolható, de ferromágneses esetben $\chi\sim 10-10^{4}$. 
   
   Különböző anyagi minőségű tartományok határfelületén a mágneses mennyiségekre is vonatkoznak határfeltételek: a $\vect{B}$-nek a normális, a $\vect{H}$-nak pedig a tangenciális komponense halad át változatlanul.
   Ez a vektorpotenciálra azt a megkötést adja, hogy $\vect{A}$ folytonos és $\frac{1}{\mu_1}\rot{\vect{A_1}}=\frac{1}{\mu_2}\rot{\vect{A_2}}$

  \subsection{Időfüggő Maxwell-egyenletek}
   
   Időfüggő folyamatoknál figyelembe vesszük, hogy a töltések elmozdulhatnak.
   A töltés megmaradó mennyiség, így igaz, hogy 
   \al{
    &\ointl{\partial V}{}\df_\rv \Jv(t,\rv)=\der{}{t}\intl{V}{}\drh\rho(t,\rv)
    &\Rightarrow
    &&\partial_t\rho(t,\rv)+\divo\Jv(t,\rv)=0.
   }
   Magnetosztatikában az áramok időben állandóak, így $\partial_t\rho(\rv)=0=\divo\Jv(\rv)$.
   
   A Faraday-törvény kimondja, hogy 
   \eq{
    \oint\limits_{\partial F}\dd\vect{l}\;\vect{E}=-\der{}{t}\intl{F}{}\df\vect{B},
   }
   melyben, ha a felület időben állandó, a deriválás és az integrálás sorrendjét megcserélve, majd a Stokes-tétellel átalakítva:
   \eq{
    \rot{\vect{E}(\vect{r})}=-\partial_t \vect{B}(\vect{r}).
   }
   Ez visszaadja a stacioner esetet is. 
   
   \Eqaref{eq:01-MX4s} egyenlet sem érvényes, hiszen a bal oldal divergenciája nulla, a jobb oldalé pedig: $\mu_0\divo{\vect{J}(\vect{r})}=-\mu_0\partial_t\rho(\vect{r})\neq 0$. Így az egyenletet ki kell egészíteni egy taggal:
   \eq{
    \divo{\left[\rot\vect{B}(\vect{r})-\mu_0\vect{J}(\vect{r})\right]} = \mu_0\partial_t\rho(\vect{r}) = \mu_0\ep_0\partial_t\divo{\vect{E}(\vect{r})}
   }
   \eq{
    \divo{\left[\rot\vect{B}(\vect{r})-\mu_0\vect{J}(\vect{r})-\mu_0\ep_0\partial_t\vect{E}(\vect{r})\right]}=0.
   }
   Innen nem következik egyértelműen, de a sztatikával akkor kapunk egyezést, ha a $\divo$ argumentuma eltűnik. 
   
   Ezzel a négy Maxwell-egyenlet vákuumban:
   \al{
    \divo{\vect{E}(\vect{r})}&=\frac{1}{\ep_0}\rho(\vect{r}) &
    \divo\vect{B}(\vect{r})&=0 \\
    \rot{\vect{E}(\vect{r})}&=-\partial_t\vect{B}(\vect{r}) 
    &\rot{\vect{B}(\vect{r})}&=\mu_0\vect{J}(\vect{r})+\frac{1}{c^2}\partial_t\vect{E}(\vect{r}), 
   }
   ahol $\frac{1}{c^2}=\mu_0\ep_0$.
   
   Anyag jelenlétében a töltéssűrűség $\rho_\text{tot}=\rho+\rho_\text{ind}$ alakú.
   Itt $\rho_\text{ind}=-\divo{\vect{P}}$ (\eqref{eq:01-toltsuru} egyenlet).
   Az indukált töltések viszont az áramsűrűségben is adnak járulékot, így   $\vect{J}_\text{tot}=\vect{J}+\rot{\vect{M}}+\vect{J}_\text{ind}$. $\vect{J}_\text{ind}$-det a kontinuitási egyenlet kapcsolja össze az indukált töltéssűrűséggel: 
   \eq{
    0=\partial_t\rho_\text{ind}+\divo{\vect{J}_\text{ind}}=\divo{\left[\vect{J}_\text{ind}-\partial_t\vect{P}\right]}.
   }
   A divergenciában lévő mennyiséget választhatjuk nullának.
   Ha egy függvény rotációját hozzáadnánk, akkor csak az $\vect{M}$-ben okoznánk változást.
   Tehát:
   \eq{
    \vect{J}_\text{ind}=\partial_t\vect{P},
   }
   amivel a homogén anyagban igaz időfüggő Maxwell egyenletek:
  \\[6pt]
   \fbox{
    \addtolength{\linewidth}{-10\fboxsep}%
    \addtolength{\linewidth}{-5\fboxrule}%
    \begin{minipage}{\linewidth}
     \vspace{-12pt}
     \aln{
      \divo{\vect{D}(\vect{r})}&=\rho(\vect{r}) 
       &\divo\vect{B}(\vect{r})&=0 \label{eq:01-MXanyagban1}\\
       \rot{\vect{E}(\vect{r})}&=-\partial_t\vect{B}(\vect{r}) 
       &\rot{\vect{H}(\vect{r})}&=\vect{J}(\vect{r})+\partial_t\vect{D}(\vect{r}),\label{eq:01-MXanyagban2}
     }
    \end{minipage}
   }
  \\[10pt]
  melyekhez a határfeltételek, hogy $D_n$, $E_t$, $B_n$ és $H_t$ folytonosan mennek át a határfelületeken.

  \subsection{Sztatikus, kvázisztatikus és gyorsan változó terek}\label{ss:01-eldidofugges}
   
   \paragraph{Sztatika}
   
    A megoldandó egyenletrendszer Coulomb-mértékben:
    \al{
     &\Delta\phi(\vect{r})=-\frac{1}{\ep_0}\rho(\vect{r})
     &\Delta\vect{A}(\vect{r})=-\mu_0\vect{J}(\vect{r}),
    }
    Melyhez szükséges az időben állandó töltéseloszlást és áramsűrűséget ismerni.
   A határfeltételek: Dirichlet: a potenciálok a határfelületen adottak, Neumann: a potenciálok felületre merőleges gradiense adott a határfelületen.
   A megoldás egyértelmű. 
  
   \paragraph{Kvázisztatika}
    
    Az időfüggő Maxwell-egyenletekből az $\frac{1}{c^2}$-tel elnyomott tagot elhagyjuk, mert az nagyon kicsi:
    \al{
     \divo{\vect{E}(\vect{r})}&=\frac{1}{\ep_0}\rho(\vect{r}) &
     \divo\vect{B}(\vect{r})&=0 \\
     \rot{\vect{E}(\vect{r})}&=-\partial_t\vect{B}(\vect{r}) 
     &\rot{\vect{B}(\vect{r})}&=\mu_0\vect{J}(\vect{r}). 
    }
    Ezek egy, a vezetési jelenségeket leíró egyenlettel kiegészítve megoldhatóak.
   Fémben, ahol $\vect{J}=\sigma\vect{E}$, a skin-effektust láthatjuk: $\vect{B}$-re és $\vect{E}$-re is ugyanolyan hővezetési egyenleteket kapunk, a megoldás pedig a fém belseje felé exponenciálisan levág. 
    
   \paragraph{Gyorsan változó terek}
    
    Ekkor a teljes időfüggő Maxwell-egyenleteket kell megoldani.
   Forrásmentes esetben Lorentz-mértéket $\left(0=\divo \Av+\frac{1}{c^2}\partial_t\phi\right)$ használva a Maxwell-egyenletek az alábbi két egyenletben foglalhatóak össze:
    \al{
     &&\left(\Delta-\frac{1}{c^2}\partial_t^2\right)\phi=-\frac{1}{\ep_0}\rho &&& \left(\Delta-\frac{1}{c^2}\partial_t^2\right)\vect{A}=-\mu_0\vect{J}. &
    }
    
    Ez két hullámegyenlet ugyanakkora terjedési sebességgel, ezek megoldásai az elektromágneses hullámok.
   Az egyenletek teljes forrásokkal történő megoldása Green-függvényekkel lehetséges. 

 \section{Kvantummechanika}\label{ss:01-kvantum}
  
  \subsection{A kvantummechanika axiómái}
   
   \begin{enumerate}[I.]
    \item Egy fizikai rendszer állapotait egy $\mathbb{H}$ szeparálható Hilbert-tér normált vektoraiként reprezentáljuk: $\ket{\psi}\in\mathbb{H}$. 
    \item A rendszert jellemző fizikai mennyiségeket a Hilbert-tér elemein ható önadjungált operátorokkal reprezentáljuk: $\hat{A}\in\Lin{(\mathbb{H})}$.
    
    Az operátornak a sajátértékegyenlete az alábbi:
    \eq{
     \hat{A}\ket{n}=a_n\ket{n}.
    }
    Önadjungált operátorok sajátfüggvényei ($\{\ket{n}\}_{n=1}^{\infty}$) teljes ortonormált rendszert alkotnak, vagyis $\suml{n=1}{ }\ket{n}\bra{n}=\hat{I}$, és minden $\ket{\psi}\in\mathbb{H}:$ $\ket{\psi}=\suml{n}{} \bra{n}\et{\psi}\ket{n}=\suml{n}{} c_n\ket{n}$, ahol $c_n$ a $\ket{\psi}$ kifejtési együtthatói az $\{\ket{n}\}$ bázisra nézve.
   Az $\{a_n\}$ sajátértékek valós számok az operátor önadjungáltsága miatt.
    
    \item Az $\hat{A}$ operátorhoz tartozó fizikai mennyiség lehetséges értékei csak az $\{a_n\}$ értékek közül kerülhet ki.
   Ha a rendszer $\ket{\psi}$ állapotban van, akkor annak a valószínűsége, hogy méréssel az $a_n$ értéket kapjuk, éppen $W(k)=\abs{c_k}^2$.
    
    Az $\hat{A}$ operátorhoz tartozó fizikai mennyiség várható értéke: $\mv{\hat{A}}=\lim\limits_{N\to\infty}{\suml{n}{}a_n\frac{N_n}{N}}$, ahol $N$ a mérések száma, amivel a végtelenbe tartunk, miközben $N_n$-szer kapuk az $a_n$ értéket.
   Fejtsük ki, hogy ez mit jelent:
    \al{
     \mv{\hat{A}}
     &= \suml{n}{}a_n\lim\limits_{N\to\infty}{\frac{N_n}{N}}
      = \suml{n}{}a_n W(n) 
      = \suml{n}{}a_n \abs{c_n}^2 
      = \suml{n,m}{}a_n c_n^*c_m \delta_{n,m} 
      = \suml{n,m}{}a_n c_n^*c_m \bra{n}\et{m} \\ 
     &= \suml{n,m}{} c_n^*c_m \bra{a_n n}\et{m} 
      = \suml{n,m}{} c_n^*c_m \bra{\hat{A} n}\et{m} 
      = \suml{n}{} c_n^* \bra{\hat{A} n}\suml{m}{}c_m\et{m}
      = \suml{n}{} c_n^* \bra{\hat{A} n}\et{\psi}  \\
     &= \suml{n}{} c_n^* \bra{ n}\et{\hat{A}\psi} 
      = \bra{\suml{n}{} c_n n}\et{\hat{A}\psi} 
      = \bra{\psi}\et{\hat{A}\psi}
      = \bra{\psi}\hat{A}\ket\psi,
    }
    vagyis 
    \eqn{
     \boxed{\mv{A}=\bra{\psi}\hat{A}\ket\psi}.\label{eq:01-meres}
    }
    \item A kvantummechanikai állapot időfejlődését a:
    \eqn{
     \boxed{(i\hbar\partial_t-\hat{H}(t))\ket{\psi(t)}=0}
    }
    egyenlet írja le.

    A hullámfüggvények és a Hamilton-operátor itt az időt mint paramétert tartalmazzák. 
    \item Ha az $\hat{A}$ operátor mérése során $a_k$ értéket kapunk, akkor ezután a rendszer biztosan a $\ket{k}$ állapotban található.
   \end{enumerate}
   
  \subsection{Méréselmélet, sűrűségoperátor}\label{ss:01-mereselmelet}
   
   Egy rendszer tiszta állapotban van, ha létezik $\ket{\psi}=\suml{n}{}c_n\ket{n}\in\mathbb{H}$, hogy a rendszernek ez az állapota.
   Ha a rendszer állapota nem reprezentálható egy Hilbert-tér elemmel, akkor az kevert állapotban van: $p_i$ valószínűséggel van $\ket{\psi_i}$ állapotban, ahol $\suml{i=1}{N}p_i=1$. 
   
   A sűrűségoperátor definíciója tiszta és kevert állapotra:
   \aln{
    &\hat{\rho} = \ket{\psi}\bra{\psi} &\hat{\rho} = \suml{i=1}{N}p_i\ket{\psi_i}\bra{\psi_i}=\suml{i=1}{N}p_i\hat{\rho}_i
   }
   A sűrűségoperátor tulajdonságai: a sűrűségoperátor nyoma tiszta állapotra:
   \eq{
    \tr{\hat{\rho}}=
     \suml{n}{}\bra{n}\hat{\rho}\ket{n} 
     = \suml{n}{}\bra{n}\et{\psi}\bra{\psi}\et{n} 
     = \suml{n}{}c_i^*c_i 
     = 1, 
   }
   illetve kevert állapotban:
   \eq{
    \tr{\hat{\rho}}=
     \suml{n}{}\bra{n}\bigg(\suml{i=1}{N}p_i\ket{\psi_i}\bra{\psi_i}\bigg)\ket{n} 
     = \suml{i=1}{N}p_i\suml{n}{}\bra{n}\et{\psi_i}\bra{\psi_i}\et{n} 
     = \suml{i=1}{N}p_i\suml{n}{}c_n^*c_n 
     = \suml{i=1}{N}p_i
     = 1,
   }
   vagyis mindenképp $\tr{\hat{\rho}}=1$.
   A $\hat\rho$ hermitikus: $\hat\rho^+=\hat\rho$.
   Nézzük a négyzetének a nyomát tiszta:
   \eq{
    \tr{\hat\rho^2} = \tr{\ket{\psi}\underbrace{\bra{\psi}\et{\psi}}_{=1}\bra{\psi}}= \tr{\ket{\psi}\bra{\psi}}=1,
   }
   illetve kevert állapotban:
   \al{
    \tr{\hat\rho^2} 
     &= \tr{\left(\suml{i,j=1}{N}p_ip_j\ket{\psi_i}\bra{\psi_i}\et{\psi_j}\bra{\psi_j}\right)}
      = \tr{\left(\suml{i,j=1}{N}p_ip_j\abs{\bra{\psi_i}\et{\psi_j}}^2\right)} \\
     &= \left(\suml{i,j=1}{N}p_ip_j\abs{\bra{\psi_i}\et{\psi_j}}^2\right)
      < \left(\suml{i,j=1}{N}p_ip_j \cdot 1\right)
      = \left(\suml{i=1}{N}p_i\right) \left(\suml{j=1}{N}p_j\right)
      = 1\cdot 1 = 1.
   }
   Az egyenlőtlenségnél kihasználtuk azt, hogy a rendszer kevert állapotban van, melynek a felbontásában nem szerepelhet két ugyanolyan $\ket{\psi_i}$ állapot. 
   
   A sűrűségmátrix használható a várhatóérték kiszámítására tiszta és kevert állapotban is:
   \al{
    \mv{\hat{A}}_\text{t}&=\bra{\psi}\hat{A}\ket{\psi} = \tr{\bra{\psi}\hat{A}\ket{\psi}} = \tr{\left(\ket{\psi}\bra{\psi}\hat{A}\right)} = \tr{\hat{\rho}\hat{A}} \\
    \mv{\hat{A}}_\text{k}&=\suml{i=1}{N}p_i\bra{\psi_i}\hat{A}\ket{\psi_i}= \dots = \tr{\hat{\rho}\hat{A}}
   }
   
   A rendszernek tiszta állapotban csak látszólagosan van statisztikus jellege a mérés szempontjából.
   Ez a látszólagos statisztikus jelleg a kvantummechanika sajátja, a tiszta állapotok velejárója, tőlünk független, a $c_n$ kifejtési együtthatók által jelenik meg.
   Kevert állapotban viszont a fizikai rendszernek valódi statisztikus jellege van, amelyeket a $p_i$ együtthatók fejeznek ki. 
   
   A tiszta és kevert állapotok egymástól méréssel különböztethetőek meg.
   Ha a mérendő mennyisége $\hat{A}$, és a rendszer bázisa a $\hat B$ operátor sajátfüggvényeiből épül fel, akkor az $\hat A$ mérésével csak akkor lehetséges elkülöníteni a kevert és a tiszta állapotokat, ha $[\hat{A},\hat{B}]\neq 0$.
   
   Ha több részrendszerből épül fel a kvantummechanikai rendszerünk, akkor a rendszer tiszta állapotát $\mathbb{H}_1\otimes\mathbb{H}_2\ni\ket\psi=\suml{n,m}{}c_{n,m}\ket{1,n}\ket{2,m}$ formában írhatjuk fel általánosan, ahol persze $\suml{n,m}{}c_{n,m}=1$.
   Ennek a rendszernek a sűrűségoperátora: $\hat\rho= \ket{\psi}\bra{\psi}$.
   
   Mérjünk a rendszeren egy olyan mennyiséget, amely csak az 1. alrendszerhez csatolódik: $\tr{\hat\rho\hat A_1}=\tr_1{\big(\tr_2{(\hat\rho\hat A_1)}\big)}=\tr_1{\big(\left(\tr_2{\hat\rho}\right)\hat A_1\big)}= \tr_1\big(\hat\rho_1\hat A_1\big)$.
   Itt $\hat\rho_1=\tr_2\hat\rho$ az 1. alrendszerhez tartozó redukált sűrűségoperátor.
   Lehet, hogy $\hat\rho$ még tiszta állapotot jellemzett, de $\hat\rho_1$ már valószínű, hogy nem azt fog.
   Akkor látom az első részrendszer tiszta állapotban, ha a két részrendszer egymástól független, vagyis $\ket{\psi}=\ket{1,\phi_1}\ket{2,\phi_2}$, ekkor pedig $\hat\rho=\hat\rho_1\hat\rho_2$. 
   
  \subsection{Folytonos és diszkrét reprezentációk}
   
   Fontos kérdés az, hogy milyen teljes ortonormált bázist választunk.
   Az első eset a koordináta-reprezentáció.
   A koordinátához tartozó hermitikus operátor $\hat{x}$:
   \eq{
    \hat{x}\ket{x}=x\ket{x}. 
   }
   Innen $\bra{x}\hat{x}\ket{x'}=x\delta (x-x')$.
   Az impulzus operátor mátrixelemei $\bra{x}\hat{p}\ket{x'}=\frac{\hbar}{i}\delta (x-x')\pder{}{x}$-nek kell lenni, hogy teljesüljön a kommutációs reláció:
   \eq{
    \bra{x}[\hat{p},\hat{x}]\ket{x'}=\frac{\hbar}{i}\delta (x-x').
   }
   Egy tetszőleges $\ket{\psi}$ vektor kifejthető az $\ket{x}$ báziselemek segítségével:
   \eq{
    \ket{\psi}=\int\dd x\; \ket{x}\bra{x}\et{\psi}=\int\dd x \;\psi(x)\ket{x}.
   }
   Itt $\psi(x)$ a $\ket\psi$ vektor koordináta-reprezentált alakja, egy sima függvény, amelyre persze teljesül, hogy $\int \dd x \abs{\psi(x)}^2=1$.
   
   A Hamilton-operátor hatását egy időfüggetlen rendszer kvantummechanikai állapotára is felírhatjuk koordináta-reprezentációban:
   \al{
    0&=\big(\hat{H}(\hat{p},\hat{x},\dots)-E\big)\ket{\psi} 
      =\bra{x}\big(\hat{H}(\hat{p},\hat{x},\dots)-E\big)\ket{\psi} \\
     &=\bra{x}(\hat{H}\big(\hat{p},\hat{x},\dots)-E\big)\left(\int \dd x'\ket{x'}\bra{x'}\right)\ket{\psi} \\
     &=\int \dd x'\bra{x}\big(\hat{H}(\hat{p},\hat{x},\dots)-E\big)\ket{x'}\bra{x'}\et{\psi} \\
     &=\int \dd x'\left[H\left(\frac{\hbar}{i}\pder{}{x},x,\dots\right)-E\right]\delta(x-x')\psi(x') \\
     &= \left[H\left(\frac{\hbar}{i}\pder{}{x},x,\dots\right)-E\right]\psi(x),
   }
   ami az időfüggetlen Schrödinger-egyenlet. 
   
   Az impulzus reprezentáció teljesen hasonló: itt a $\hat{p}\ket{p}=p\ket{p}$ teljes ortonormált bázist használjuk. $\hat{p}$ impulzus-reprezentált alakja $p$, $\hat{x}$-é $-\frac{\hbar}{i}\pder{}{p}$.
   A kommutációs relációk reprezentációfüggetlenek. 
   
   A diszkrét reprezentációk esetében olyan operátor sajátrendszerét választjuk, melynek spektruma diszkrét: $\{\ket{n}\}_{n=1}^{\infty}$.
   Minden hasonlít a fentiekhez.
   Egyedül annyival másabb ez a reprezentáció, hogy az operátorok felírhatók végtelen mátrix alakban. 
   
   {\color{red} Esetleg valamit lehetne a repik között áttérésről.}
   
  \subsection{Kvantummechanikai képek}
   
   \paragraph{Schrödinger-kép}
    
    A Schrödinger-képben az időfüggést a hullámfüggvények hordozzák: $\ket{\psi}=\ket{\psi(t)}$, $\hat{A}=\hat{A}$.
   A hullámfüggvények időfüggését a Schrödinger-egyenlet írja le:
    \eq{
     i\hbar\partial_t\ket{\psi_S(t)}=\hat{H}_S(t)\ket{\psi_S(t)}.
    }
    A Schrödinger-egyenletet formálisan integrálva:
    \eq{
     \ket{\psi_S(t)}=\ket{\psi_S(0)}-\frac{i}{\hbar}\intl{0}{t}\dd t'\hat{H}_S(t')\ket{\psi_S(t')},
    }
    melynek megoldása:
    \al{
     \ket{\psi_S(t)}
     &= \left[ 1-\frac{i}{\hbar}\intl{0}{t}\dd t'\hat{H}_S(t')+\left(-\frac{i}{\hbar}\right)^2\intl{0}{t}\dd t'\intl{0}{t'}\dd t''\hat{H}_S(t')\hat{H}_S(t'')`+\dots \right] \ket{\psi_S(0)} \\
     &= \left[ \suml{n=0}{\infty}\frac{1}{n!}\left(-\frac{i}{\hbar}\right)^n\intl{0}{1}\dotsi\intl{0}{t}\dd\tau_1\dots\dd\tau_n \T\big(\hat{H}_S(\tau_1)\cdots \hat{H}_S(\tau_n)\big)\right] \ket{\psi_S(0)} \\
     &= \left[ \T e^{-\frac{i}{\hbar}\intl{0}{t}\dd \tau \hat{H}_S(\tau)}\right] \ket{\psi_S(0)},
    }
    ahol $\T$ az időrendező operátor.
   Az időfejlesztés az $\hat{R}(t,t')=\T e^{-\frac{i}{\hbar}\intl{t'}{t}\dd \tau \opH_S(\tau)}$ rezolvens végzi:
    \eq{
     \ket{\psi(t)}=R(t,t')\ket{\psi(t')}.
    }
    
   \paragraph{Heisenberg-kép}
   
    A Heisenberg-képben az időfejlődést az operátorokra hárítjuk.
   Legyen $t_0=0$ kezdeti időpillanat.
   Az operátorok időfejlődését az alapján lehet meghatározni, hogy a kvantummechanikai várható értékeknek képektől függetleneknek kell lennie:
    \eq{
     \mv{\hat{A}(t)}
      = \bra{\psi_S(t)}\hat{A}_S\ket{\psi_S(t)}=\bra{\psi_S(0)}\hat{R}^+(t)\hat{A}_S\hat{R}(t)\ket{\psi_S(0)}=\bra{\psi_H}\hat{A}_H(t)\ket{\psi_H},
    }
    így tehát
    \aln{
     &\ket{\psi_H}= \ket{\psi_S(0)}, &A_H(t)=\hat{R}^+(t)\hat{A}_S\hat{R}(t).
    }
    
    Az operátorok időfejlődéséhez kiszámoljuk az operátor idő szerinti deriváltját.
   Figyelem, $\opH_S$ és $\opR$ általában nem cserélhető fel az időrendezés miatt!
    \eq{
     \partial_t \hat{A}_H(t)=\partial_t\big(\hat{R}^+(t)\hat{A}_S\hat{R}(t)\big) 
      = \partial_t\hat{R}^+(t)\hat{A}_S\hat{R}(t)+\hat{R}^+(t)\hat{A}_S\partial_t\hat{R}(t).
    }
    Itt az első tagot átalakítjuk, a másodikat is ugyanúgy lehet:
    \al{
     \partial_t\hat{R}^+(t)\hat{A}_S\hat{R}(t)
      &= \frac{i}{\hbar}\hat{R}^+(t)\hat{H}_S(t)\hat{A}_S\hat{R}(t) 
       = \frac{i}{\hbar}\hat{R}^+(t)\hat{H}_S(t)\underbrace{\hat{R}(t)\hat{R}^+(t)}_{=1}\hat{A}_S\hat{R}(t) \\
      &= \frac{i}{\hbar}\underbrace{\hat{R}^+(t)\hat{H}_S(t)\hat{R}(t)}_{=\hat{H}_H(t)}\underbrace{\hat{R}^+(t)\hat{A}_S\hat{R}(t)}_{=\hat{A}_H(t)} 
       = \frac{i}{\hbar}\hat{H}_H(t)\hat{A}_H(t),
    }
    vagyis:
    \eq{
     \boxed{\partial_t \hat{A}_H(t)=\frac{i}{\hbar}\big[\hat{H}_H(t),\hat{A}_H(t)\big]}.
    }
   
   \paragraph{Kölcsönhatási, Dirac-kép}\label{ss:A01-dirac}
    A kölcsönhatási vagy Dirac-képben a hullámfüggvények és az operátorok is hordoznak időfüggést.
   Schrödinger-képben a rendszer Hamilton-operátora: $\hat{H}_S(t)=\hat{H}_S^0(t)+\hat{V}_S(t)$. 
   
    Az operátorok a kölcsönhatásmentes Hamilton-operátor szerint fejlődnek:
    \al{
     &\hat{R}_0(t,t')=\T e^{-\frac{i}{\hbar}\intl{t'}{t}\dd \tau  \hat{H}_S^0(\tau)} &\hat{A}_i(t)=\hat{R}_0^+(t)\hat{A}_S\hat{R}_0(t).
    }
    A várható értékek állandósága miatt a hullámfüggvények a
    \eq{
     \ket{\psi_i(t)}=\hat{R}_0^+(t)\ket{\psi_S(t)}
    }
    alakúak.
   A hullámfüggvények időfejlődése kölcsönhatási képben:
    \al{
     \partial_t\ket{\psi_i(t)}
      &= \partial_t\hat{R}_0^+(t)\ket{\psi_S(t)}+\hat{R}_0^+(t)\partial_t\ket{\psi_S(t)} \\
      &= \frac{i}{\hbar}\hat{R}_0^+(t)\hat{H}_S(t)\ket{\psi_S(t)}-\frac{i}{\hbar}\hat{R}_0^+(t)\big(\hat{H}_S^0(t)+\hat{V}_S(t)\big)\ket{\psi_S(t)} \\
      &= -\frac{i}{\hbar}\hat{R}_0^+(t)\hat{V}_S(t)\ket{\psi_S(t)} \\
      &= -\frac{i}{\hbar}\hat{R}_0^+(t)\hat{V}_S(t)\hat{R}_0(t)\hat{R}^+_0(t)\ket{\psi_S(t)} \\
      &= -\frac{i}{\hbar}\hat{V}_i(t)\ket{\psi_i(t)},
    }
    vagyis a hullámfüggvényeket az 
    \eq{
     \boxed{S(t,t')=\T e^{-\frac{i}{\hbar}\intl{t'}{t}\dd \tau \op{V}_i(\tau)}}
    }
    rezolvens fejleszti. 
   
  \subsection{Korrespondencia-elv}
   
   A kvantummechanikai effektusok a klasszikus energia és méretskálán nem láthatóak.
   A kvantumelmélet akkor lehet csak helyes, ha visszaadja a nagyenergiás határesetet.
   Ez a $\hbar\to 0$, ha diszkrét az energiaspektrum akkor az $n\to \infty$ esetnek felel meg. 
