\chapter{Sokas\'agok, eloszl\'asok, potenci\'alok}
 
 \section{Mikrokanonikus sokaság}
  
  Lásd \aref{ss:B03-mikrokansok}. fejezetet.
  
 \section{Kanonikus sokaság ($T$,$V$,$N$)}
  
  Kanonikus sokaságban egy zárt rendszer kis alrendszerét tekintjük.
   Az egész rendszer energiája rögzített ($E_0=E+E_\text{K}$), ahol $E$ az alrendszer energiája, míg $E_\text{K}$ a környezeté.
   Mivel a környezet jóval nagyobb, mint az alrendszer, ezért $E\approx E_\text{K}=\text{áll}$. 
  
  \subsection{A kanonikus eloszlás}
   
   Az eloszlás meghatározásához kiszámoljuk annak a valószínűségét, hogy az alrendszer az $i$-edik mikroállapotban van.
   Ez a szám azzal lesz egyenlő ahányféleképpen a környezet energiája $E_0-E_i$ lehet:
   \al{
    \rho(i)
     =\frac{\Omega_\text{K}(E_0-E_i,\delta E)}{\Omega_0(E_0,\delta E)}.
   }
   Mivel $E_i$ kicsi, így ezt sorbafejthetjük (a logaritmusát):
   \al{
    \ln\rho(i)
     &=\underbrace{\ln\frac{\Omega_\text{K}(E_0,\delta E)}{\Omega_0(E_0,\delta E)}}_{\text{const.}}-\pder{\ln\Omega_\text{K}(E_0,\delta E)}{E}\cdot E_i+\mathcal{O}(E_i^2)\\
     &=\underbrace{\ln\frac{\Omega_\text{K}(E_0,\delta E)}{\Omega_0(E_0,\delta E)}}_{\text{const.}}-\underbrace{\pder{\ln\Omega_\text{K}(E_\text{K},\delta E)}{E}}_{=\frac{1}{\kB T_\text{K}}}\cdot E_i+\mathcal{O}(E_i^2)
      \approx\text{const.}-\beta_\text{K}E_i.
   }
   Igy $\rho(i)\sim e^{-\beta_\text{K} E_i}$.
   A normálással együtt klasszikus és kvantumos rendszerre:
   \al{
    &\text{Klasszikus rendszerre:}
    &\rho(q,p)&=\frac{1}{Z}e^{-\beta E(q,p)}
    &Z&=\intl{}{}\frac{\dd q^{dN}\dd p^{dN}}{h^{dN}N!}\,e^{-\beta E(q,p)}\\
    &\text{Kvantumos rendszerre:}
    &\op\rho&=\frac{1}{Z}e^{-\beta \opH}
    &Z&=\tr\left(e^{-\beta \opH}\right),
   }
   ahol megelőlegeztük, hogy a környezet és a rendszer hőmérséklete egyenlő.
   Ehhez vizsgáljuk meg, hogy mekkora annak a valószínűségi sűrűsége, hogy az általunk vizsgált alrendszer energiája $E$.
   Ez abból adódik, hogy a rendszer mekkora valószínűséggel van az $i$-edik mikroállapotban, illetve hogy az állapotok milyen sűrűn vannak ($\omega(E)$ az energia-állapotsűrűség): 
   \al{
    P(E)=\rho(E)\omega(E)=\frac{e^{-\beta_\text{K}E}\omega(E)}{Z}.
   }
   Ez egy éles eloszlás, hiszen a számláló első tagja az energia növelésével élesen levág, míg $\omega(E)$ az energia növelésével élesen nő.
   Az éles eloszlás miatt a várható érték és a legvalószínűbb érték itt is jó közelítéssel megegyezik.
   Az eloszlás (logaritmusának) maximuma:
   \al{
    \ln P(E)&=-\beta_\text{K}E+\ln\omega(E)+\text{const.}\\
    0&=\pder{\ln P(E)}{E}=-\beta_\text{K}+\underbrace{\pder{\ln \omega(E)}{E}}_{\frac{1}{\kB T}}
    &\Rightarrow
    &&T=T_\text{K}.
   }
  
  \subsection{Energia várható értéke, fluktuációi}
   
   Az energia várható értéke:
   \al{
    \mv{E}
     &=\suml{i}{}E_i\rho(i)
      =\suml{i}{}\frac{E_i e^{-\beta E_i}}{\suml{j}{}e^{-\beta E_j}}
      =-\pder{\ln Z}{\beta},
   }
   illetve szórásnégyzete:
   \al{
    \mv{(\Delta E)^2}
     &=\mv{E^2}-\mv{E}^2
      =\frac{1}{Z}\pder{^2 Z}{\beta^2}-\left(\frac{1}{Z}\pder{Z}{\beta}\right)^2
      =\pder{}{\beta}\left(\frac{1}{Z}\pder{Z}{\beta}\right)
      =\pder{}{\beta}\left(\pder{\ln Z}{\beta}\right)\\
     &=-\pder{E}{\beta}
      =-\kB\underbrace{\pder{E}{T}}_{C_V}\underbrace{\pder{T}{\frac{1}{T}}}_{-T^2}
      =\kB T^2 C_V.
   }
   Az energia szórásnégyzete egy statisztikus fizikai mennyiség, míg a jobb oldalon állók termodinamikai mennyiségek.
   Mivel feltételeztük, hogy normál rendszerben vagyunk, így $C_V>0$ adódik.
   A termodinamikában ez stabilitási feltétel.
   
  \subsection{Szabadenergia}
   
   A termodinamikában az energiából Legendre-transzformációval előállítjuk a szabadenergiát: $F=E-TS$.
   A statisztikus fizikában:
   \al{
    F=-\kB T\ln Z,
   }
   ahol $Z$ az állapotszám.
   Mivel az eloszlás nagyon éles, ezért az állapotösszeg közelíthető:
   \aln{
    Z=\intl{E_\text{min}}{\infty}\dd E\,e^{-\beta E}\omega(E)
     =e^{-\beta E} \omega(E)\cdot \Delta E,\label{eq:B04-kanallszam}
   }
   ahol $\Delta E$ az eloszlás szélessége.
   A szórásnégyzetből adódik, hogy $\Delta E\sim C_V\sim N$. Így tehát
   \al{
    F=-\kB T\ln Z
     =-\kB T \big(\underbrace{-\beta E}_{\mathcal{O}(N)} +\underbrace{\ln\omega(E)}_{\mathcal{O}(N)}+ \underbrace{\ln\Delta E}_{\mathcal{O}(\ln N)}\big)
     \approx -\kB T (-\beta E + \ln\omega(E))
     =E-TS,
   }
   vagyis makroszkopikusan megegyezik a termodinamikai definícióval.
   A termodinamikai deriváltaknak megfelelő összefüggéseket is megkapjuk:
   \al{
    \pder{F}{T}
     &=\pder{}{T}(-\kB T \ln Z)
      =-\kB\ln Z-\kB T\pder{\ln Z}{T}
      =\frac{F}{T}+\frac{1}{T}\pder{\ln Z}{\beta}
      =\frac{F}{T}-\frac{E}{T}
      =-S\\
    \pder{F}{V}
     &=\pder{}{V}(-\kB T \ln Z)
      =-\kB T \pder{\ln Z}{V}
      =-\kB T \frac{1}{Z}\suml{i}{}\beta\left(-\pder{E_i}{V}\right)e^{-\beta E_i}
      =\frac{1}{Z}\suml{i}{}\pder{E_i}{V}e^{-\beta E_i}\\
     &=\mv{\pder{E}{V}}=-p
   }
   
   $F$ extenzív mennyiség, hiszen független alrendszerek összességére $Z$ szorzódik.
   Az információs entrópiába helyettesítve a kanonikus eloszlás az $S$ entrópiát adja:
   \al{
    \kB S_\text{inf}
     &=-\kB\suml{}{}\rho(i)\ln\rho(i)
      =-\kB\frac{1}{Z}\suml{}{}e^{-\beta E_i}\ln\left(\frac{1}{Z}e^{-\beta E_i}\right)\\
     &=-\kB\frac{1}{Z}\suml{}{}e^{-\beta E_i}\left(-\beta E_i-\ln Z \right)
      =\kB\beta\underbrace{\frac{1}{Z}\suml{}{} E_i e^{-\beta E_i}}_{E}
       +\kB\ln Z\underbrace{\frac{1}{Z}\suml{}{}e^{-\beta E_i}}_{1}\\
     &=\frac{E}{T}-\frac{F}{T}
      =S.
   }
   Az információs entrópiát a mikrokanonikus eloszlás maximalizálta.
   A kanonikus eloszlás az információs entrópiát az a mellékfeltétel mellett maximalizálja, hogy $E$ állandó.
   Ezt onnan tudjuk, hogy $T$-t rögzíti a környezet, és $T$ az átlagenergia, így fix részecskeszám mellett $E=\suml{}{}E_i\rho(i)=\text{const.}$ A variálás:
   \al{
    0&=\delta\left(-\suml{i}{}\rho(i)\ln\rho(i)-\lambda\suml{i}{}E_i\rho(i)\right)
      =-\suml{i}{}\left(\delta\rho(i)\ln\rho(i)+\rho(i)\frac{1}{\rho(i)}\delta\rho(i)\right)-\lambda\suml{i}{}E_i\delta\rho(i)\\
     &=\suml{i}{}\delta\rho(i)\big(-\ln\rho(i)-1-\lambda E_i\big)\\
    0&=-\ln\rho(i)-1-\lambda E_i\\
    \rho(i)&=C e^{-\lambda E_i},
   }
   ahol $C=Z$ és $\lambda=\beta$. 
   
  \section{Nagykanonikus sokaság ($T$,$V$,$\mu$)}
   
   A konstrukció hasonló az előzőhöz: most is egy elszigetelt nagy rendszer pici alrendszerét vizsgáljuk.
   Az alrendszer és a környezet között anyagi kölcsönhatás és energiatranszfer történhet, de mechanikai munka nincs ($\delta V=0$). 
   
   Bevezetjük a nagykanonikus potenciált:
   \al{
    \Phi(T,V,N)&=E-TS-\mu N=-pV
    &\dd\Phi=-S\dd T-p\dd V-N\dd\mu
   }
   A termodinamikai deriváltak:
   \al{
    &\left(\pder{\Phi}{T}\right)_{V,\mu}=-S
    &&\left(\pder{\Phi}{V}\right)_{T,\mu}=-p
    &&\left(\pder{\Phi}{\mu}\right)_{T,V}=-N.
   }
   Ebben az esetben egy mikroállapot valószínűsége: $\rho(N,i_N)$.
   A mikroállapotok valószínűsége attól is függ, hogy hány részecske van épp a rendszerben.
   Ez felírva a környezet és a teljes rendszer állapotszámaival:
   \al{
    \rho(N,i_N)=\frac{\Omega_{0,\text{K}}(E_0-E,N_0-N)}{\Omega_0(E_0,N_0)}.
   }
   Itt is sorfejtést alkalmazunk a kis $E$ és $N$-re:
   \al{
    \ln\rho(N,i_N)=\text{const}-\pder{\ln\Omega_\text{K}(E_0,N_0)}{E}E_i-\pder{\ln\Omega_\text{K}(E_0,N_0)}{N}N,
   }
   ahonnan exponencializálva és elnevezve az együtthatókat:
   \al{
    &\rho(N,i_N)=\frac{1}{\mathcal{Z}}e^{-\beta_\text{K}(E_i-\mu_\text{K} N)}
    &\mathcal{Z}=\mathcal{Z}(T,V,\mu)=\suml{N=0}{\infty}\suml{i_N}{}e^{-\beta_\text{K}(E_i-\mu_\text{K} N)}
   }
   Itt is a környezet kémiai potenciálja és redukált hőmérséklete szerepel.
   A kanonikus esethez hasonlóan itt is be lehet látni, hogy a $\beta_\text{K}=\beta$ és $\mu_\text{K}=\mu$.
   
   A nagykanonikus potenciál statisztikus fizikai definíciója:
   \al{
    \Phi=-\kB T\ln\mathcal{Z},
   }
   melyről be lehet látni, hogy az megegyezik a termodinamikai definícióval:
   \al{
    \Phi
     &=-\kB T\ln\mathcal{Z}
      =-\kB T\ln\left(\suml{i}{}e^{-\beta(E_i-\mu N)}\right)
      =-\kB T\ln\left(\suml{N=0}{\infty}e^{\beta\mu N} Z_n\right),
   }
   ahol $Z_N$ a kanonikus állapotszám $N$ részecskére.
   Használva \eqaref{eq:B04-kanallszam} egyenletet, illetve hogy az eloszlás $N$ szerint is éles
   \al{
    \Phi
     &=-\kB T\ln\left(\suml{N=0}{\infty}e^{\beta\mu N} e^{-\beta E} \omega(E)\cdot \Delta E\right)
      =-\kB T\ln\left(e^{\beta\mu N} e^{-\beta E} \omega(E)\cdot \Delta E\Delta N\right)\\
     &=-\kB T\left(\beta\mu N -\beta E +\ln\omega(E)+\ln\Delta E+\ln\Delta N\right)
      \approx -\mu N + E -\kB T\ln\omega(E)\\
     &=E-TS-\mu N.
   }
   
 \section{TPN sokaság ($T$,$p$,$N$)}
  
  Tekintsünk egy olyan alrendszer, amely mechanikai kapcsolatban van a környezettel és hőátadás is lehetséges.
   Ennek a termodinamikai potenciálja a szabadentalpia:
  \al{
   &G(T,p,N)=E-TS+pV=\mu N
   &\dd G=-S\dd T+V\dd p+\mu\dd N.
  }
  
  A mikroállapotok valószínűsége:
  \al{
   &\rho(V,i_V)=\frac{1}{Y}e^{-\beta(E_{i_V}+pV)}
   &Y=Y(T,p,N)=\intl{V}{}\drh\suml{i_V}{}e^{-\beta(E_{i_V}+pV)}.
  }
  
  A statisztikus fizikai potenciál definíciója:
  \al{
   G=-\kB T\ln Y,
  }
  amiről szintén meg lehet mutatni, hogy a termodinamikai határesetben megegyezik a szabadentalpiával.
   A termodinamikai deriváltak:
  \al{
   \pder{G}{T}
    &=-\kB\ln Y-\kB T\pder{\ln Y}{T}
     =\frac{G}{T}+\frac{1}{T}\pder{\ln Y}{\beta}
     =\frac{G}{T}-\frac{E+pV}{T}
     =-S\\
   \pder{G}{p}
    &=-\kB T\pder{\ln Y}{p}
     =\kB T\beta V
     =V\\
   \mv{(\Delta E)^2}
    &=\mv{E^2}-\mv{E}^2
     =\frac{1}{Y}\pder{^2 Y}{\beta^2}-\left(\frac{1}{Y}\pder{Y}{\beta}\right)^2
     =\pder{}{\beta}\left(\frac{1}{Y}\pder{Y}{\beta}\right)
     =\pder{}{\beta}\left(\pder{\ln Y}{\beta}\right)\\
    &=-\pder{E}{\beta}
     =\kB T^2\pder{E}{T}
     =\kB T^2 C_V\\
   \mv{(\Delta V)^2}
    &=\mv{V^2}-\mv{V}^2
     =\frac{1}{Y}\frac{1}{\beta^2}\pder{^2 Y}{p^2}-\left(-\frac{1}{Y}\frac{1}{\beta}\pder{Y}{p}\right)^2
     =\frac{1}{\beta^2}\pder{}{p}\left(\frac{1}{Y}\pder{Y}{p}\right) 
     =\frac{1}{\beta^2}\pder{}{p}\left(\pder{\ln Y}{p}\right) \\
    &=-\kB T\pder{V}{p}
     =\kB T V \kappa_T.
  }
