\chapter{A relativisztikus fizika alapegyenletei}\label{2tetel}
 
 \section{Mechanika és elektrodinamika}
  
  \subsection{A relativitás elve}
   
   A Newton-törvények inerciarendszerekben igazak.
   Egy vonatkoztatási rendszer akkor inerciarendszer, ha bármely test, amely nem hat kölcsön semmilyen más objektummal, állandó impulzussal rendelkezik.
   A fizika relativitásának elve azt mondja ki, hogy bármely ilyen rendszerben a fizikai folyamatokat leíró egyenletek ugyanolyan alakúak.
   Ez azt jeleni, hogy a megfelelő törvényekben szereplő mennyiségekre megadható egy egyértelmű transzformáció, hogy az egyenlet továbbra is igaz maradjon.
   Legyen általánosan egy összefüggés $\Psi(a,b,c,\dots)=0$, mely a transzformált rendszerben a $\Psi(\mathcal{T}_{a}a,\mathcal{T}_{c}c,\mathcal{T}_{c}c,\dots)$ alakú, $\Psi$ nem változik.
   
   A klasszikus mechanikában ez a transzformáció a Galilei-transzformáció.
   A transzformáció csak a helykoordinátákra hat: ha a $\bar{K}$ rendszer $\vect{v}_0$ állandó sebességgel mozog a $K$ rendszerhez képest, akkor a $\bar{\vect{r}}=\vect{r}-\vect{v}t$. 
  
   Egy a probléma: a mozgó töltések által létrehozott mezők (Lienard--Wiechert-potenciálok) nem invariánsak a Galilei-transzformációra, azokat pedig közvetlenül a Maxwell-egyenletekből vezettük le, vagyis azok sem invariánsak.
   Az ellentmondás kétféleképpen oldható fel: a) az elektrodinamika csak egy kitüntetett koordináta-rendszerben érvényes, ez pedig az éter, b) nem helyes a Galilei-transzformáció, mozgó koordináta-rendszerek között más szabály szerint kell áttérni. 
  
   Michelson--Morley megmutatta, hogy ha van éter, akkor az hozzánk képest áll.
   Az éter hipotézis elbonyolódik, és Einstein megmutatta, hogy szükségtelenné válik.
   Ehelyett a b) utat járjuk, megkeressük az új áttérést, a Lorentz-transzformációt.
      
  \subsection{Az elektrodinamika kovariáns formalizmusa}
   
   Áttérünk a négydimenziós Minkowski-térre.
   Egy ebbe tartozó vektor $\minv{x}^\mu=(ct,\vect{r})$, ahol $\mu=0,1,2,3$.
   A deriválást a $\partial_\mu=\left(\frac{1}{c}\pder{}{t}, \pder{}{x^i}\right)$ kovariáns vektor reprezentálja.
   Integrálás $\intl{}{}\dd^4\minv{x}=\intl{}{}\dd(ct)\intl{}{}\drh$ szerint.
   A tér metrikáját a 
   \eq{
   g_{\mu\nu}=(g^{-1})^{\mu\nu}=\begin{pmatrix}
                                 1 & 0 & 0 & 0 \\
                                 0 & -1 & 0 & 0 \\
                                 0 & 0 & -1 & 0 \\
                                 0 & 0 & 0 & -1
                                \end{pmatrix}
   }
   metrikus tenzorral írjuk le.
   Ezzel: $a_\mu=g_{\mu\nu}a^\nu$, $a^\mu=g^{\mu\nu}a_\nu$ és $a\cdot b=a^\mu g_{\mu\nu}b^\nu$. 
   
   Egy vektor hossza, $\minv x^2=c^2t^2-\vect{r}^2$, lehet nulla akkor is, ha $\vect{r}^2\neq 0$.
   Ezek a fényszerű vektorok.
   Az origóból fénysebességnél lassabban ($\vect{r}^2=v^2t^2$) elérhető vektorok időszerűek: $x^2=(c^2-v^2)t^2>0$, ha pedig $x^2<0$, akkor térszerűek. 
   
   Egy ponttöltés töltéssűrűségéből és áramsűrűségéből is képezhetünk egy négyesvektort:
   \eq{
    \minv{J}^\mu=(c\rho,\vect{J}).
   }
   Ekkor a kontinuitási egyenlet egyszerű alakú:
   \eq{
    0=\partial_t\rho+\divo{\vect{J}}=0=\frac{1}{c}\partial_t(c\rho)+\partial_i J_i=\partial_\mu\minv{J}^\mu
   }
   
   A statikus pontöltés potenciálja $\phi(\vect{r})=\frac{\mu_0qc^2}{4\pi r}$ és a statikus áram által létrehozott vektorpotenciál $\vect{A}=\frac{\mu_0q\vect{v}}{4\pi r}$ nagyon hasonló.
   Ebből az intuícióból felírható a négyes potenciál:
   \eq{
    \minv{A}^\mu= \left(\frac 1c\phi,\vect{A}\right).
   }
   
   Bevezetjük a térerősség tenzort:
   \eq{
    \minv{F}^{\mu\nu}=\partial^\mu \minv{A}^\nu-\partial^\nu \minv{A}^\mu,
   }
   Ami antiszimmetrikus ($\minv{F}^{\mu\nu}=-\minv{F}^{\nu\mu}$).
   Ennek elemei:
   \eq{
    \minv{F}^{0i}
    =\partial^0 \minv{A}^i-\partial^i \minv{A}^0
    =\partial_0 \minv{A}^i+\partial_i \minv{A}^0
    =\frac{1}{c}\partial_t A_i + \grad{\left(\frac{1}{c}\phi\right)}=-\frac{1}{c} E_i
   }
   \eq{
    \ep_{ijk}\minv{F}^{ij}
    =\ep_{ijk}\left(\partial^i \minv{A}^j-\partial^j \minv{A}^i\right)
    =2\ep_{ijk}\partial^i \minv{A}^j=-2\ep_{ijk}\partial_i \minv{A}^j=-2(\rot{\vect{A}})_k,
   }
   így a teljes tenzor kétszeresen kontravariánsan:
   \eq{
    \minv{F}^{\mu\nu} = 
    \begin{pmatrix}
     0 & -E_1/c & -E_2/c & -E_3/c \\
     E_1/c & 0 & -B_3 & B_2 \\
     E_2/c & B_3 & 0 & -B_1 \\
     E_3/c & -B_2 & B_1 & 0
    \end{pmatrix}
   }
   Ennek deriváltja azonos a második és a harmadik Maxwell-egyenlettel:
   \al{
    \partial_\mu \minv{F}^{\mu0}&=\partial_i\minv{F}^{i0}=\frac{1}{c}\partial_i E_i=\frac{1}{c}\frac{\rho}{\ep_0}=\mu_0\minv{J}^0 \\
    \partial_\mu \minv{F}^{\mu i}&=\partial_0F^{0i}+\partial_j\minv{F}^{ji}=-\frac{1}{c}\partial_t\frac{1}{c}E_i+\ep_{ijk}\partial_jB_k=\mu_0\minv{J}^i,
   }
   ezek pedig összefoglalva:
   \eqn{
    \boxed{\partial_\mu \minv{F}^{\mu\nu}=\mu_0\minv{J}^{\nu}}.\label{eq:02-MX34}
   }
   
   A másik két Maxwell-egyenlet ebben a formalizmusban azonosság.
   Onnan következnek, hogy az $F$ tenzort már a négyespotenciál létezését feltételezve írtuk fel. 
   
   Ez a két Maxwell-egyenlet kontravariáns alakban történő felírásához bevezetjük a duális térerősség tenzort:
   \eq{
    \tilde{\minv{F}}^{\mu\nu}=\frac 12 \ep^{\mu\nu\rho\sigma}\minv{F}_{\rho\sigma},
   }
   melynek elemei kiírva:
   \eq{
    \tilde{\minv{F}}^{\mu\nu}=
    \begin{pmatrix}
     0 & -B_1 & -B_2 & -B_3 \\
      B_1 & 0 & E_3/c & -E_2/c \\
      B_2 & -E_3/c & 0 & E_1/c \\
      B_3 & E_2/c & -E_1/c & 0
    \end{pmatrix}
   }
   Ennek a négyesderiváltja automatikusan nulla (szimmetrikus--antiszimmetrikus tenzor szorzata).
   Ezek kiírva komponensenként:
   \eq{
    \partial_\mu\tilde{\minv{F}}^{\mu 0}=\frac{1}{2}\partial_i \ep^{i0jk}\minv{F}_{jk}=\frac{1}{2}\partial_i(-\ep_{ijk})(-\ep_{jkl})B_l=\partial_i B_i=0
   }
   \al{
    \partial_\mu\tilde{\minv{F}}^{\mu i}&=\partial_0\tilde{\minv{F}}^{0 i}+\partial_j\tilde{\minv{F}}^{ji}
    =\frac{1}{2}\partial_0 \ep^{0ijk}\minv{F}_{jk}+\frac{1}{2}\partial_j \left(\ep^{ji0k}\minv{F}_{0k}+\ep^{jik0}\minv{F}_{k0}\right) \\
    &=\frac{1}{2c}\partial_t\ep_{ijk}(-\ep_{jkl})B_l+\frac 1c \partial_j\ep_{jik}E_k=-\frac{1}{c}\left[\partial_t B_i+(\rot{\vect{E}})_i\right]=0,
   }
   vagyis megkapjuk az első és a negyedik Maxwell-egyenletet. 
   
  \subsection{A Lorentz-transzformáció}
   
   Keressünk egy olyan transzformációt, amely rendelkezik az alábbi tulajdonságokkal:
   \begin{enumerate}
    \item Lineáris: $\Lambda\colon \bar{\minv{a}}^\mu=\Lambda^\mu_{\phantom{\mu}\nu} \minv{a}^\nu$.
    \item A skaláris szorzatot invariánsan hagyja: $\forall\; \minv{a},\minv{b}$: $\minv{a}\cdot \minv{b}=\bar{\minv{a}}\cdot\bar{\minv{b}}$ $\Leftrightarrow$ $\minv{a}^\mu \minv{b}^\nu g_{\mu\nu}=\bar{\minv{a}}^\mu \bar{\minv{b}}^\nu g_{\mu\nu}$, vagyis
    \al{
     &\Lambda^\mu_{\phantom{\mu}\mu'} \minv{a}^{\mu'}\Lambda^\nu_{\phantom{\nu}\nu'} \minv{b}^{\nu'}g_{\mu\nu}=\minv{a}^{\mu'} \minv{b}^{\nu'} g_{\mu'\nu'}
     &\Rightarrow &&
     & \Lambda^\mu_{\phantom{\mu}\mu'} \Lambda^\nu_{\phantom{\nu}\nu'}g_{\mu\nu}=g_{\mu'\nu'}
    }
   \end{enumerate}
   Ezek a Lorentz-transzformációk.
   Mivel a skaláris szorzatot invariánsan hagyják, ezért az ilyen alakú összefüggések is, vagyis a Maxwell-egyenletek mind invariánsak lesznek erre tekintve. 
   
   A Lorentz-transzformáció két fontos tulajdonsága: $\Lambda_\mu^{\phantom{\mu}\nu}\Lambda^\mu_{\phantom{\mu}\sigma}=\delta_\sigma^\nu$, és $(\Lambda^{-1})^\mu_{\phantom{\mu}\nu}=\Lambda_\nu^{\phantom{\nu}\mu}$, illetve általános tulajdonság, hogy $\big(\minv{M}^\text{T}\big)^\mu_{\phantom{\mu}\nu}=\minv{M}^{\phantom{\nu}\mu}_{\nu}$.
   
   A helyvektor transzformációja $\bar{\minv{x}}^\mu=\Lambda^\mu_{\phantom{\mu}\nu} \minv{x}^\nu$, ahonnan a kovariáns derivált:
   \eq{
    \bar{\partial}_\mu=\pder{}{\bar{\minv{x}}^\mu}=\pder{}{\minv{x}^\nu}\pder{\minv{x}^\nu}{\bar{\minv{x}}^\mu} = \pder{}{\minv{x}^\nu}\left(\Lambda^{-1}\right)^\nu_{\phantom{\nu}\mu}=\Lambda_\mu^{\phantom{\mu}\nu}\partial_\nu.
   }
   
   A transzformáció mezőkre: az elforgatott mező az elforgatott helyen olyan mintha az eredeti mezőt az eredeti helyen nézném, és elforgatnám az egészet:
   \aln{
    &\bar{\minv{A}}(\bar{\minv{x}})=\Lambda \minv{A}(\minv{x}) 
    &\bar{\minv{A}}(\minv{x})=\Lambda \minv{A}(\Lambda^{-1}\minv{x}).\label{eq:02-ltraf}
   }
   
   A Maxwell-egyenletek tényleg invariánsak egy ilyen transzformációra:
   \al{
    \bar{\partial}_\mu\bar{F}^{\mu\nu}(\bar{\minv{x}})
    &= \left(\Lambda_\mu^{\phantom{\mu}\mu'}\partial_{\mu'}\right)
       \left(\Lambda^\mu_{\phantom{\mu}\rho} \Lambda^\nu_{\phantom{\nu}\nu'} F^{\rho\nu'}(\minv{x})\right)
     = \underbrace{\left(\Lambda_\mu^{\phantom{\mu}\mu'}\Lambda^\mu_{\phantom{\mu}\rho} \right)}_{=\delta_\rho^{\mu'}}\Lambda^\nu_{\phantom{\nu}\nu'} \partial_{\mu'}F^{\rho\nu'}(\minv{x})
     = \Lambda^\nu_{\phantom{\nu}\nu'} \partial_{\mu'}F^{\mu'\nu'}(\minv{x}) \\
    &= \Lambda^\nu_{\phantom{\nu}\nu'} \mu_0\minv{J}^{\nu'}(\minv{x})
     = \mu_0\bar{\minv{J}}^{\nu'}(\minv{x}).
   }
   
   \paragraph{A Lorentz-transzformáció mátrixelemei}
    
    A $\Lambda$ mátrixa 16 elemet tartalmaz.
   A második feltétel mátrixos alakban:
    \al{
     \Lambda^\text{T}g\Lambda=g,
    }
    ami szimmetrikus a transzponálásra, így összesen 10 független egyenletet fogalmaz meg.
   Ezek szerint 6 paraméterrel jellemezhetjük a Lorentz-transzformációkat. 
    
    Ha veszünk egy speciális alakú 
    $\Lambda=\begin{pmatrix}
              1 & 0 \\
              0 & \mat{O}
             \end{pmatrix}$
    transzformációt, és helyettesítsük a második feltételbe. Így az $\mat{O}^T\mat{O}=1$ összefüggést kapjuk.
   Ezek a valós térbeli forgatásokat írják le.
   A hat paraméterből három ezekhez a transzformációkhoz tartozik.
    
    Tekinthetünk egy másik speciális alakú transzformációt:
    \eq{
     \Lambda=\begin{pmatrix}
              e & f & 0 & 0 \\
              g & h & 0 & 0 \\
              0 & 0 & 1 & 0 \\
              0 & 0 & 0 & 1 
             \end{pmatrix}
    }
    A 2. feltétel ennek a felső blokkjára a következőt adja:
    \al{
    &g\Lambda^\text{T}g\Lambda=1 
    &\Rightarrow &
    &\begin{pmatrix}
      1 & 0 \\
      0 & -1 
     \end{pmatrix}
     \begin{pmatrix}
      e & g \\
      f & h 
     \end{pmatrix}
     \begin{pmatrix}
      1 & 0 \\
      0 & -1 
     \end{pmatrix}
     \begin{pmatrix}
      e & f \\
      g & h 
     \end{pmatrix}
     =
     \begin{pmatrix}
      1 & 0 \\
      0 & 1 
     \end{pmatrix}
    } 
    \eq{
     \left.\begin{aligned}
            e^2-g^2&=1  \\
            h^2-f^2&=1  \\
              ef-hg&=0   
           \end{aligned}
     \right\}
     \Rightarrow
     \Lambda=\begin{pmatrix}
              \cosh \eta & \sinh \eta & 0 & 0 \\
              \sinh \eta & \cosh \eta  & 0 & 0 \\
               0 & 0 & 1 & 0 \\
               0 & 0 & 0 & 1 
             \end{pmatrix}
    }
    Az $\eta$ fizikai jelentése a helyvektor transzformációjából: legyen $\minv{x}$ a vesszős rendszer origójának helye.
   Ekkor $\bar{\minv{x}}^1=0=\sinh\eta\; \minv{x}^0+\cosh\eta \;\minv{x}^1 = \sinh\eta\; ct+\cosh\eta \;x $, ahonnan $x=-c\tanh\eta\;t$, vagyis $\tanh{\eta}=-\frac{v}{c}$.
   Ebből a transzformáció mátrixa:
    \eq{
    \Lambda=
     \begin{pmatrix}
      \frac{1}{\sqrt{1-\frac{v^2}{c^2}}} & \frac{-\frac{v}{c}}{\sqrt{1-\frac{v^2}{c^2}}} & 0 & 0\\
      \frac{-\frac{v}{c}}{\sqrt{1-\frac{v^2}{c^2}}} & \frac{1}{\sqrt{1-\frac{v^2}{c^2}}} & 0 & 0 \\
      0 & 0 & 1 & 0 \\
      0 & 0 & 0 & 1
     \end{pmatrix}
    }
    
    Ennek a transzformációnak a paramétere a mozgó koordináta-rendszer sebességétől függ: ezek a boostok.
   Mindhárom térbeli irányba lehetséges egy ilyen transzformáció: a maradék 3 paraméter a három $\eta$ (rapiditás). 
    
    A négyesvektorok transzformációja a boostok hatására:
    \begin{description}
     \item[Helyvektor:] $\minv{x}=(ct,x,y,z)$
      \al{
       &\bar{t}=\frac{t-\frac{vx}{c^2}}{\sqrt{1-\frac{v^2}{c^2}}}&
       &\bar{x}=\frac{x-vt}{\sqrt{1-\frac{v^2}{c^2}}} &
       &\bar{y}=y&
       &\bar{z}=z&
      } 
     \item[Áramsűrűség:] $\minv{J}=(c\rho,\vect{J})=(c\rho,J_x,J_y,J_z)$
      \al{
       &\bar{\rho}=\frac{\rho-\frac{vJ_x}{c^2}}{\sqrt{1-\frac{v^2}{c^2}}}&
       &\bar{J}_x=\frac{J_x-vt}{\sqrt{1-\frac{v^2}{c^2}}} &
       &\bar{J}_y=J_y&
       &\bar{J}_z=J_z&
      }
     \item[Négyespotenciál:] $\minv{A}=\left(\frac{1}{c}\phi,\vect{A}\right)=\left(\frac{1}{c}\phi,A_x,A_y,A_z\right)$
      \al{
       &\bar{\phi}=\frac{\phi-vA_x}{\sqrt{1-\frac{v^2}{c^2}}}&
       &\bar{A}_x= \frac{A_x-\frac{v\phi}{c^2}}{\sqrt{1-\frac{v^2}{c^2}}}&
       &\bar{A}_y=A_y&
       &\bar{A}_z=A_z&
      }
    \end{description}
    A térerősség tenzor transzformációja is kiszámolható: $\bar{F}^{\mu\nu}=\Lambda^\mu_{\phantom{\mu}\mu'}\Lambda^\nu_{\phantom{\nu}\nu'} F^{\mu'\nu'}$, a legegyszerűbb elemenként ($F$-ben nincs diagonális, $\Lambda$-ban sok elem nulla\dots).
   Eredmény:
    \aln{
     &\bar{E}_x=E_x&
     &\bar{E}_y=\frac{E_y-vB_z}{\sqrt{1-\frac{v^2}{c^2}}}&
     &\bar{E}_z=\frac{E_z+vB_y}{\sqrt{1-\frac{v^2}{c^2}}}&\label{eq:A2-Etrafo}
     \\
     &\bar{B}_x=B_x&
     &\bar{B}_y=\frac{B_y+\frac{vE_z}{c^2}}{\sqrt{1-\frac{v^2}{c^2}}}&
     &\bar{B}_z=\frac{B_z-\frac{vE_y}{c^2}}{\sqrt{1-\frac{v^2}{c^2}}}&\label{eq:A2-Btrafo}
    }
   
   \paragraph{Infinitezimális Lorentz-transzformációk}
    
    Keressük a Lorentz-transzformáció generátorát, azaz egy olyan $\ep^\mu_{\phantom{\mu}\nu}$ mátrixot, mellyel:
    \al{
     &\delta\Lambda^\mu_{\phantom{\mu}\nu}=\ep^\mu_{\phantom{\mu}\nu}
     &\Rightarrow
     &&\Lambda^\mu_{\phantom{\mu}\nu}=e^{\ep^\mu_{\phantom{\mu}\nu}}.
    } 
    Ezt az alakot behelyettesítve a Lorentz-transzformációra megadott egyenletbe:
    \al{
     \Lambda^{-1}&=g\Lambda^\text{T}g\\
     e^{-\ep}&=g e^{\ep^\text{T}}g=e^{g\ep^\text{T}g}\\
     \Downarrow\\
     -\ep&=g\ep^\text{T}g,
    }
    ahol felhasználtuk, hogy $gg=1$.
   Az egyenlet alapján $\ep$ elemeiről a következőket tudjuk: $\ep^\mu_{\phantom{\mu}\mu}=0$, $\ep^0_{\phantom{0}j}=\ep^j_{\phantom{j}0}$, ha $j=1,2,3$ és $\ep^j_{\phantom{j}k}=-\ep^k_{\phantom{k}j}$, ha $j,k=1,2,3$.
   Egyszerűen látszik az is, hogy $\ep^{\mu\nu}=-\ep^{\nu\mu}$ és $\ep_{\mu\nu}=-\ep_{\nu\mu}$.
   Ez összesen 6 paraméter az előző megfontolásoknak megfelelően.

  \subsection{Speciális relativitáselmélet}
   
   Einstein: ha egy tömegpont kölcsönhat az elektromágneses térrel, akkor a tömegpont mozgását leíró fizikai törvényeknél is a Lorentz-transzformációnak kell érvényesnek lenni a Galilei-transzformációval szemben.
   Einstein fizikai elvekből vezeti le a Lorentz-transzformációt:
   \begin{enumerate}
    \item A vákuumbeli fénysebesség minden inerciarendszerben ugyanakkora: $c$.
    \item Minden inerciarendszer egymással ekvivalens, vagyis a fizikai törvények ugyanolyan alakúak.
   \end{enumerate}
   
   A fenti két posztulátumot feltételezve, illetve a transzformációt lineárisnak keresve megkapjuk a Lorentz-transzformációt.
   A transzformáció nem hagyja invariánsan az időt sem: érdemes itt is bevezetni a négyes vektorokat és a négydimenziós Minkowski-féle téridőt.
   
  \subsection{A mechanika kovariáns formalizmusa}
   
   A helyvektorok az $\minv{x}=(ct,\vect{r})$ alakúak.
   A tömegpont mozgásának leírásához szükséges a pálya ($\gamma(t)$) általánosítása.
   Ehhez $s$ egy paraméter: $\gamma^\mu(s)=(ct(s),\vect{r}(s))$.
   A pálya ívhossza, mivel skalárszorzatot tartalmaz, így invariáns.
   Paraméterezzük a mozgást az idővel: $\minv{\gamma}^\mu(t)=(ct,\vect{r}(t))$.
   Ennek segítségével definiálhatunk egy újabb invariánst, a sajátidőt:
   \al{
    &c\tau=\intl{t_1}{t_2}\dd t\;\sqrt{\frac{\dd\gamma^\mu}{\dd t}\frac{\dd\gamma_\mu}{\dd t}}
     =\intl{t_1}{t_2}\dd t\;\sqrt{(c,\dot{\vect{r}})\cdot(c,\dot{\vect{r}})}
     =\intl{t_1}{t_2}\dd t\;\sqrt{c^2-v^2}&
    &\Rightarrow&
    &\boxed{\dd\tau=\dd t\sqrt{1-\frac{v^2}{c^2}}}&
   }
   
   A négyessebesség a kontravariáns koordináta sajátidő szerinti deriváltja:
   \eq{
    \minv{u}^\mu=\der{\minv{x}^\mu}{\tau}=\left(\frac{c}{\sqrt{1-\frac{v^2}{c^2}}},\frac{\vect{v}}{\sqrt{1-\frac{v^2}{c^2}}}\right).
   }
   Ez is kontravariáns vektor, így ennek is a hossznégyzete megmarad, speciálisan $\minv{u}^\mu\minv{u}_\mu=c^2$. 
   
   A négyessebességet egy invariáns skalárral megszorozva is kontravariáns vektort kapunk.
   Legyen ez a skalár a nyugalmi tömeg ($m_0$), vagyis az együttmozgó koordináta-rendszerben mért tömeg, így a négyesimpulzust kapjuk:
   \eq{
    \minv{p}^\mu=m_0\minv{u}^\mu=\left(\frac{m_0c}{\sqrt{1-\frac{v^2}{c^2}}},\frac{m_0\vect{v}}{\sqrt{1-\frac{v^2}{c^2}}}\right),
   }
   aminek hossznégyzete: 
   \eqn{\minv{p}^\mu\minv{p}_\mu=m_0^2c^2.\label{eq:02-kinimphossz}}
   
   A négyesimpulzus 0. komponensét fejtsük ki $v\ll c$ határesetben:
   \eq{
    \frac{m_0c}{\sqrt{1-\frac{v^2}{c^2}}}\approx \frac{1}{c}\left(m_0c^2+\frac 12m_0 v^2+\dots\right).
   }
   Ennek az első tagja egy konstans, vagyis határesetben a zárójelben szereplő összeg megfelel a klasszikus mozgási energiának.
   Ezentúl rögzítjük az energia nullpontját, ez legyen $m_0c^2$.
   A teljes energia: $E=\frac{m_0c^2}{\sqrt{1-\frac{v^2}{c^2}}}=mc^2$, mely a klasszikus határesetet úgy adja vissza, ha $E=m_0c^2+E_\text{kin}$, azaz $E_\text{kin}=E-m_0c^2$.
   Bevezettük a mozgási tömeget:
   \eq{
    m:=\frac{m_0}{\sqrt{1-\frac{v^2}{c^2}}},
   }
   mellyel a négyesimpulzus:
   \eq{
    \minv{p}^\mu=(mc,m\vect{v})=\left(\frac{E}{c},\vect{p}\right).
   }
   Az invariáns hossznégyzetre vonatkozó összefüggés így kifejthető:
   \al{
    &\minv{p}^\mu\minv{p}_\mu=m_0^2c^2=\frac{E^2}{c^2}-\vect{p}^2&
    &\Leftrightarrow&
    &\boxed{E^2=c^2\vect{p}^2+m_0^2c^4}.&
   }
   
   Készítsük el a négyesimpulzus sajátidő szerinti deriváltját, a négyeserőt:
   \eq{
    \minv{K}^\mu=\der{\minv{p}^\mu}{\tau}=\left(\frac{1}{\sqrt{1-\frac{v^2}{c^2}}}\frac{1}{c}\der{E}{t},\frac{1}{\sqrt{1-\frac{v^2}{c^2}}}\der{\vect{p}}{t}\right).
   }
   Vizsgáljuk meg ennek tagjait.
   A négyesimpulzus hossza skalár, így időtől független: $0=\der{\minv{p}^2}{t}=2\minv{p}^0\der{\minv{p}^0}{t}-\suml{i=1}{3}2\minv{p}^i\der{\minv{p}^i}{t}$ $\Rightarrow$ $cm\der{(cm)}{t}=m\vect{v}\der{(m\vect{v})}{t}$ $\Rightarrow$ $\der{E}{t}=\der{(mc^2)}{t}=\vect{v}\der{\vect{p}}{t}$.
   A hármasimpulzus időbeli változása a Newton-egyenletnek megfelelően kell, hogy történjen, mely a 
   \eqn{
    \der{E}{t}=\der{}{t}(mc^2)=\vect{v}\vect{F}\label{eq:02-relmunkatetel}
   }
   összefüggést adja, vagyis a teljes energia megváltozása az $\vect{F}$ erő által végzett munka.
   A négyeserő második komponensében is felhasználjuk a Newton-egyenletet, így:
   \eq{
    \minv{K}^\mu=\left(\frac{1}{\sqrt{1-\frac{v^2}{c^2}}}\frac{\vect{v}\vect{F}}{c},\frac{1}{\sqrt{1-\frac{v^2}{c^2}}}\vect{F}\right).
   }
   
   Így tehát a Newton-egyenlet relativisztikus általánosítása:
   \eqn{
    \boxed{\minv{K}^\mu=\der{\minv{p}^\mu}{\tau}}.\label{eq:02-relNewton}
   }
   
   Ha az erők potenciálosak, akkor \eqaref{eq:02-relmunkatetel} egyenlet alapján a teljes energia átírható, az új megmaradó mennyiség a 
   \eqn{
    E_\text{tot}=E+U=mc^2+U=m_0c^2+E_\text{kin}+U.
   }
  
  \subsection{Töltött relativisztikus részecske Lagrange- és Hamilton-függvénye}\label{ss:02-relqLagrangeHamilton}
   
   Egy olyan Lagrange-függvényt keresünk, amelyből a megfelelő Euler--Lagrange-egyenletek előállítják a a külső térbe helyezett töltésre vonatkozó mozgásegyenleteket.
   Először vizsgáljuk a szabad részecske esetét. 
   
   A hatás:
   \eq{
    S=\intl{t_1}{t_2}\dd t\; L = S=\intl{\tau_1}{\tau_2}\dd \tau\;\frac{1}{\sqrt{1-\frac{v^2}{c^2}}} L. 
   }
   A hatásnak Lorentz-invariánsnak kell lennie, hiszen annak ugyanakkorának kell lennie minden inerciarendszerben.
   Mivel a sajátidő invariáns, ezért az integrandusnak is annak kell lenni minden sajátidő-pillanatban.
   A szabad részecske Lagrange-függvénye csak a tömegtől és a részecske négyessebességétől függhet, és négyessebesség Lorentz-invariáns alakja: $\minv{u}^\mu\minv{u}_\mu=c^2$, ezért egy kézenfekvő megoldás az $L=\alpha\sqrt{1-\frac{v^2}{c^2}}$ választás.
   Az $\alpha$ illesztése a klasszikus határesethez:
   \al{
    &L=\alpha\left(1-\frac{v^2}{2c^2}+\dots\right)\overset{!}{=}\text{const}+\frac 12 m_0v^2&
    &\Rightarrow&
    &\alpha=-m_0c^2,&
   }
   így
   \eq{
    L=-m_0c^2\sqrt{1-\frac{v^2}{c^2}}.
   }
   
   A kölcsönhatást a Lagrange-függvényen kiegészül egy additív taggal ($L_\text{int}$) tudjuk figyelembe venni.
   A klasszikus határesetben ($v\to 0$) a potenciális energia $U=q\phi$.
   Mivel $\phi$ az $\minv{A}^\mu$-nek egy komponense, és szükséges a Lorentz-invariancia itt is, ezért feltehetjük, hogy a négyespotenciál egy másik négyesvektorral vett skaláris szorzata szerepel a kölcsönhatásban.
   Ez a másik vektor a négyessebesség és -koordináta lehet.
   A transzlációinvariancia a másodikat kizárja, így szükségképpen:
   \eq{
    L_\text{int}=-U=-q\minv{u}^\mu\minv{A}_\mu\sqrt{1-\frac{v^2}{c^2}}=-q\phi+q\vect{A}\vect{v},
   }
   vagyis az elektromos térben mozgó részecske Lagrange-függvénye:
   \eqn{
    \boxed{L=\sqrt{1-\frac{v^2}{c^2}}\left(-m_0 c^2-q\minv{u}^\mu \minv{A}_\mu\right)=-m_0c^2\sqrt{1-\frac{v^2}{c^2}}-q\phi+q\vect{A}\vect{v}}.\label{eq:02-relLagrange}
   }
   
   A kanonikus impulzus:
   \eqn{
    \vect{p}=\pder{L}{\vect{v}}=\frac{m_0\vect{v}}{\sqrt{1-\frac{v^2}{c^2}}}+q\vect{A}=m\vect{v}+q\vect{A}=\vect{k}+q\vect{A},\label{eq:02-kinimpdef}
   }
   ahol $\vect{k}=m\vect{v}$ a kinetikus impulzus (áttértünk innentől erre a jelölésre), illetve a Hamilton-függvény:
   \eq{
    H=\vect{p}\vect{v}-L=m\vect{v}^2+q\vect{A}\vect{v}+m(c^2-\vect{v}^2)+q\phi-q\vect{A}\vect{v}=mc^2+q\phi.
   }
   
   Elkészíthetjük az elektromágneses térben mozgó relativisztikus részecske Hamilton-Jacobi egyenletét is.
   Ehhez $H$-t átalakítjuk:
   \eq{
   H=E+q\phi=\sqrt{c^2\vect{k}^2+m_0^2c^4}+q\phi
    =c\sqrt{\left(\vect{p}-q\vect{A}\right)^2+m_0^2c^2}+q\phi.
   }
   A kanonikus transzformáció szerint pedig $\vect{p}=\pder{S}{\vect{r}}$ és $H=-\pder{S}{t}$, melyekkel:
   \eqn{
    \boxed{\left(\pder{S}{t}+q\phi\right)^2-c^2\left(\pder{S}{\vect{r}}-q\vect{A}\right)^2=m_0^2c^4}.
   }
   
 \section{Kvantummechanika}\label{ss:02-kvantum}
   
  \subsection{A Klein--Gordon-egyenlet}
   
   A kommutációs relációkat meg kell tartania a relativisztikus általánosításnak is.
   Fontos, hogy az impulzus operátora a kanonikus impulzushoz tartozik, így a $[\op{\minv{p}}^\mu,\op{\minv{x}}^\nu]=\frac{\hbar}{i}\delta^{\mu\nu}$.
   A $\minv{p}^\mu$ operátort úgy választjuk meg, hogy teljesüljön a fenti reláció (lásd \ref{ss:05-kankinimp}. fejezet):
   \al{
   &\op{\minv{x}}^\mu
      =\minv{x}^\mu\cdot
      =\left(ct,\rv\right)\cdot \\
   &\op{\minv{p}}^\mu
      =i\hbar\partial^\mu 
      =\left(-\frac{\hbar}{i}\frac 1c\partial_t,\frac{\hbar}{i}\vects{\nabla}\right)
   & \Leftrightarrow&
   &\op{\minv{p}}_\mu=i\hbar\partial_\mu=\left(-\frac{\hbar}{i}\frac 1c\partial_t,-\frac{\hbar}{i}\vects{\nabla}\right).
   }
   
   A kinetikus impulzus operátora \eqaref{eq:02-kinimpdef} egyenletnek megfelelően: 
   \eqn{
    \op{\minv{k}}_\mu=\op{\minv{p}}_\mu-q\op{\minv{A}}_\mu
    =\left(-\frac{\hbar}{i}\frac{1}{c}\partial_t-\frac qc\phi,-\frac{\hbar}{i}\vects{\nabla}+q\vect{A}\right).\label{eq:02-kinimpopdef}
   }
   
   \Eqaref{eq:02-kinimphossz} egyenlet alapján (figyelem, az összefüggés fent is a kinetikus impulzusra volt igaz, ott nem volt még $\vect{A}$!):
   \al{
    &\ominv{k}^\mu\ominv{k}_\mu=m_0^2c^2&
    &\Rightarrow&
    &\left(-\frac{\hbar}{i}\frac{1}{c}\partial_t-\frac qc\phi\right)^2-\left(-\frac{\hbar}{i}\vects{\nabla}+q\vect{A}\right)^2=m_0^2c^2,&
   }
   Ez pedig mivel azonosság, formálisan hathat egy $\psi(t,\vect{r})$ hullámfüggvényre is:
   \eqn{
    \boxed{\left[
    \left(\frac{\hbar}{i}\vects{\nabla}-q\vect{A}\right)^2 - \frac{1}{c^2}\left(i\hbar\partial_t-q\phi\right)^2
    \right ]\psi(t,\vect{r})
    =-m_0^2c^2\psi(t,\vect{r})}.\label{eq:02-kleingordon}
   }
   
   Ennek a nulla terekhez tartozó esete egy egyszerű hullámegyenlet: 
   \al{
    &\left(\Delta -\frac{1}{c^2}\partial_t^2\right)\psi(t,\vect{r})=\frac{m_0^2c^2}{\hbar^2}\psi(t,\vect{r})
    &-\partial^\mu\partial_\mu\psi(t,\vect{r})=\frac{m_0^2c^2}{\hbar^2}\psi(t,\vect{r}),
   }
   ami szembeötlően Lorentz-invariáns. 
   
   A Klein--Gordon-egyenlet azonban hibás eredményt a H-atom finomszerkezetére, és a hullámfüggvény mint valószínűségi amplitúdó sem értelmezhető, illetve a spin létére sem ad magyarázatot.
   A probléma a kétszeres időderiválás, Dirac: keressünk lineáris, Lorentz-invariáns egyenletet.
   
  \subsection{Dirac-egyenlet}
   
   Keressük a szabad részecske mozgásegyenletét az alábbi feltételek mellett:
   \begin{enumerate}
    \item Legyen a deriválásban lineáris az egyenlet.
    \item Teljesüljön az energia-impulzus közötti kapcsolat.
   \end{enumerate}
   Ezeknek megfelel az, ha az operátort az alábbi alakban keressük:
    \eq{
     \big(\op{\minv{p}}^\mu\op{\minv{p}}_\mu-m_0^2c^2\big)=
     \left(\op{\gamma}^\mu\op{\minv{p}}_\mu+m_0c\right)
     \left(\op{\gamma}^\nu\op{\minv{p}}_\nu-m_0c\right)
    }
    Itt megjelenik egy új szabadsági fokokat tartalmazó Hilbert-tér, ahol $\op{\gamma}^\mu$ hat.
   A fenti egyenlőség teljesüléséhez szükséges, hogy a $\op{\gamma}^\mu$ mátrixokra igaz legyen az alábbi összefüggés:
   \al{
    \op{\gamma}^\mu\op{\gamma}^\nu\op{\minv{p}}_\mu\op{\minv{p}}_\nu
     &=\frac 12 \left(
        \op{\gamma}^\mu\op{\gamma}^\nu\op{\minv{p}}_\mu\op{\minv{p}}_\nu
       +\op{\gamma}^\nu\op{\gamma}^\mu\op{\minv{p}}_\nu\op{\minv{p}}_\mu
      \right)
     =\frac 12 \left(
        \op{\gamma}^\mu\op{\gamma}^\nu\op{\minv{p}}_\mu\op{\minv{p}}_\nu
       +\op{\gamma}^\nu\op{\gamma}^\mu\op{\minv{p}}_\mu\op{\minv{p}}_\nu
      \right)\\
     &=\frac 12 \left(\op{\gamma}^\mu\op{\gamma}^\nu+\op{\gamma}^\nu\op{\gamma}^\mu\right)\op{\minv{p}}_\mu\op{\minv{p}}_\nu
     \overset{!}{=}
     \op{\minv{p}}^\mu\op{\minv{p}}_\mu
     =\op{\minv{p}}_{\mu}\op{\minv{p}}_\nu \minv{g}^{\mu\nu}
   }
   vagyis
   \al{
     &\op{\gamma}^\mu\op{\gamma}^\nu+\op{\gamma}^\nu\op{\gamma}^\mu=2\minv{g}^{\mu\nu}\op{I}
     &\Rightarrow
     &&\boxed{\big\{\op{\gamma}^\mu,\op{\gamma}^\nu\big\}=2\minv{g}^{\mu\nu}\op{I}}.
   }
   Ebből láthatjuk, hogy $(\op{\gamma}^0)^2=\op I$, $(\op{\gamma}^k)^2=-\op I$.
   Innen az következik, hogy $\op{\gamma}^0$ sajátértékei a $\pm1$, illetve $\op{\gamma}^j$ sajátértékei $\pm i$.
   Ezeket tehát választhatom rendre unitér és antiunitér operátoroknak. 
   
   Keressük ezeknek az operátoroknak egy véges dimenziós reprezentációját.
   A $\op{\gamma}^\mu$-t véges dimenziós mátrixokként szeretnénk felírni, ekkor az ehhez tartozó Hilbert-tér is véges dimenziós, vagyis a hullámfüggvény a négyzetesen integrálható függvények és egy véges dimenziós Hilbert-tér direkt szorzatának az eleme.
   
   \paragraph{Dirac-mátrixok}
    
    $\mu\ne\nu$-re $\op\gamma^\mu\op\gamma^\nu=-\op\gamma^\nu\op\gamma^\mu$, így
    \al{
     \det{\op\gamma^\mu\op\gamma^\nu}
     &=\det{\big(-\op\gamma^\nu\op\gamma^\mu\big)}
      =(-1)^d\det{\op\gamma^\nu\op\gamma^\mu}
      =(-1)^d\det{\op\gamma^\nu}\det{\op\gamma^\mu}
      =(-1)^d\det{\op\gamma^\mu}\det{\op\gamma^\nu}\\
     &=(-1)^d\det{\op\gamma^\mu\op\gamma^\nu}.
    } 
    A mátrixok determinánsára nem szeretnénk kikötni, hogy nulla legyen, így $d=2k$ lehet csak. 
    
    Két dimenzióban csak három antikommutáló mátrix létezik (Pauli-mátrixok), így legalább négy dimenziósnak kell lennie az ábrázolásnak.
   Négy dimenzióban van is ilyen, a Dirac-féle ábrázolás:
    \al{
     &\gamma^0=\begin{pmatrix}
                1 & &  &  \\
                & 1 &  &  \\
                & & -1 & \\
                & & & -1
               \end{pmatrix}
     &\gamma^1=\begin{pmatrix}
                & & & 1 \\
                & & 1 & \\
                & -1 & & \\
                -1 & & & 
               \end{pmatrix}
     & \\
     &\gamma^2=\begin{pmatrix}
                & & & -i \\
                & & i & \\
                & i & & \\
                -i & & & 
               \end{pmatrix}
     &\gamma^3=\begin{pmatrix}
                & & 1 &  \\
                & & & -1 \\
                -1 & & & \\
                 & 1 & & 
               \end{pmatrix},
     &
    }
   Ami a Pauli-mátrixokkal, $\sigma_1=\bigl(\begin{smallmatrix}
                                        & 1 \\
                                       1&  \\
                                      \end{smallmatrix}\bigr)$,
                            $\sigma_2=\bigl(\begin{smallmatrix}
                                        & -i \\
                                       i & \\
                                      \end{smallmatrix}\bigr)$ és 
                            $\sigma_3=\bigl(\begin{smallmatrix}
                                       1 &  \\
                                        & -1\\
                                      \end{smallmatrix}\bigr)$összefoglalható: 
   \al{
    &\gamma^0=\begin{pmatrix}
               I_2 & \\
                & -I_2
              \end{pmatrix}.
    &\gamma^j=\begin{pmatrix}
               & \sigma_j \\
              -\sigma_j &  \\
             \end{pmatrix}\quad j=1,2,3
   }
   Összefoglalva tehát a Dirac-egyenlet szabad részecskék esetében:
   \al{
    (\op{\gamma}^\mu\op{\minv{p}}_\mu-m_0c\op{I})\ket{\Psi}=0.
   }
   
   Elektromágneses terek esetében a $\op{\minv{p}}_\mu$ helyett a kinetikus impulzussal kel felírni az egyenletet:
   \eqn{
   (\op{\gamma}^\mu\op{\minv{k}}_\mu-m_0c\op{I})\ket{\Psi}=0.\label{eq:02-DiracEM}
   }
   
   Ennek kifejtéséhez használjuk a $\op{\gamma}^\mu$ operátorok fenti reprezentációját és a másik Hilbert-téren pedig koordináta-reprezentációt. 
   
   Az operátort kifejtve:
   \eq{
   \Big[\gamma^\mu\left(-\frac{1}{c}\frac{\hbar}{i}\partial_t-\frac qc\phi\,,\,-\vect{p}+q\vect{A}\right)-m_0c\op{I}\Big]\Psi(\minv{x})=0
   }
   majd mindkét oldalt $c\gamma^0$-lal bővítve:
   \al{
     \left[\left(i\hbar\partial_t-q\phi\right)+c\gamma^0\vects{\gamma}\left(-\vect{p}+q\vect{A}\right)-m_0c^2\gamma^0\right]\Psi(\minv{x})=0,
   }
   
   Bevezetjük az $\op\alpha_k$ és a $\beta$ mátrixokat:
   \al{
    &\vects{\op\alpha}=\gamma^0\vects{\gamma}=\begin{pmatrix}
                                             0 & \vects{\sigma} \\
                                             \vects{\sigma} & 0
                                            \end{pmatrix}
    &\beta=\begin{pmatrix}
            I_2 & 0 \\
            0 & -I_2
           \end{pmatrix},
   }
   melyekkel az időfüggő Dirac-egyenlet elektromágneses terek jelenlétében:
   \eqn{
    \boxed{i\hbar\partial_t\Psi(\minv{x})=\big[c\vects{\op\alpha}\left(\vect{p}-q\vect{A}\right)+q\phi+m_0c^2\op\beta\big]\Psi(\minv{x})},\label{eq:02-Dirac}
   }
   melyről a Dirac--Hamilton-operátor leolvasható:
   \eqn{
    \boxed{
    H=c\vects{\op{\alpha}}\left(\op{\vect{p}}-q\vect{A}\right)+q\phi+m_0c^2\op{\beta}}.\label{eq:02-DE}
   }

  \subsection{A Pauli--Schrödinger-egyenlet származtatása}
   
   Induljunk ki \eqaref{eq:02-DiracEM} egyenletből:
   \al{
    0
     &=(\op{\gamma}^\mu\op{\minv{k}}_\mu-m_0c\op{I})\ket{\Psi}
      =(\op{\gamma}^\nu\op{\minv{k}}_\nu+m_0c\op{I})(\op{\gamma}^\mu\op{\minv{k}}_\mu-m_0c\op{I})\ket{\Psi}\\
     &=(\op{\gamma}^\nu\op{\gamma}^\mu\op{\minv{k}}_\nu\op{\minv{k}}_\mu-m_0^2c^2\op{I})\ket{\Psi}
      =\left(\big(\op{\minv{k}}^\mu\op{\minv{k}}_\mu-m_0^2c^2\big)\op{I}+\suml{\mu\ne\nu}{}\op{\gamma}^\nu\op{\gamma}^\mu\op{\minv{k}}_\nu\op{\minv{k}}_\mu\right)\ket{\Psi}.
   }
   A második tag:
   \al{
    \suml{\mu\ne\nu}{}\op{\gamma}^\nu\op{\gamma}^\mu\op{\minv{k}}_\nu\op{\minv{k}}_\mu
     &=\frac{1}{2}\suml{\mu\ne\nu}{}\Big(\op{\gamma}^\nu\op{\gamma}^\mu\op{\minv{k}}_\nu\op{\minv{k}}_\mu+\op{\gamma}^\mu\op{\gamma}^\nu\op{\minv{k}}_\mu\op{\minv{k}}_\nu\Big)\\
     &=\frac{1}{2}\suml{\mu\ne\nu}{}\Big(\big(\op{\gamma}^\nu\op{\gamma}^\mu+\op{\gamma}^\mu\op{\gamma}^\nu\big)\op{\minv{k}}_\nu\op{\minv{k}}_\mu+\op{\gamma}^\mu\op{\gamma}^\nu\big[\op{\minv{k}}_\mu,\op{\minv{k}}_\nu\big]\Big)\\
     &=\frac{1}{2}\suml{\mu\ne\nu}{}\op{\gamma}^\mu\op{\gamma}^\nu\big[\op{\minv{k}}_\mu,\op{\minv{k}}_\nu\big]
   }
   Felhasználjuk a kinetikus impulzus kommutációs relációit (\eqref{eq:05-kinkomm00}, \eqref{eq:05-kinkomm0i} és \eqref{eq:05-kinkommij} egyenletek), melyekkel:
   \al{
    \suml{\mu\ne\nu}{}\op{\gamma}^\nu\op{\gamma}^\mu\op{\minv{k}}_\nu\op{\minv{k}}_\mu
     &=\frac{1}{2}\suml{i\ne j}{}\op{\gamma}^i\op{\gamma}^j\big[\op{\minv{k}}_i,\op{\minv{k}}_j\big]+\op{\gamma}^0\op{\gamma}^i\big[\op{\minv{k}}_0,\op{\minv{k}}_i\big]\\
     &=-\frac{1}{2}\op{\gamma}^i\op{\gamma}^j\frac{\hbar q}{i}\ep_{ijk}\vect{B}_k+\op{\gamma}^0\op{\gamma}^i\frac{\hbar q}{ic}\vect{E}_i.
   }
   Behelyettesítve:
   \al{
    0=
      \left(
       \big(\op{\minv{k}}^\mu\op{\minv{k}}_\mu-m_0^2c^2\big)\op{I}
       -\frac{\hbar q}{2i}\ep_{ijk}\op{\gamma}^i\op{\gamma}^j\vect{B}_k+\frac{\hbar q}{ic}\op{\gamma}^0\op{\gamma}^i\vect{E}_i
      \right)
      \ket{\Psi}.
   }
   A Klein--Gordon részt behelyettesítjük, illetve átalakítjuk a mágneses teret tartalmazó tagot:
   \al{
    \ep_{ijk}\op{\gamma}^i\op{\gamma}^j
     =\ep_{ijk}\begin{pmatrix}
                   & \vects{\sigma}_i \\
                  -\vects{\sigma}_i &
                \end{pmatrix}\cdot
                \begin{pmatrix}
                   & \vects{\sigma}_j \\
                  -\vects{\sigma}_j &
                \end{pmatrix}
      =-\ep_{ijk}\vects{\sigma}_i\vects{\sigma}_j\op{I}
      =-2i\vects{\sigma}_k\op{I}
      =-2i\begin{pmatrix}
           \vects{\sigma} & 0\\
           0 & \vects{\sigma}
          \end{pmatrix}_k
      =-2i\vects{\Sigma}_k,
   }
   így
   \al{
    0=
      \left[
       \frac{1}{c^2}
       \left(
        i\hbar\partial_t-q\phi
       \right)^2
       -\left(
        \vect{\op{p}}-q\vect{A}
       \right)^2
       -m_0^2c^2
       +\frac{\hbar q}{ic}\vects{\alpha}\vect{E}
       +\hbar q\vects{\Sigma}\vect{B}
      \right]\ket{\Psi}.
   }
   
   Tekintsük a kisenergiás (nemrelativisztikus) határesetet.
   Ekkor
   \al{
    m_0^2c^2-\frac{1}{c^2}\left(i\hbar\partial_t-q\phi\right)^2
     &=\frac{1}{c^2}\big(m_0c^2-i\hbar\partial_t+q\phi\big)\underbrace{\big(m_0c^2+i\hbar\partial_t-q\phi\big)}_{\sim2m_0c^2}\\
     &\approx 2m_0\big[m_0c^2-i\hbar\partial_t+q\phi\big],
   }
   vagyis
   \al{
    i\hbar\partial_t\ket{\Psi}=
      \left[
       \frac{1}{2m_0}\left(
        \vect{\op{p}}-q\vect{A}
       \right)^2
       +m_0c^2+q\phi
       -\frac{\hbar q}{2im_0c}\vects{\alpha}\vect{E}
       -\frac{\hbar q}{2m_0}\vects{\Sigma}\vect{B}
      \right]\ket{\Psi}.
   }
   
   Az elektromos térerősséget tartalmazó tag $c$-vel el van nyomva $q\phi$ mellett, így azt is jó közelítéssel elhanyagolhatjuk.
   Most már előállt Pauli--Schrödinger-egyenlet.
   Ha az energiát $m_0c^2$-től mérjük, és elektronra írjuk fel az egyenletet, akkor 
   \al{
    E\ket{\Psi}=\Bigg[
       \frac{1}{2m_0}\left(
        \vect{\op{p}}-q\vect{A}
       \right)^2
       +q\phi
       +2\underbrace{\frac{\hbar e}{2m_0}}_{\mu_\text{B}}\underbrace{\frac{1}{2}\vects{\Sigma}}_{\op{\vect{S}}}\vect{B}
      \Bigg]\ket{\Psi}.
   }
   
  \subsection{Relativisztikus korrekciók: kinetikus energia tag, Darwin-tag, Spin--pálya kölcsönhatás}
   
   Tekintsük az időtől független Dirac-egyenletet, melyet \eqaref{eq:02-DE} operátorral írunk fel:
   \al{
    \Big(c\vects{\op{\alpha}}\left(\op{\vect{p}}-q\vect{A}\right)+q\phi+m_0c^2\op{\beta}\Big)\Psi=E\Psi.
   }
   A $\Psi$ bispinort felírjuk két spinort tartalmazó alakban: $\Psi=\begin{pmatrix}\chi\\ \varphi\end{pmatrix}$, mellyel a fenti egyenlet:
   \al{
    \begin{pmatrix}
     E-m_0c^2-q\phi & -c\vects{\sigma}\op{\vect{k}}\\
     -c\vects{\sigma}\op{\vect{k}} & E+m_0c^2-q\phi
    \end{pmatrix}
    \begin{pmatrix}
     \chi \\ \varphi
    \end{pmatrix}
    =0
   }
   Komponensenként külön felírva:
   \al{
    \big(E-m_0c^2-q\phi\big)\chi-c\vects{\sigma}\op{\vect{k}}\varphi&=0\\
    -c\vects{\sigma}\op{\vect{k}}\chi+\big(E+m_0c^2-q\phi\big)\varphi&=0.
   }
   Fejezzük ki a második egyenletből $\varphi$-t:
   \al{
    \varphi
     =\frac{1}{E+m_0c^2-q\phi}c\vects{\sigma}\op{\vect{k}}\chi
   }
   Felhasználjuk, hogy kis energiákon vagyunk.
   Ehhez a nevezőt sorba fejtsük: 
   \al{
    \frac{1}{E+m_0c^2-q\phi}
     &=\frac{1}{2m_0c^2+E-m_0c^2-q\phi}
      =\frac{1}{2m_0c^2+E'-q\phi}
      =\frac{1}{2m_0c^2}\cdot\frac{1}{1+\frac{E'-q\phi}{2m_0c^2}}\\
     &=\frac{1}{2m_0c^2}\cdot\left(1+\frac{E'-q\phi}{2m_0c^2}\right)^{-1}
      \approx\frac{1}{2m_0c^2}\cdot\left(1-\frac{E'-q\phi}{2m_0c^2}\right).
   }
   Ha itt megálltunk volna a nulladrendű tagnál, akkor a Pauli--Schrödinger-egyenletet kaptuk volna vissza.
   Ehhez képest most még az $\frac{1}{c^2}$-tel arányos tagokat is megtartjuk.
   Ezt az első egyenletbe helyettesítve a nagykomponensre az alábbi egyenlet adódik:
   \al{
    (E'-q\phi)\chi=\frac{1}{2m_0}\cdot(\vects{\sigma}\op{\vect{k}})\left(1-\frac{E'-q\phi}{2m_0c^2}\right)(\vects{\sigma}\op{\vect{k}})\chi.
   }
   Foglalkozzunk a jobb oldal kifejtésével.
   Ehhez
   \al{
    (\vects{\sigma}\op{\vect{k}})(\vects{\sigma}\op{\vect{k}})
     &=\op{\vect{k}}^2+i\vects{\sigma}\big(\op{\vect{k}}\times\op{\vect{k}}\big)
      =\big(\op{\vect{p}}-q\vect{A}\big)^2+i\frac{1}{2}\ep_{ijk}\vects{\sigma}_k\big[\op{\vect{k}}_i,\op{\vect{k}}_j\big]\\
     &=\big(\op{\vect{p}}-q\vect{A}\big)^2-i\frac{1}{2}\ep_{ijk}\vects{\sigma}_k\frac{\hbar q}{i}\ep_{ijk}\vect{B}_k
      =\big(\op{\vect{p}}-q\vect{A}\big)^2-\hbar q\vects{\sigma}\vect{B},
   }
   mellyel
   \al{
    (E'-q\phi)\chi
     &=\frac{1}{2m_0}\cdot(\vects{\sigma}\op{\vect{k}})(\vects{\sigma}\op{\vect{k}})\chi
      -\frac{1}{2m_0}\cdot(\vects{\sigma}\op{\vect{k}})\frac{E'-q\phi}{2m_0c^2}(\vects{\sigma}\op{\vect{k}})\chi\\
    E'\chi
     &=\underbrace{\left(\frac{\big(\op{\vect{p}}-q\vect{A}\big)^2}{2m_0}-\frac{\hbar q}{2m_0}\vects{\sigma}\vect{B}+q\phi\right)}_{\op{H}_\text{PS}}\chi
      -\frac{1}{2m_0}\cdot(\vects{\sigma}\op{\vect{k}})\frac{E'-q\phi}{2m_0c^2}(\vects{\sigma}\op{\vect{k}})\chi
   }
   Láthatjuk tehát, hogy a Pauli--Schrödinger Hamilton-operátorhoz a korrekciókat a második tag adja.
   Ezt átalakítjuk:
   \al{
    -\frac{1}{2m_0}\cdot(\vects{\sigma}\op{\vect{k}})\frac{E'-q\phi}{2m_0c^2}(\vects{\sigma}\op{\vect{k}})\chi
    =\underbrace{-\frac{1}{2m_0}\cdot(\vects{\sigma}\op{\vect{k}})(\vects{\sigma}\op{\vect{k}})\frac{E'-q\phi}{2m_0c^2}}_{\op{H}_\text{K}}\chi
    +\underbrace{\frac{1}{2m_0}\cdot(\vects{\sigma}\op{\vect{k}})\left[(\vects{\sigma}\op{\vect{k}}),\frac{E'-q\phi}{2m_0c^2}\right]}_{\text{II.}}\chi,
   }
   ahol $\frac{1}{c^2}$ rendig:
   \al{
    \op{H}_\text{K}
     &=-\left(\op{H}_\text{PS}-q\phi\right)\frac{1}{2m_0c^2}\big(E'-q\phi\big)
      \approx-\left(\op{H}_\text{PS}-q\phi\right)\frac{1}{2m_0c^2}\left(\op{H}_\text{PS}-q\phi\right)\\
     &=-\frac{1}{2m_0c^2}\left(\op{H}_\text{PS}-q\phi\right)^2
      \approx\frac{1}{2m_0c^2}\left(\frac{\op{\vect{k}}^2}{2m_0}\right)^2
      =-\frac{1}{8m_0^3c^2}\op{\vect{k}}^4
   }
   \al{
    \text{II.}
     &=-\frac{1}{4m_0^2c^2}\cdot \sigma_i\op{\vect{k}}_i\left[\vects{\sigma}_j\op{\vect{k}}_j,q\phi\right]
      =-\frac{1}{4m_0^2c^2}\cdot \vects{\sigma}_i\vects{\sigma}_j\op{\vect{k}}_i\left[\op{\vect{k}}_j,q\phi\right]\\
     &=-\frac{1}{4m_0^2c^2}\cdot \big(\delta_{ij}+i\ep_{ijk}\vects{\sigma}_k\big)\op{\vect{k}}_i\left[\op{\vect{k}}_j,q\phi\right]
     =\underbrace{-\frac{1}{4m_0^2c^2}\cdot \vect{\op{k}}\left[\vect{\op{k}},q\phi\right]}_{\op{H}_\text{D1}}
      \underbrace{-\frac{i}{4m_0^2c^2}\sigv\Big(\vect{\op{k}}\times\left[\vect{\op{k}},q\phi\right]\Big)}_{\op{H}_\text{SP}}
   }
   Itt a második tag:
   \al{
    \op{H}_\text{SP}
     &=-\frac{i}{4m_0^2c^2}\vects{\sigma}_i\ep_{ijk}\op{\vect{k}}_j\left[\op{\vect{k}}_k,q\phi\right]
      =-\frac{i}{4m_0^2c^2}\vects{\sigma}_i\left(\left[\ep_{ijk}\op{\vect{k}}_j\op{\vect{k}}_k,q\phi\right]
       -\ep_{ijk}\left[\op{\vect{k}}_j,q\phi\right]\op{\vect{k}}_k\right)\\
     &=-\frac{i}{4m_0^2c^2}\vects{\sigma}_i\Bigg(\underbrace{\left[-\frac{\hbar q}{i}\vect{B}_i,q\phi\right]}_{=0}
       -\ep_{ijk}\left[\op{\vect{A}}_j,q\phi\right]\op{\vect{k}}_k\Bigg)
      =\frac{i}{4m_0^2c^2}
      \sigv\big(\left[\op{\vect{p}},q\phi\right]\times\op{\vect{k}}\big) \\
     &=\frac{\hbar}{4m_0^2c^2}
      \sigv\big(\left(\vects{\nabla}q\phi\right)\times\op{\vect{k}}\big)
      =-\frac{\hbar}{4m_0^2c^2}
      \sigv\big(q\vect{E}\times\op{\vect{k}}\big)
    }
   
   Az első taghoz pedig hozzávesszük azt is, hogy a bispinor normált:
   \al{
    1&=\intl{}{}\drh\Psi^+\Psi
      =\intl{}{}\drh\big(\chi^+\chi+\varphi^+\varphi\big)
      \approx\intl{}{}\drh\chi^+\chi+\frac{1}{4m_0^2c^2}\intl{}{}\drh\chi^+(\op{\vects{\sigma}}\op{\vect{k}})(\op{\vects{\sigma}}\op{\vect{k}})\chi\\
     &\approx\intl{}{}\drh\chi^+\chi+\frac{1}{4m_0^2c^2}\intl{}{}\drh\chi^+\op{\vect{k}}^2\chi
      =\intl{}{}\drh\chi^+\left(1+\frac{1}{4m_0^2c^2}\op{\vect{k}}^2\right)\chi.
   }
   Innen tehát a normált $\chi$ komponens ($1/c^2$ rendig): $\chi_S=\left(1+\frac{1}{8m_0^2c^2}\op{\vect{k}}^2\right)\chi$.
   Erre a $\chi_S$-re kellene valójában felírni a fenti egyenletet, ezt behelyettesítve:
   \al{
    E'\chi_S&\approx\left(1+\frac{1}{8m_0^2c^2}\op{\vect{k}}^2\right)\op{H}\left(1-\frac{1}{8m_0^2c^2}\op{\vect{k}}^2\right)\chi_S
     \approx\left(\op{H}+\left[\frac{1}{8m_0^2c^2}\op{\vect{k}}^2,\op{H}\right]\right)\chi_S\\
     &\approx\left(\op{H}+\frac{1}{8m_0^2c^2}\left[\op{\vect{k}}^2,q\phi\right]\right)\chi_S
   }
   Ennek a tagnak és $\op{H}_\text{D1}$-nek a korrekcióját együtt tekintve:
   \al{
    \op{H}_\text{D}
     &=-\frac{1}{4m_0^2c^2}\cdot \vect{\op{k}}\left[\vect{\op{k}},q\phi\right]+\frac{1}{8m_0^2c^2}\left[\op{\vect{k}}^2,q\phi\right]
      =\frac{q}{8m_0^2c^2}\cdot\left(\left[\op{\vect{k}}^2,\phi\right]-2\vect{\op{k}}\left[\vect{\op{k}},\phi\right]\right)\\
     &=\frac{q}{8m_0^2c^2}\cdot\left(\left[\op{\vect{k}},\phi\right]\op{\vect{k}}-\vect{\op{k}}\left[\vect{\op{k}},\phi\right]\right)
      =-\frac{\hbar^2q}{8m_0^2c^2}\cdot\big(\left(\vects{\nabla}\phi\right)\vects{\nabla}-\vects{\nabla}\left(\vects{\nabla}\phi\right)\big)\\
     &=\frac{\hbar^2q}{8m_0^2c^2}\cdot(\Delta\phi).
   }
   
   Összefoglalva tehát a normált nagykomponensre $1/c^2$-es pontosságú relativisztikus korrekciókkal felírt Pauli--Schrödinger-egyenlet az alábbi:
   \\[6pt]
   \fbox{
    \addtolength{\linewidth}{-10\fboxsep}%
    \addtolength{\linewidth}{-5\fboxrule}%
    \begin{minipage}{\linewidth}
     \vspace{-14pt}
     \aln{
      E'\chi_S=\big(\op{H}_\text{PS}+\op{H}_\text{K}+\op{H}_\text{SP}+\op{H}_\text{D}\big)\chi_S
     }
     \begin{subequations}
     \aln{
      \text{Pauli--Schrödinger-tag:}&&\op{H}_\text{PS}&=\left(\frac{\big(\op{\vect{p}}-q\vect{A}\big)^2}{2m_0}-\frac{\hbar q}{2m_0}\vects{\sigma}\vect{B}+q\phi\right)\\
      \text{Kinetikus energia tag:}&&\op{H}_\text{K}&=-\frac{1}{8m_0^3c^2}\op{\vect{k}}^4\\
      \text{Spin--pálya kölcsönhatás:}&&\op{H}_\text{SP}&=-\frac{\hbar}{4m_0^2c^2}
        \sigv\big(q\vect{E}\times\op{\vect{k}}\big)\\
      \text{Darwin-tag:}&&\op{H}_\text{D}&=\frac{\hbar^2q}{8m_0^2c^2}\cdot(\Delta\phi)
     }
     \end{subequations}
    \end{minipage}
   }
