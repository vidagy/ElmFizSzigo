\chapter{Hullámegyenletek}
 
 \section{Mechanika}
  
  \subsection{Hooke-törvény}
   
   Potenciálosnak akkor nevezünk egy anyagot, ha a közeg rugalmas tulajdonságai időben állandóak.
   Ekkor a felhalmozódott rugalmas energia a deformáció egyértelmű függvénye:
   \al{
    U=\intl{}{}\drh\varphi\big(\ep_{ij}(t,\rv),\rv\big).
   }
   Ennek megváltozása, ha a tartomány rögzített:
   \al{
    \partial_t U
     =\intl{}{}\drh\partial_t\varphi\big(\ep_{ij}(t,\rv),\rv\big)
     =\intl{}{}\drh\pder{\varphi\big(\ep_{ij}(t,\rv),\rv\big)}{\ep_{ij}}\partial_t\ep_{ij},
   }
   ennek integrandusa pedig egyenlő a deformációs teljesítménysűrűséget leíró taggal \eqaref{eq:08-kindiff} egyenletben:
   \eq{
    \sigma_{ij}=\pder{\varphi}{\ep_{ij}}.
   }
%    
   Innen láthatjuk, hogy potenciálos anyagnál egyértelmű kapcsolt van $\ep_{ij}$ és $\sigma_{ij}$ között.
   Kis deformációkra fejtsük sorba a feszültségtenzort mint a deformáció függvényét:
   \al{
    \sigma_{ij}=\sigma_{ij}(0)+\left.\pder{\sigma_{ij}}{\ep_{kl}}\right|_{\ep_{kl}=0}\ep_{kl}+\dots
   }
   Deformáció nélkül nincs feszültség az anyagban, illetve csak a lineáris rendet figyelembe véve:
   \al{
    &\sigma_{ij}=C_{ijkl}\ep_{kl}
    &C_{ijkl}=\left.\pder{\sigma_{ij}}{\ep_{kl}}\right|_{\ep_{kl}=0}.
   }
   Behelyettesítve:
   \aln{
    &\pder{\varphi}{\ep_{ij}}=C_{ijkl}\ep_{kl}
    &\Rightarrow
    &&&C_{ijkl}=\pder{^2\varphi}{\ep_{kl}\partial\ep_{ij}}\label{eq:09-cphi}\\
    &&&&&\varphi=\frac{1}{2}C_{ijkl}\ep_{ij}\ep_{kl}
          =\frac{1}{2}\sigma_{ij}\ep_{ij}
          =\frac{1}{2}C_{ijkl}\partial_j s_i\partial_l s_k\label{eq:09-phic}
   }
%    
   Láthatjuk \eqaref{eq:09-cphi} alapján hogy $C_{ijkl}$ szimmetrikus a $\{ij\}\leftrightarrow\{kl\}$ cserére.
   Mivel $\ep_{ij}$ is szimmetrikus ezért $C$ szimmetrikus az $i\leftrightarrow j$ és a $k\leftrightarrow l$ cserére is külön-külön.
   
   Ezen szimmetriák miatt a $3^4=81$ elemből csak 21 független. az $\{i,j\}$ és a $\{k,l\}$ párokban csak az azok különböznek, ahol $i\leq j$ és $k\leq l$ (6 féle lehetőség), és még az is szükséges, hogy $\{ij\} \leq \{kl\}$ lexikografikus rendezésben.
   Ez összesen $6+5+4+3+2+1=21$ lehetőség. 
   
   Egyéb szimmetriák esetén a független elemek száma tovább csökken, pl. köbös kristályban $C_{1111},\,C_{1122},\,C_{1212}$ független csak.
   Amorf anyagban 2 független állandó van csak: $\lambda$ és $\mu$.
   Ezekkel az energiasűrűség:
   \aln{
    &\varphi=\frac{\lambda}{2}(\ep_{ii})^2+\mu\ep_{ij}\ep_{ij}
    &\Rightarrow
    &&&\sigma_{ij}=\pder{\varphi}{\ep_{ij}}=\lambda\delta_{ij}\ep_{kk}+2\mu\ep_{ij}\nonumber\\
    &&&&&
       C_{ijkl}=\pder{\sigma_{ij}}{\ep_{kl}}
               =\pder{}{\ep_{kl}}\big(\lambda\delta_{ij}\ep_{kk}+2\mu\ep_{ij}\big)
               =\pder{}{\ep_{kl}}\big(\lambda\delta_{ij}\ep_{kk}+\mu(\ep_{ij}+\ep_{ji})\big)\nonumber\\
    &&&&&\phantom{C_{ijkl}}
               =\lambda\delta_{ij}\delta_{kl}+\mu(\delta_{ik}\delta_{jl}+\delta_{il}\delta_{jk})\label{eq:09-chooke}
   }
   
   Bontsuk szét a mátrixokat úgy, hogy leválasztjuk azok nyomát:
   \aln{
    \ep_{ij}&=\left(\ep_{ij}-\frac{1}{3}\delta_{ij}\ep_{kk}\right)+\frac{1}{3}\delta_{ij}\ep_{kk}\label{eq:09-essm}\\
    \sigma_{ij}
     &=\lambda\delta_{ij}\ep_{kk}+2\mu\left[\left(\ep_{ij}-\frac{1}{3}\delta_{ij}\ep_{kk}\right)+\frac{1}{3}\delta_{ij}\ep_{kk}\right]\nonumber\\
     &\qquad=2\mu\left(\ep_{ij}-\frac{1}{3}\delta_{ij}\ep_{kk}\right)+\underbrace{\left(\lambda+\frac{2}{3}\mu\right)}_{K}\delta_{ij}\ep_{kk}\label{eq:09-sessm}\\
     \varphi&=\frac{1}{2}\sigma_{ij}\ep_{ij}
      =\mu\left(\ep_{ij}-\frac{1}{3}\delta_{ij}\ep_{kk}\right)^2+\frac{K}{2}\ep_{kk}^2\label{eq:09-pessm}.
   }
   ahol $K$ a kompressziómodulus és $\mu$ a torziómodulus.
   Invertáljuk a $\sigma_{ij}(\ep_{kl})$ összefüggést.
   Ehhez
   \al{
    &\sigma_{ii}=3K\ep_{kk}
    &\Rightarrow
    &&\ep_{kk}=\frac{\sigma_{kk}}{3K},
   }
   melyet \eqaref{eq:09-sessm} egyenletbe helyettesítve 
   \al{
    \ep_{ij}
     &=\frac{1}{2\mu}\left(\sigma_{ij}-K\delta_{ij}\ep_{kk}\right)+\frac{1}{3}\delta_{ij}\ep_{kk}
      =\frac{1}{2\mu}\left(\sigma_{ij}-K\delta_{ij}\frac{\sigma_{kk}}       {3K}\right)+\frac{1}{3}\delta_{ij}\frac{\sigma_{kk}}{3K}\\
     &=\frac{1}{2\mu}\left(\sigma_{ij}-\frac{1}{3}\delta_{ij}\sigma_{kk}\right)+\frac{1}{9K}\delta_{ij}\sigma_{kk}
      =\frac{1}{2\mu}\sigma_{ij}+\left(\frac{1}{9K}-\frac{1}{6\mu}\right)\delta_{ij}\sigma_{kk}
   }
   
  \subsection{A rugalmas közeg állandói}
   
   \paragraph{Young-modulus, Poisson-szám}
    Tekintsünk egy olyan esetet, amikor húzóerő csak az $x$ irányban hat, vagyis:
    \al{
     \sigma_{ij}=
      \begin{pmatrix}
       p & 0 & 0 \\
       0 & 0 & 0 \\
       0 & 0 & 0 
      \end{pmatrix}.
    }
    A deformációs tenzor:
    \al{
     \ep_{ij}
      &=\frac{1}{2\mu}
        \begin{pmatrix}
         p & 0 & 0 \\
         0 & 0 & 0 \\
         0 & 0 & 0 
        \end{pmatrix}
        +\left(\frac{1}{9K}-\frac{1}{6\mu}\right)\mathbf{1}p
       =\frac{1}{2\mu}
        \begin{pmatrix}
         p & 0 & 0 \\
         0 & 0 & 0 \\
         0 & 0 & 0 
        \end{pmatrix}
        -\frac{\lambda}{2\mu(3\lambda+2\mu)}\mathbf{1}p\\
      &=\begin{pmatrix}
         p\frac{\lambda+\mu}{\mu (3\lambda+2\mu)} & 0 & 0 \\
         0 & -p\frac{\lambda}{2\mu(3\lambda+2\mu)} & 0 \\
         0 & 0 & -p\frac{\lambda}{2\mu(3\lambda+2\mu)}
        \end{pmatrix} 
    }
    A Young-modulus a nyomás irányú relatív megnyúlás és a nyomás között teremt kapcsolatot:
    \al{
     &\frac{\delta l}{l}E=p
     &\Rightarrow
     &&E=\frac{p}{\frac{\delta l}{l}}=\frac{p}{\ep_{11}}=\frac{\mu (3\lambda+2\mu)}{\lambda+\mu}.
    }
    A haránt-összehúzódást a Poisson-szám jellemzi, az adja meg a a relatív megnyúlások (ellentettjének) arányát:
    \al{
     v=-\frac{\ep_{22}}{\ep_{ii}}=\frac{\lambda}{2(\lambda+\mu)}.
    }
   \paragraph{Kompresszibilitás}
    
    Hassunk minden irányból azonos nyomással a testre, ekkor
    \al{
     &\sigma_{ij}=
      \begin{pmatrix}
       -p & 0 & 0 \\
       0 & -p & 0 \\
       0 & 0 & -p
      \end{pmatrix}
     &\Rightarrow
     &&\ep_{ij}=
      \begin{pmatrix}
       -\frac{p}{3K} & 0 & 0 \\
       0 & -\frac{p}{3K} & 0 \\
       0 & 0 & -\frac{p}{3K}
      \end{pmatrix}.
    }
    A relatív térfogatváltozás $\ep_{ii}=\frac{\delta V}{V}$, ahonnan
    \al{
     p=-K\frac{\delta V}{V}.
    }
   
   \paragraph{Nyírási modulus}
    
    Tiszta nyírást szeretnénk létrehozni, ezért az $xy$ síkra $y$ irányú erővel hatunk.
   Az elmozdulás ekkor arányos a $z$ koordinátával:
    \al{
     s_i=\begin{pmatrix}
          0\\Cx_3\\0
         \end{pmatrix},
    }
    ahol $C$ a nyírás szögének tangense, kis szögekre pedig maga a szög.
   A deformációtenzor:
    \al{
     &\ep_{ij}
      =\frac{1}{2}\big(\partial_i s_k+\partial_k s_i\big)
      =\frac{1}{2}\big(\delta_{i3}\delta_{k2}+\delta_{i2}\delta_{k3}\big)
      =\frac{C}{2}\begin{pmatrix}
        0 & 0 & 0 \\
        0 & 0 & 1 \\
        0 & 1 & 0
       \end{pmatrix}
     &\Rightarrow
     &&\sigma_{ij}
      =\mu C\begin{pmatrix}
        0 & 0 & 0 \\
        0 & 0 & 1 \\
        0 & 1 & 0
       \end{pmatrix}
    }
    Vagyis $\mu C=\frac{F}{A}$.
   Itt $C$ az $y$ irányú megnyúlásnak és a $z$ irányú hossznak az aránya.
   Innen leolvashatjuk, hogy $\mu$ a nyírási modulus. (A jele általános esetben $G$.)
    
  \subsection{Hullámegyenlet}
   
   A rugalmas anyag mozgását meghatározó egyenletek:
   \aln{
    &&\text{Mozgásegyenlet:}&&&\rho\dd_t v_i=f_i+\partial_j\sigma_{ij}&\label{eq:09-me}\\
    &&\text{Hooke-törvény:}&&&\sigma_{ij}=C_{ijkl}\ep_{kl}&\label{eq:09-ht}\\
    &&\text{Kontinuitási egyenlet:}&&&\partial_t\rho+\partial_i(\rho v_i)=0&\label{eq:09-ke}
   }
   Az egyenletrendszer megoldásait keressük homogén esetben.
   Helyettesítsük \eqaref{eq:09-ht} egyenletet \eqaref{eq:09-me} egyenletbe:
   \al{
    \rho\dd_t v_i
     =f_i+\partial_jC_{ijkl}\ep_{kl}
     =f_i+C_{ijkl}\frac{1}{2}\partial_j\big(\partial_k s_l+\partial_l s_k\big)
     =f_i+C_{ijkl}\partial_j\partial_k s_l,
   }
   illetve $\dd_t v_i=\partial_t v_i+v_j\partial_j v_i$, ahol a második tagot elhagyjuk. Áttérve itt is az elmozdulásokra, $\partial_tv_i=\partial_t^2s_i$:
   \aln{
    \rho\partial_t^2s_i-C_{ijkl}\partial_j\partial_k s_l=f_i\label{eq:09-rugeq}
   }
   \paragraph{Statikus deformáció}
    
    Ekkor az elmozdulásmező időtől független, így \eqaref{eq:09-rugeq} egyenlet:
    \al{
     C_{ijkl}\partial_j\partial_k s_l=-f_i.
    }
    Homogén amorf anyagra, ahol a Hooke-tenzor \eqaref{eq:09-chooke} alakú:
    \al{
     -f_i
      &=\big[\lambda\delta_{ij}\delta_{kl}+\mu(\delta_{ik}\delta_{jl}+\delta_{il}\delta_{jk})\big]\partial_j\partial_k s_l
       =\lambda\partial_i\partial_k s_k+\mu(\partial_i\partial_k s_k+\partial_k\partial_k s_i)\\
       -\vect{f}&=(\lambda+\mu)\grad(\divo{\vect{s}})+\mu\Delta\vect{s}.
    }
    
   \paragraph{Haladóhullám megoldások, erőmentes eset}
    
    Keressük \eqaref{eq:09-rugeq} egyenlet erőket nem tartalmazó alakjának megoldásait haladó hullám alakban:
    \al{
     s_i=S_i\Psi(\nv\rv-vt),
    }
    ahol $\vect{S}$ és $\vect{n}$ egy-egy (helytől és időtől független) vektor.
   Behelyettesítve:
    \al{
     \rho\partial_t^2S_i\Psi(n_pr_p-vt)&=C_{ijkl}\partial_j\partial_k S_l\Psi''\\
     \rho v^2 S_i \Psi''(n_pr_p-vt)&=C_{ijkl}n_j n_k S_l\Psi''\\
     0&=\big(\rho v^2 S_i -C_{ijkl}n_j n_k S_l\big)\Psi''.
    }
    A zárójelben lévő kifejezés eltűnését követeljük meg:
    \al{
     0=\big(\rho v^2 \delta_{il} -C_{ijkl}n_j n_k\big)S_l
    }
    $S_l$-re akkor nem triviális az egyenlet megoldása, ha a mátrix determinánsa eltűnik:
    \al{
    0=
      \begin{vmatrix}
       \rho v^2 -C_{1jk1}n_jn_k & -C_{1jk2}n_jn_k & -C_{1jk3}n_jn_k \\
       -C_{2jk1}n_jn_k & \rho v^2-C_{2jk2}n_jn_k & -C_{2jk3}n_jn_k \\
       -C_{3jk1}n_jn_k & -C_{3jk2}n_jn_k & \rho v^2-C_{3jk3}n_jn_k
      \end{vmatrix}.
    }
    Ez $v^2$-re egy harmadfokú egyenlet, aminek legfeljebb 3 különböző megoldása lehet. Így $v$-re a lehetséges értékek: $\pm v_1$, $\pm v_2$, $\pm v_3$.
   Az ellentétes előjelek csak az ellentétes irányokba terjedő hullámokra vonatkoznak. 
    
    Egy adott terjedési irányra ($\vect{n}$) tehát háromféle $v$-t kapunk, amelyekhez tartozik három rezgési irány ($\vect{S}$) is. 
    
   \paragraph{Haladóhullámok izotrop, amorf anyagokban}
    
    \Eqaref{eq:09-chooke} egyenletet használva:
    \al{
     0
      &=\rho v^2 S_i -\Big[\lambda\delta_{ij}\delta_{kl}+\mu(\delta_{ik}\delta_{jl}+\delta_{il}\delta_{jk})\Big]n_j n_k S_l
       =\rho v^2 S_i -\lambda n_i n_k S_k-\mu(n_i n_k S_k+\underbrace{n_k n_k}_{1} S_i)\\
      &=\big(\rho v^2-\mu \big)S_i-(\lambda+\mu) n_i n_k S_k
    }
    \al{
     \frac{\big(\rho v^2-\mu \big)}{\lambda+\mu}S_i&= n_i n_k S_k\\
     \frac{\big(\rho v^2-\mu \big)}{\lambda+\mu}\vect{S}&= (\vect{n}\circ\vect{n})\vect{S}
    }
    Az $\vect{n}\circ\vect{n}$ az $\vect{n}$ irányra való projekció.
   Ennek sajátértékei a 0 (kétszer) és az 1 (egyszer).
   Ezek alapján:
    \al{
     &\frac{\big(\rho v^2-\mu \big)}{\lambda+\mu}=0
     &\Rightarrow
     &&v_{1,2}=\pm\sqrt{\frac{\mu}{\rho}}
     &&\text{vagy}
     &&\frac{\big(\rho v^2-\mu \big)}{\lambda+\mu}=1
     &&\Rightarrow
     &&v_{3,4,5,6}=\pm\sqrt{\frac{2\mu+\lambda}{\rho}}.
    }
    Az első esetben $\vect{S}_{1,2}\perp\vect{n}$, hiszen a projektor sajátértéke nulla, míg a második esetben $\vect{S}_{3,4,5,6}\parallel\vect{n}$. 
    
 \section{Elektrodinamika}
  
  \subsection{A vákuumbeli időfüggő Maxwell-egyenletek megoldása}\label{ss:9-retpot}
   
   A Maxwell-egyenletek vákuumban:
   \al{
   & &\divo{\vect{E}}=\frac{1}{\ep_0}\rho &&
    &\divo\vect{B}=0& \\
    & &\rot{\vect{E}}=-\partial_t\vect{B} &&
    &\rot{\vect{B}}=\mu_0\vect{J}+\frac{1}{c^2}\partial_t\vect{E}, &
   }
   Keressük a megoldásokat a potenciálokon keresztül: $\vect{E}=-\partial_t\vect{A}-\grad\vect{\phi}$, és $\vect{B}=\rot{\vect{A}}$.
   Lorentz-mértéket, $\divo\vect{A}+\frac{1}{c^2}\partial_t\phi=0$, használva:
   \al{
     & \left(\Delta-\frac{1}{c^2}\partial_t^2\right)\phi=-\frac{1}{\ep_0}\rho
     & \left(\Delta-\frac{1}{c^2}\partial_t^2\right)\vect{A}=-\mu_0\vect{J}
    }
   Ezek az egyenletek mind ugyanolyan alakúak: ez a inhomogén d'Alambert-egyenlet.
   Először kezdjünk a homogén megoldásával. 
   
   \paragraph{Homogén megoldás}
    
    Tehát keressük a $\square\Psi=0$ egyenlet megoldását.
   Fourier-transzformáljuk az egyenletet térben és időben is:
    \al{
     &\Psi(t,\rv)
      =\intl{-\infty}{\infty}\frac{\dd \omega}{2\pi}\,\intl{}{}\frac{\dd^3 \kv}{(2\pi)^3}\,e^{-i\omega t+i\kv\rv}\Psi(\omega,\kv)
     &\Psi(\omega,\kv)
      =\intl{-\infty}{\infty}\dd t\,\intl{}{}\drh e^{i\omega t-i\kv\rv}\Psi(t,\rv),
    }
    így 
    \al{
     \left(\frac{\omega^2}{c^2}-k^2\right)\Psi(\omega,\kv)=0
    }
    Az egyenlet megoldása:
    \al{
     \Psi(\omega,\kv)=a(\kv)2\pi\delta(\omega-kc)+b(\kv)2\pi\delta(\omega+kc),
    }
    ahol $a(\kv)$ és $b(\kv)$ tetszőleges függvények.
   Az inverz Fourier-transzformációt elvégezzük:
    \al{
     \Psi(t,\rv)
      &=\intl{-\infty}{\infty}\frac{\dd \omega}{2\pi}\,\intl{}{}\frac{\dd^3 \kv}{(2\pi)^3}\,e^{-i\omega t+i\kv\rv}\big(a(\kv)2\pi\delta(\omega-kc)+b(\kv)2\pi\delta(\omega+kc)\big)\\
      &=\intl{}{}\frac{\dd^3 \kv}{(2\pi)^3}\,\left(a(\kv)e^{-ikc t+i\kv\rv}+b(\kv)e^{ikc t+i\kv\rv}\right)
    }
    $\Psi(t,\rv)$ valósságából következik, hogy $a^*(-\kv)=b(\kv)$, így $a(\kv)$ lehet komplex és így nincs szükségünk $b(\kv)$-re.
   Ezzel
    \al{
     \Psi(t,\rv)
      &=2\Re\left[\intl{}{}\frac{\dd^3 \kv}{(2\pi)^3}\,a(\kv)e^{-i\omega(\kv) t+i\kv\rv}\right] & \omega(\kv)&=ck
    }
   \paragraph{Elektrodinamikai hullámok végtelen forrásmentes térben}
    
    A potenciálokra vonatkozó egyenletek Coulomb-mértékben ($\divo{\Av}=0$):
    \al{
     &\Delta\phi=0 & \Rightarrow &&& \phi=0\\
     &\square\vect{A}=0 & \Rightarrow &&& \vect{A}(t,\vect{rv})=\intl{}{}\frac{\dd^3 \kv}{(2\pi)^3}\,\vect{A}_0(\kv)e^{-i\omega(\kv) t+i\kv\rv}
    }
    A mértékválasztás miatt $\kv\cdot\Av_0(\kv)=0$.
   Az egyes monokromatikus komponensek ($\kv$ fix):
    \al{
     \vect{E}&=-\partial_t\vect{A}=i\omega\vect{A}_0(\kv)e^{-i\omega(\kv) t+i\kv\rv}=\vect{E}_0 e^{-i\omega(\kv) t+i\kv\rv}&\vect{E}_0&=i\omega\vect{A}_0(\kv)\\
     \vect{B}&=\rot\vect{A}=i\kv\times\vect{A}_0(\kv)e^{-i\omega(\kv)t+i\kv\rv}=\vect{B}_0 e^{-i\omega(\kv) t+i\kv\rv}&\vect{B}_0&=i\kv\times\vect{A}_0(\kv)=\frac{1}{c}\frac{\kv}{k}\times\vect{E}_0.
    }
    A hullámban tárolt energiasűrűség:
    \al{
     w=\frac{\ep_0}{2}\vect{E}^2+\frac{1}{2\mu_0}\vect{B}^2
      =\frac{\ep_0}{2}\vect{E}^2+\frac{1}{2\mu_0}\frac{1}{c^2}\left(\frac{\kv}{k}\times\vect{E}\right)^2
      =\ep_0\vect{E}^2.
    }
    Egy periódusra kiátlagolva az időfüggést:
    \al{
     \mv{w}=\frac{\ep_0}{2}\Ev_0^2.
    }
    A Poynting-vektor:
    \al{
     \vect{S}
      =\frac{1}{\mu_0}\vect{E}\times\vect{B}
      =\frac{1}{c\mu_0}\vect{E}\times\left(\frac{\kv}{k}\times\vect{E}\right)
      =\frac{1}{c\mu_0}\vect{E}^2\cdot \frac{\kv}{k}
      =cw\frac{\kv}{k},
    }
    tehát az energia áramsűrűség mindig a $\kv$ vektor irányába mutat. 
    
   \paragraph{Teljes időfüggés, források}
    
    Pontforrásra oldjuk meg az egyenletet végtelen térben, így a Green-függvényt kapjuk, ebből tetszőleges forrás tagra előállíthatóak a megoldások.
    \al{
     &\square G(t,\rv;t',\rv')=-\delta(t-t')\delta(\rv-\rv')
     &\Rightarrow
     &&\Psi(t,\rv)=\intl{-\infty}{\infty}\dd t'\,\intl{}{}\dd^3\rv'\, G(t,\rv;t',\rv')f(t',\rv').
    }
    
    A jobb oldal csak $t$ és $t'$, illetve $\rv$ és $\rv'$ különbségétől függ, így ez a Green-függvényre is igaz, $G(t-t',\rv-\rv';0,0)$.
    
    Fourier-transzformáljuk a differenciálegyenletet időben:
    \al{
     (\Delta+k^2)G(\omega,\rv)=-\delta(\rv).
    }
    Mivel a jobb oldal forgásinvariáns, illetve a határfeltételek is azok, azért a $G(t,\rv)$ csak a radiális koordinátától függ.
   Gömbi koordináta-rendszerben ($r\ne0$):
    \al{
     &\frac{1}{r}\der{^2}{r^2}\big(rG(\omega,r)\big)+k^2G(\omega,r)=0
     &\Rightarrow
     &&G_{R/A}(\omega,r)=C\frac{e^{\pm i k r}}{r}.
    }
    $k=0$-ra a Laplace-egyenlet Green-függvényét kell kapjuk, ahonnan következik, hogy $C=\frac{1}{4\pi}$.
   Valós térben a Green-függvény:
    \al{
     G_{R/A}(t,r)
      =\frac{1}{4\pi r}\intl{-\infty}{\infty}\frac{\dd\omega}{2\pi}\,e^{-i\omega t}e^{\pm i\omega/c r}
      =\frac{1}{4\pi r}\delta\left(t\mp \frac{r}{c}\right),
    }
    ami visszaírva Descartes-koordinátarendszerbe
    \al{
     G_{R/A}(t-t',\rv-\rv')=\frac{1}{4\pi \abs{\rv-\rv'}}\delta\left((t-t')\mp \frac{\abs{\rv-\rv'}}{c}\right).
    }

   \paragraph{Retardált potenciálok}
    
    Legyen a forrástag egy időben változó ponttöltés, ami végig az origóban van, $f(t,\rv)=\frac{1}{\ep_0}q(t)\delta(\rv)$.
   Ekkor a két megoldás:
    \al{
     \Psi_R(t,\rv)
     &=\intl{-\infty}{\infty}\dd t'\,\intl{}{}\dd^3\rv'\, G_R(t,\rv;t',\rv')f(t',\rv')\\
     &=\intl{-\infty}{\infty}\dd t'\,\intl{}{}\dd^3\rv'\, \frac{1}{4\pi\ep_0 \abs{\rv-\rv'}}\delta\left((t-t')- \frac{\abs{\rv-\rv'}}{c}\right)q(t')\delta(\rv')
      =\frac{1}{4\pi \ep_0\abs{\rv}}q\left(t-\frac{\abs{\rv}}{c}\right)\\
     \Psi_A(t,\rv)
      &=\intl{-\infty}{\infty}\dd t'\,\intl{}{}\dd^3\rv'\, \frac{1}{4\pi\ep_0 \abs{\rv-\rv'}}\delta\left((t-t')+ \frac{\abs{\rv-\rv'}}{c}\right)q(t')\delta(\rv')
      =\frac{1}{4\pi\ep_0 \abs{\rv}}q\left(t+\frac{\abs{\rv}}{c}\right)
    }
    Ha a ponttöltés töltése nem időfüggő, akkor mindkét egyenlet visszaadja a statika megoldását.
   A retardált esetben a megoldást a $(t,\rv)$ helyen a forrás korábbi értéke határozza meg.
   Ennek fizikai jelentése tehát az, hogy ha $t_0$-ban bekapcsolok egy forrást, akkor ez után a megoldás a retardált megoldással készíthető el:
    \al{
     \Psi(t,\rv)=\underbrace{\Psi(t_0,\rv)}_\text{kezd. felt.}+\Psi_R(t,\rv).
    }
    
    Az avanzsált esetben a megoldás a $(t,\rv)$ helyen a forrás jövőbeli értékétől függ.
   Ha egy forrás $t_v$ időpontig működött, és tudom, hogy ekkor mi volt a megoldás, akkor egy korábbi $t$ időpontban a megoldás:
    \al{
     \Psi(t,\rv)=\Psi(t_v,\rv)+\Psi_A(t,\rv).
    } 
    
    Az elektrodinamikai esetben a retardált potenciálok:
    \aln{
     \phi_R(t,\rv)
      &=\frac{1}{4\pi\ep_0}\intl{}{}\dd^3\rv'\,\frac{\rho\left(t-\frac{\abs{\rv-\rv'}}{c},\rv'\right)}{\abs{\rv-\rv'}}\label{eq:09-retpotphi}\\
     \vect{A}_R(t,\rv)
      &=\frac{\mu_0}{4\pi}\intl{}{}\dd^3\rv'\,\frac{\vect{J}\left(t-\frac{\abs{\rv-\rv'}}{c},\rv'\right)}{\abs{\rv-\rv'}}\label{eq:09-retpotA}.
    }
    
  \subsection{Fázis- és csoportsebesség}
   
   \paragraph{Fázissebesség} Tekintsünk csak egy módust.
   Ennek azonos fázisban lévő pontjai:
   \al{
    &e^{-i\omega(\kv)t+i\kv\rv}=e^{i\kv\rv_0}
    &\Rightarrow
    && \kv(\rv-\rv_0)=\omega(\kv)t+2\pi n\quad n\in\mathbb{Z}.
   }
   Az egyenlet megoldását ($\rv$-re) keressük az alábbi alakban: $\rv=\rv_0+(v_\text{f} t + \lambda n)\frac{\kv}{k}+\beta\kv_{\perp}$, ahol a zárójelben szereplő szám  és $\beta$ egy-egy (furcsán megválasztott) konstans.
   Behelyettesítve:
   \al{
    &v_\text{f}(\kv)=\frac{\omega(\kv)}{k}
    &\lambda=\frac{2\pi}{k}.
   } 
   $\kv$-re merőlegesen vannak az azonos fázisú pontok.
   Ezek a felületek egymást $\lambda$ távolságra követik.
   A felületek pedig $\kv$ irányában haladnak $v_\text{f}$ sebességgel. 
   
   \paragraph{Csoportsebesség} Vegyünk több komponenst, melyek egy irányban terjednek, illetve tegyük fel, hogy $\omega(\kv)$ lassan változik egy átlagos $\kv_0$ körül, ekkor a megoldás:
   \al{
    \Psi(t,\rv)=\intl{}{}\frac{\dd k}{2\pi}\,a(k)e^{-i\omega(k)t+ikx}.
   }
   Fejtsük sorba $\omega$-t: $\omega(k)=\omega(k_0)+\left.\der{\omega}{k}\right|_{k_0}(k-k_0)\dots\approx\omega0+v_\text{cs}(k-k_0)$, így
   \al{
    \Psi(t,\rv)
    &=\intl{}{}\frac{\dd k}{2\pi}\,a(k)e^{-i\big(\omega_0+v_\text{cs}(k-k_0)\big)t+ikx}
     =e^{-i\big(\omega_0-v_\text{cs}k_0\big)t}\intl{}{}\frac{\dd k}{2\pi}\,a(k)e^{-iv_\text{cs}kt+ikx}\\
    &=e^{-i\omega_0\left(1-\frac{v_\text{cs}}{v_\text{f}}\right)t}\Psi(0,x-v_\text{cs}t).
   }
   A hullám alakja megmarad, csak egy fázisfaktorral szorzódik, a hullám viszont halad előre $v_\text{cs}$ sebességgel.
   A fázisfaktor a hullám burkolója.
   
  \subsection{Hullámvezetők}
   
   A téglalap keresztmetszetű fémfalú hullámvezetőben az elektromágneses tereket a forrásmentes vákuumban felírt Maxwell-egyenletek szabják meg.
   A vezető fala ideális fém, azaz a határfeltételek: $\vect{E}_\parallel=0$, $\vect{B}_\perp=0$.
   
   A terek megoldásai a falak irányában állóhullámok a terjedés irányában pedig síkhullám.
   Ezzel $k_x=\frac{n\pi}{a}$ és  $k_y=\frac{m\pi}{b}$, ahol $q$ és $b$ oldaltávolságok, $n$ és $m$ pedig egész számok.
   Itt $\omega(\kv)=kc$.
   A fázis- és a csoportsebesség, az $\omega$ hullámszámú $k_z$-vel terjedő hullámokra:
   \al{
    v_\text{f}
     &=\frac{\omega}{k_z}
      =\frac{c\sqrt{ \left(\frac{n\pi}{a}\right)^2+\left(\frac{m\pi}{b}\right)^2+k_z^2}}{k_z}
      =c\sqrt{\left(\frac{n\pi}{a k_z}\right)^2+\left(\frac{m\pi}{b k_z}\right)^2+1}>c\\
    v_\text{cs}
     &=\der{\omega}{k_z}
      =\frac{c}{\sqrt{\left(\frac{n\pi}{a k_z}\right)^2+\left(\frac{m\pi}{b k_z}\right)^2+1}}<c.
   }
   
 \section{Kvantummechanika}
  
  Hullámegyenletek, hullámfüggvények, jelentésük: lásd \ref{1tetel}--\ref{2tetel}. tétel kvantummechanika része, illetve a mérlegegyenletek \aref{ss:A08-kv}. fejezetben.
