\chapter{Transzport, fluktu\'aci\'ok, id\H{o}f\"ugg\H{o} korrel\'aci\'ok, kereszteffektusok, Onsager-rel\'aciok} 
 
 \section{Időfüggő egyensúlyi korrelációs függvények}
  
  Legyen két időfüggő fizikai mennyiség $X(t)$ és $Y(t)$.
   Ezek átlaguktól való eltérése: $x(t)=X(t)-\mv{X}(t)$ és $y(t)=Y(t)-\mv{Y}(t)$.
   Az időfüggő korrelációs függvény:
  \al{
   C_{xy}(t)=\mv{x(t')y(t'+t)}_{t'}
    =\lim_{T\to\infty}\frac{1}{T}\intl{0}{T}\dd t'\, x(t')y(t+t').
  }
  Ha a rendszer stacioner és ergodikus, akkor az így definiált időátlag, $\mv{\,\cdot\,}_{t'}$, megegyezik a sokaságátlaggal, $\mv{\,\cdot\,}$.
   A korrelációs függvény tulajdonságai:
  \begin{itemize}
   \item 
    $t\to\infty$-re a korrelációk függetlenek lesznek egymástól:
    \al{
     \lim_{t\to\infty}C_{xy}(t)
      &=\lim_{t\to\infty}\mv{x(t')y(t'+t)}_{t'}
       =\lim_{t\to\infty}\mv{x(0)y(t)}
       =\mv{x(0)}\lim_{t\to\infty}\mv{y(t)}
       =0
    }
    
   \item
    $t=0$-ra az időfüggő korrelációs függvény megegyezik a keresztkorrelációs együtthatókkal:
    \al{
     C_{xy}(t=0)
      &=\mv{x(t')y(t')}_{t'}
       =\mv{x(0)y(0)}
    }
    
   \item 
    \al{
     C_{xy}(t)
      &=\mv{x(t')y(t'+t)}_{t'}
       =\mv{x(t'-t)y(t')}_{t'}
       =\mv{y(t')x(t'+(-t))}_{t'}
       =C_{yx}(-t)
    }
   \item 
    A korrelációs függvény szimmetrikus: $C_{xy}(t)=C_{yx}(t)$ (ha a rendszer időtükrözésre invariáns).
   Ehhez fejtsük ki a fenti sokaságátlagot:
    \al{
     C_{xy}(t)
      &=\intl{}{}\dd x\,\intl{}{}\dd y\, xy\cdot P(x) P(y,t|x) ,
    }
    ahol $P(x)$ annak a valószínűsége, hogy $x$ fluktuációja $x$, illetve $P(t,y|0,x) $ annak a valószínűsége, hogy $y$ a fluktuáció értéke $t$-ben, ha $t=0$-ban $x$ értéke $x$ volt.
   A részletes egyensúly következménye, hogy $P(x) P(t,y|0,x)=P(y) P(t,x|0,y)$.
   Ezt behelyettesítve:
    \al{
     C_{xy}(t)
      &=\intl{}{}\dd x\,\intl{}{}\dd y\, xy\cdot P(y) P(t,x|0,y)
       =C_{yx}(t).
    }
   
   \item 
    Az előző kettőből $C_{yx}(t)=C_{xy}(-t)$, vagyis a korrelációs függvény időben is szimmetrikus. 
  \end{itemize}
  
  A korrelációs függvények kiterjeszthetőek a kvantummechanikai esetre is.
   Itt azonban figyelembe kell venni, hogy az operátorok nem felcserélhetőek.
   Ehhez szimmetrizálni/antiszimmetrizálni kell:
  \al{
   C^{F/B}_{\opx\opy}=\frac{1}{2}\mv{\mv{\opx(t')\opy(t'+t)\pm\opy(t'+t)\opx(t')}}_{t'}.
  } 
  A belső várható érték a kvantummechanikai várható értéket jelenti, a külső pedig az időbelit.
   Az előbb bizonyított tulajdonságok továbbra is fennállnak. 
  
 \section{Lineáris transzport}
  
  Egyensúlyban a fluktuációk várható értéke nulla, azonban az értékük az egyes időpillanatokban nem nulla.
   Az fluktuáció oka a statisztikus jelleg, az, hogy az eloszlásnak van valamilyen szélessége.
   A fluktuáció véges értékét úgy is el tudjuk érni, hogy a rendszerre valamilyen külső teret kapcsolunk, majd ezt a teret kikapcsoljuk.
   Amíg a teret bekapcsoltuk, a rendszerbe beleavatkoztunk, így az nem-egyensúlyi állapotban van.
  
  Bárhogy is jött létre az átlagértéktől való eltérés, a várható értéke $t\to\infty$-re újra lecseng:
  \al{
   \lim_{t\to\infty}\mv{\xv(t)|\xv(t=0)=\xv_0}=0.
  }
  
  A nem egyensúlyi sokaságok lényege, hogy a valamilyen módon előállt $x_0$ makroállapotnak megfelelő sokaságot állítjuk elő, azzal a feltételes eloszlással, hogy $x_0$ fennáll.
   A rendszerben a dinamika változatlan, egyedül az eloszlás különbözik. 
  
  Térítsünk ki egy mennyiséget az egyensúlyi értékéből.
   Az átlaghoz való visszatérési sebessége ekkor arányos lesz a kitéréssel, vagyis:
  \al{
   \mv{\dot x_i}_{\xv_0}=-\suml{k}{}\lambda_{ik}\mv{x_k}_{\xv_0}.
  }
  Definiáljunk a termodinamikai erőt: $y_i=-\pder{S}{x_i}$.
   Az entrópia sorfejtése: $S=S_0-\frac{1}{2}\suml{ij}{}x_ig_{ij}x_j$, ahol $g_{ij}$ szimmetrikus.
   Innen $y_i=\suml{j}{}g_{ij}x_j$, ami megfordítva $x_i=\suml{j}{}\big[g^{-1}\big]_{ij}y_j$.
   Ezt felhasználva:
  \al{
   \mv{\dot x_i}_{\xv_0}
    =-\suml{k}{}\lambda_{ik}\suml{j}{}\big[g^{-1}\big]_{kj}\mv{y_j}_{\xv_0}
    =-\suml{j}{}\underbrace{\suml{k}{}\lambda_{ik}\big[g^{-1}\big]_{kj}}_{L_{ij}}\mv{y_j}_{\xv_0}
    =-\suml{j}{}L_{ij}\mv{y_j}_{\xv_0}
  }
  Ez a transzportegyenlet.
   A termodinamikai erőre másik definíciót is adhatunk.
   Az előző alapján felírhatjuk, hogy $\dd S=-\suml{i}{}y_i\dd x_i$.
   Innen az entrópiaprodukció: $\dot S=-\suml{i}{}y_i\dot x_i$, vagyis $y_i=-\pder{\dot S}{\dot x_i}$. 
  
  \subsection{Onsager-hipotézis}
   
   Az Onsager-hipotézis lényege, hogy az egyensúlyi fluktuációk és a nem egyensúlyi kitérítés között nincs különbség, miközben a rendszer relaxál mindegy, hogy mi volt az eredeti kitérés oka.
   Ez azt jelenti, hogy a transzportegyenlet a korrelációs függvényekre felírva is igaz:
   \al{
    \der{}{t}\mv{x_i(t),x_j(0)}=-\suml{k}{}L_{ik}\mv{y_k(t),x_j(0)}.
   }
   A hipotézis következményének belátásához felhasználjuk a korrelációs függvény szimmetrikusságát ($C_{ij}(t)=C_{ji}(t)$), illetve az alábbi azonosságot: $\mv{y_k x_j}=\kB\delta_{kj}$.
   Ennek belátásához nézzük a 
   \al{
    0
     &=N\intl{}{}\dd^{n}x\,\pder{}{x_j}\left(x_i e^{-\frac{1}{2\kB}\suml{lm}{}x_l g_{lm}x_m}\right)\\
     &=N\intl{}{}\dd^{n}x\,\underbrace{\pder{x_i}{x_j}}_{\delta_{ij}} e^{-\frac{1}{2\kB}\suml{lm}{}x_l g_{lm}x_m}+N\intl{}{}\dd^{n}x\,x_i\pder{}{x_j} e^{-\frac{1}{2\kB}\suml{lm}{}x_l g_{lm}x_m}\\
     &=\delta_{ij}+N\intl{}{}\dd^{n}x\,x_i \left(-\frac{1}{\kB}\suml{l}{} g_{jl}x_l\right)e^{-\frac{1}{2\kB}\suml{lm}{}x_l g_{lm}x_m}
      =\delta_{ij}-\frac{1}{\kB}\mv{x_i y_j}.
   }
   Az első egyenlőség triviálisan igaz, hiszen ha $x_j$ szerint kiintegrálunk, akkor az kiintegrált rész eltűnik a határokon. 
   
   Tehát használjuk fel a korrelációs függvény szimmetrikusságát:
   \al{
    &\der{}{t}\mv{x_i(t),x_j(0)}=\der{}{t}\mv{x_j(t),x_i(0)}
    &\Rightarrow&
    &-\suml{k}{}L_{ik}\mv{y_k(t),x_j(0)}=-\suml{k}{}L_{jk}\mv{y_k(t),x_i(0)},
   }
   majd tekintsük a relációt $t=0$-ban.
   Ekkor felhasználva a másik azonosságot:
   \al{
    L_{ij}=L_{ji}
   }
   adódik, ami az Onsager-féle reciprocitási törvény.
   Mivel a korrelációs törvény szimmetrikusságának követelménye az volt, hogy a mikroszkopikus folyamatok reverzibilisek, ezért ezt itt is meg kell követelni.
   Az időtükrözési invariancia lehetséges, hogy nem teljesül pl. mágneses tér jelenlétében, vannak olyan mennyiségek, amelyek előjelet váltanak az időtükrözésre.
   Ekkor $C_{xy}(t)=I_x I_y C_{yx}(t)$, ahol $I$ az időtökrözéshez tartozó sajátérték, és így $L_{xy}=I_x I_y L_{yx}$.
   
  \subsection{Példa: hő- és elektromos vezetés}
   
   \begin{description}
    \item[Elektromos vezetés:] Az entrópiaprodukció: $\der{S}{t}=\frac{P}{T}=\frac{EJ}{T}$, ahol $E$ az elektromos térerősség, $J$ az elektromos áramsűrűség és $T$ a hőmérséklet.
   A fenti jelöléssel $\dot x_e=J$.
   Az ehhez tartozó termodinamikai erő: $y_e=-\pder{\dot S}{\dot x_e}=-\frac{E}{T}$.
    
    Ezek alapján a transzportegyenlet: 
    \al{
     J=-L_{ee}\left(-\frac{E}{T}\right).
    }
    Ez az Ohm-törvény, ahol $L_{ee}=\sigma T$.
    
    \item[Hővezetés: ] Az entrópiaprodukció itt arányos a hőárammal és a hőmérséklet inverzének gradiensével: $\dot S=w\nabla\frac{1}{T}$.
   Innen $\dot x_{q}=w$, vagyis $y_q=-\pder{\dot S}{\dot x_{q}}=-\grad\frac{1}{T}=\frac{1}{T^2}\nabla T$.
    
    Itt a transzportegyenlet:
    \al{
     w=-L_{qq}\frac{1}{T^2}\nabla T,
    }
    ami a hővezetési egyenletnek felel meg $L_{qq}=\lambda T^2$.
    
    \item[Kereszteffektusok: ] Az elektromos áram létrehozhat hőáramot, illetve fordítva.
   Ezeket is figyelembe véve a transzportegyenletek:
    \al{
     J&=-L_{ee}\left(-\frac{E}{T}\right)-L_{eq}\frac{1}{T^2}\nabla T
      &
     E&=\frac{1}{\sigma}J+\eta\nabla T\\
     w&=-L_{qe}\left(-\frac{E}{T}\right)-L_{qq}\frac{1}{T^2}\nabla T
      &
     w&=\pi J-\lambda\nabla T
    }
    A jobb oldalon a korábbi ``kísérleti'' törvények állnak, $\pi$ a Peltier-együttható, míg $\eta$ a Seebeck-együttható.
   A kísérletek alapján a Thomson-összefüggés: $\pi=\eta T$.
   Ha $\grad T=0$, akkor $\pi=\frac{w}{J}=\frac{L_{qe}}{L_{ee}}$, illetve ha $J=0$, akkor $\eta=\frac{L_{eq}}{L_{ee}T}$.
   A Thomson-összefüggésbe helyettesítve $L_{eq}=L_{qe}$.
   \end{description}
